\documentclass{article}
\usepackage[utf8]{inputenc}
\usepackage[T1]{fontenc}
\usepackage{float}
\usepackage{caption}
\usepackage{subcaption}

\usepackage{graphicx}
\usepackage[top=1cm,bottom=2cm,right=1cm,left=1cm]{geometry}

\renewcommand{\emph}[1]{\textbf{#1}}

\title{2014 - 03 - 05}
\author{}
\date{}

\begin{document}
\maketitle
\section{Atualização do trabalho}

Seguindo o discutido na reunião, atualizei os testes separando aqueles que não
falham com Cholesky, e com categorias incluindo as variáveis de folga.
O total de problemas considerado foi 873 (todos os problemas sem variáveis
fixas).
Os problemas que tiveram falha de Cholesky foram 172.
Os problemas que não tiveram falha de Cholesky foram 701.

Outra coisa que fizemos para estes testes foi estabelecer um {\bf tempo
mínimo}. O algoritmo ALGENCAN usa duas casas decimais de precisão, de modo que


\begin{figure}[H]
  \centering
  \includegraphics[width=0.9\textwidth]{all_2014_02_20.pdf}
  \caption{Todos os problemas utilizados (Sem variáveis fixas) }
\end{figure}
\begin{figure}[H]
  \centering
  \includegraphics[width=0.9\textwidth]{cholfail_2014_02_20.pdf}
  \caption{Falha de Cholesky}
\end{figure}
\begin{figure}[H]
  \centering
  \includegraphics[width=0.9\textwidth]{cholok_2014_02_20.pdf}
  \caption{Sem as falhas de Cholesky}
\end{figure}

\newpage
\subsection{Resumo}

Como tem muita coisa pra ver, e muitos gráficos para considerar, vamos colocar
um resumo com o gráfico mais importante de cada categoria.
O gráfico mostrado é para o caso em que o número de variáveis é maior que 100 e
para qualquer quantidade \emph{positiva} de restrições, isto é, os problemas com
restrição. A maneira de se medir o número de variáveis pode ser a maneira
normal, ou utilizando as variáveis de folga, isto é, o número de variável é na
verdade o número de variáveis mais as restrições de desigualdade.

Separamos oe problemas em \emph{com} e \emph{sem} variáveis de folga, e
considerando os problemas em que Cholesky funciona separadamente.
\begin{figure}[H]
  \centering
  \begin{subfigure}{0.48\textwidth}
    \includegraphics[width=\textwidth]{vargt1e2_notunc_slackfalse_2014_02_20.pdf}
    \caption{Sem variáveis de folga}
  \end{subfigure}
  \begin{subfigure}{0.48\textwidth}
    \includegraphics[width=\textwidth]{cholok_vargt1e2_notunc_slackfalse_2014_02_20.pdf}
    \caption{Sem variáveis de folga e Cholesky funciona}
  \end{subfigure}
  \begin{subfigure}{0.48\textwidth}
    \includegraphics[width=\textwidth]{vargt1e2_notunc_slacktrue_2014_02_20.pdf}
    \caption{Com variáveis de folga}
  \end{subfigure}
  \begin{subfigure}{0.48\textwidth}
    \includegraphics[width=\textwidth]{cholok_vargt1e2_notunc_slacktrue_2014_02_20.pdf}
    \caption{Com variáveis de folga e Cholesky funciona}
  \end{subfigure}
\end{figure}

Mais detalhes nas subseções a seguir.

\newpage
\subsection{Sem variáveis de folga}

Da maneira tradicionais, separamos os problemas por número de variáveis e de
restrições. A separação, ainda sem utilizar variáveis de folga dá a tabela
\begin{center}
\begin{tabular}{|l|c|c|c|} \hline
& $ncon \leq 10^2$ & $ncon \in (10^2,10^4]$ & $ncon > 10^4$ \\ \hline
$nvar \leq 10^2$       & 445 &  36 &  2 \\ \hline
$nvar \in (10^2,10^4]$ & 139 & 200 &  0 \\ \hline
$nvar > 10^4$          &  14 &  19 & 18 \\ \hline
\end{tabular}
\end{center}

\begin{figure}[H]
\centering
\includegraphics[width=0.3\textwidth]{var1e0-1e2_con1e0-1e2_slackfalse_2014_02_20.pdf}
\includegraphics[width=0.3\textwidth]{var1e0-1e2_con1e2-1e4_slackfalse_2014_02_20.pdf}
\includegraphics[width=0.3\textwidth]{var1e0-1e2_con1e4-1e6_slackfalse_2014_02_20.pdf}
\caption{ $nvar \leq 10^2$. }
\label{fig:nvar_small}
\end{figure}
\begin{figure}[H]
\centering
\includegraphics[width=0.3\textwidth]{var1e2-1e4_con1e0-1e2_slackfalse_2014_02_20.pdf}
\includegraphics[width=0.3\textwidth]{var1e2-1e4_con1e2-1e4_slackfalse_2014_02_20.pdf}
\includegraphics[width=0.3\textwidth]{var1e2-1e4_con1e4-1e6_slackfalse_2014_02_20.pdf}
\caption{ $nvar \in (10^2,10^4]$. }
\label{fig:nvar_medium}
\end{figure}
\begin{figure}[H]
\centering
\includegraphics[width=0.3\textwidth]{var1e4-1e6_con1e0-1e2_slackfalse_2014_02_20.pdf}
\includegraphics[width=0.3\textwidth]{var1e4-1e6_con1e2-1e4_slackfalse_2014_02_20.pdf}
\includegraphics[width=0.3\textwidth]{var1e4-1e6_con1e4-1e6_slackfalse_2014_02_20.pdf}
\caption{ $nvar > 10^4$. }
\label{fig:nvar_big}
\end{figure}

\begin{figure}[H]
\centering
\includegraphics[height=0.44\textheight]{vargt1e2_congt1e2_slackfalse_2014_02_20.pdf}
\caption{ $nvar > 10^2$ e $ncon > 10^2$ }
\label{fig:both_medium_and_big}
\end{figure}

\begin{figure}[H]
\centering
\includegraphics[height=0.44\textheight]{vargt1e2_conany_slackfalse_2014_02_20.pdf}
\caption{ $nvar > 10^2$ }
\label{fig:nvar_medium_and_big}
\end{figure}

\begin{figure}[H]
\centering
\includegraphics[height=0.44\textheight]{vargt1e2_notunc_slackfalse_2014_02_20.pdf}
\caption{ $nvar > 10^2$ e apenas problemas com restrições }
\label{fig:nvar_medium_and_big_not_unc}
\end{figure}

\newpage
\subsection{Sem variáveis de folga. Sem falhas de cholesky}

Ainda separando da maneira tradicional, mas agora removendo os problemas que
falharam o Cholesky.
\begin{center}
\begin{tabular}{|l|c|c|c|} \hline
& $ncon \leq 10^2$ & $ncon \in (10^2,10^4]$ & $ncon > 10^4$ \\ \hline
$nvar \leq 10^2$       & 401 &  23 &  1 \\ \hline
$nvar \in (10^2,10^4]$ & 134 & 100 &  0 \\ \hline
$nvar > 10^4$          &  14 &  19 &  9 \\ \hline
\end{tabular}
\end{center}

\begin{figure}[H]
\centering
\includegraphics[width=0.3\textwidth]{cholok_var1e0-1e2_con1e0-1e2_slackfalse_2014_02_20.pdf}
\includegraphics[width=0.3\textwidth]{cholok_var1e0-1e2_con1e2-1e4_slackfalse_2014_02_20.pdf}
\includegraphics[width=0.3\textwidth]{cholok_var1e0-1e2_con1e4-1e6_slackfalse_2014_02_20.pdf}
\caption{ $nvar \leq 10^2$. }
\label{fig:nvar_small}
\end{figure}
\begin{figure}[H]
\centering
\includegraphics[width=0.3\textwidth]{cholok_var1e2-1e4_con1e0-1e2_slackfalse_2014_02_20.pdf}
\includegraphics[width=0.3\textwidth]{cholok_var1e2-1e4_con1e2-1e4_slackfalse_2014_02_20.pdf}
\includegraphics[width=0.3\textwidth]{cholok_var1e2-1e4_con1e4-1e6_slackfalse_2014_02_20.pdf}
\caption{ $nvar \in (10^2,10^4]$. }
\label{fig:nvar_medium}
\end{figure}
\begin{figure}[H]
\centering
\includegraphics[width=0.3\textwidth]{cholok_var1e4-1e6_con1e0-1e2_slackfalse_2014_02_20.pdf}
\includegraphics[width=0.3\textwidth]{cholok_var1e4-1e6_con1e2-1e4_slackfalse_2014_02_20.pdf}
\includegraphics[width=0.3\textwidth]{cholok_var1e4-1e6_con1e4-1e6_slackfalse_2014_02_20.pdf}
\caption{ $nvar > 10^4$. }
\label{fig:nvar_big}
\end{figure}

\begin{figure}[H]
\centering
\includegraphics[height=0.44\textheight]{cholok_vargt1e2_congt1e2_slackfalse_2014_02_20.pdf}
\caption{ $nvar > 10^2$ e $ncon > 10^2$ }
\label{fig:both_medium_and_big}
\end{figure}

\begin{figure}[H]
\centering
\includegraphics[height=0.44\textheight]{cholok_vargt1e2_conany_slackfalse_2014_02_20.pdf}
\caption{ $nvar > 10^2$ }
\label{fig:nvar_medium_and_big}
\end{figure}

\begin{figure}[H]
\centering
\includegraphics[height=0.44\textheight]{cholok_vargt1e2_notunc_slackfalse_2014_02_20.pdf}
\caption{ $nvar > 10^2$ e apenas problemas com restrições }
\label{fig:nvar_medium_and_big_not_unc}
\end{figure}

\newpage
\subsection{Com variáveis de folga}

Agora separando todos os problemas sem variáveis fixas utilizando as variáveis
de folga, isto é, $nvar$ é na verdade $nvar+ncon_I$.
\begin{center}
\begin{tabular}{|l|c|c|c|} \hline
& $ncon \leq 10^2$ & $ncon \in (10^2,10^4]$ & $ncon > 10^4$ \\ \hline
$nvar \leq 10^2$       & 438 &   1 &  0 \\ \hline
$nvar \in (10^2,10^4]$ & 142 & 221 &  0 \\ \hline
$nvar > 10^4$          &  18 &  33 & 20 \\ \hline
\end{tabular}
\end{center}

\begin{figure}[H]
\centering
\includegraphics[width=0.3\textwidth]{var1e0-1e2_con1e0-1e2_slacktrue_2014_02_20.pdf}
\includegraphics[width=0.3\textwidth]{var1e0-1e2_con1e2-1e4_slacktrue_2014_02_20.pdf}
\includegraphics[width=0.3\textwidth]{var1e0-1e2_con1e4-1e6_slacktrue_2014_02_20.pdf}
\caption{ $nvar \leq 10^2$. }
\label{fig:nvar_small}
\end{figure}
\begin{figure}[H]
\centering
\includegraphics[width=0.3\textwidth]{var1e2-1e4_con1e0-1e2_slacktrue_2014_02_20.pdf}
\includegraphics[width=0.3\textwidth]{var1e2-1e4_con1e2-1e4_slacktrue_2014_02_20.pdf}
\includegraphics[width=0.3\textwidth]{var1e2-1e4_con1e4-1e6_slacktrue_2014_02_20.pdf}
\caption{ $nvar \in (10^2,10^4]$. }
\label{fig:nvar_medium}
\end{figure}
\begin{figure}[H]
\centering
\includegraphics[width=0.3\textwidth]{var1e4-1e6_con1e0-1e2_slacktrue_2014_02_20.pdf}
\includegraphics[width=0.3\textwidth]{var1e4-1e6_con1e2-1e4_slacktrue_2014_02_20.pdf}
\includegraphics[width=0.3\textwidth]{var1e4-1e6_con1e4-1e6_slacktrue_2014_02_20.pdf}
\caption{ $nvar > 10^4$. }
\label{fig:nvar_big}
\end{figure}

\begin{figure}[H]
\centering
\includegraphics[height=0.44\textheight]{vargt1e2_congt1e2_slacktrue_2014_02_20.pdf}
\caption{ $nvar > 10^2$ e $ncon > 10^2$ }
\label{fig:both_medium_and_big}
\end{figure}

\begin{figure}[H]
\centering
\includegraphics[height=0.44\textheight]{vargt1e2_conany_slacktrue_2014_02_20.pdf}
\caption{ $nvar > 10^2$ }
\label{fig:nvar_medium_and_big}
\end{figure}

\begin{figure}[H]
\centering
\includegraphics[height=0.44\textheight]{vargt1e2_notunc_slacktrue_2014_02_20.pdf}
\caption{ $nvar > 10^2$ e apenas problemas com restrições }
\label{fig:nvar_medium_and_big_not_unc}
\end{figure}

\newpage
\subsection{Com variáveis de folga. Sem falhas de cholesky}

Agora apenas os problemas em que Cholesky funciona, utilizando as variáveis de
folga.
\begin{center}
\begin{tabular}{|l|c|c|c|} \hline
& $ncon \leq 10^2$ & $ncon \in (10^2,10^4]$ & $ncon > 10^4$ \\ \hline
$nvar \leq 10^2$       & 396 &   0 &  0 \\ \hline
$nvar \in (10^2,10^4]$ & 135 & 109 &  0 \\ \hline
$nvar > 10^4$          &  18 &  33 & 10 \\ \hline
\end{tabular}
\end{center}

\begin{figure}[H]
\centering
\includegraphics[width=0.3\textwidth]{cholok_var1e0-1e2_con1e0-1e2_slacktrue_2014_02_20.pdf}
\includegraphics[width=0.3\textwidth]{cholok_var1e0-1e2_con1e2-1e4_slacktrue_2014_02_20.pdf}
\includegraphics[width=0.3\textwidth]{cholok_var1e0-1e2_con1e4-1e6_slacktrue_2014_02_20.pdf}
\caption{ $nvar \leq 10^2$. }
\label{fig:nvar_small}
\end{figure}
\begin{figure}[H]
\centering
\includegraphics[width=0.3\textwidth]{cholok_var1e2-1e4_con1e0-1e2_slacktrue_2014_02_20.pdf}
\includegraphics[width=0.3\textwidth]{cholok_var1e2-1e4_con1e2-1e4_slacktrue_2014_02_20.pdf}
\includegraphics[width=0.3\textwidth]{cholok_var1e2-1e4_con1e4-1e6_slacktrue_2014_02_20.pdf}
\caption{ $nvar \in (10^2,10^4]$. }
\label{fig:nvar_medium}
\end{figure}
\begin{figure}[H]
\centering
\includegraphics[width=0.3\textwidth]{cholok_var1e4-1e6_con1e0-1e2_slacktrue_2014_02_20.pdf}
\includegraphics[width=0.3\textwidth]{cholok_var1e4-1e6_con1e2-1e4_slacktrue_2014_02_20.pdf}
\includegraphics[width=0.3\textwidth]{cholok_var1e4-1e6_con1e4-1e6_slacktrue_2014_02_20.pdf}
\caption{ $nvar > 10^4$. }
\label{fig:nvar_big}
\end{figure}

\begin{figure}[H]
\centering
\includegraphics[height=0.44\textheight]{cholok_vargt1e2_congt1e2_slacktrue_2014_02_20.pdf}
\caption{ $nvar > 10^2$ e $ncon > 10^2$ }
\label{fig:both_medium_and_big}
\end{figure}

\begin{figure}[H]
\centering
\includegraphics[height=0.44\textheight]{cholok_vargt1e2_conany_slacktrue_2014_02_20.pdf}
\caption{ $nvar > 10^2$ }
\label{fig:nvar_medium_and_big}
\end{figure}

\begin{figure}[H]
\centering
\includegraphics[height=0.44\textheight]{cholok_vargt1e2_notunc_slacktrue_2014_02_20.pdf}
\caption{ $nvar > 10^2$ e apenas problemas com restrições }
\label{fig:nvar_medium_and_big_not_unc}
\end{figure}

\end{document}
