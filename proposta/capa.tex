\begin{center}
  \Large \bf \titulo
\end{center}
\begin{center}
  \autor \\
  Supervisor: \orientador
\end{center}
\begin{abstract}
\iflanguage{portuguese}{
O método de Controle Dinâmico da Inviabilidade,
recentemente estendido no doutorado em progresso de Abel Soares Siqueira
para incluir restrições de desigualdade, mostrou-se um competidor
promissor entre os métodos para programação não linear.
A implementação do novo método, denominada \emph{DCICPP}, foi
comparada com o método IPOPT 
e obteve um bom desempenho.
O DCICPP já consegue encontrar eficientemente a solu\c{c}\~{a}o
de 698 problemas do CUTEr em 2 horas.
Neste projeto
pretendemos estudar maneiras de aumentar ainda mais a
robustez do {DCICPP}, avaliando
outras estratégias para resolver os subproblemas.
Tamb\'{e}m queremos melhorar aspectos pr\'{a}ticos do algoritmo,
por exemplo,
implementar uma maneira
de lidar com as variáveis fixas no problema e avaliar possibilidades de
lidar com Jacobianas mal condicionadas.
Esse projeto é uma extensão da tese de doutorado de Abel Soares Siqueira, 
da Matemática Aplicada, pelo IMECC, com previsão de defesa para Novembro de
2013, e está vinculado ao projeto temático 2013/05475-7.
}
{
The Dynamic Control of Infeasibility method, recently expanded in the Ph.D.
in progress of Abel Soares Siqueira to include inequality constraints, proved to
be a good choice amongst the methods for nonlinear programming. The method's
implementation, called DCICPP, was compared to the IPOPT method and
achieve
a good performance. The DCICPP method can already find efficiently the solution
of 698 problem of CUTEr in 2 hours.
In this project, we intend to study ways of increasing even more the strength of
the DCICPP method, considering other strategies to solve the subproblems. We also
want to improve the practical aspects of the algorithm, for instance,
accepting fixed variables in the problem, and pondering
possibilities to handle ill-conditioned Jacobian matrices.
This project is an extension of the Ph.D. of Abel Soares Siqueira, in Applied
Mathematics, from the IMECC, with
defense
expected in November, 2013, and is tied to the thematic projetc 2013/05475-7.
}

\vspace{1 cm}
{\noindent\bf Keywords:} Nonlinear Programming, Constrained Opmitization,
Composite-step Methods.
\end{abstract}
