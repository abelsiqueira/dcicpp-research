\documentclass{article}
\usepackage[utf8]{inputenc}
\usepackage[T1]{fontenc}
\usepackage{float}
\usepackage{caption}
\usepackage{subcaption}
\usepackage[numbers,sort&compress]{natbib}

\usepackage{graphicx}
\usepackage[top=1cm,bottom=2cm,right=1cm,left=1cm]{geometry}

\renewcommand{\emph}[1]{\textbf{#1}}

\title{2014 - 09 - 23}
\author{}
\date{}

\newcommand{\ftol}{f_{\mbox{tol}}}
\newcommand{\norma}[1]{\Vert{#1}\Vert}

\begin{document}
\maketitle
\section{Atualização do trabalho}

O nosso passo normal interno é uma combinação da direção de Cauchy com a
direção de Gauss-Newton para o problema
$$ \min \norma{Jd + h}^2, $$
onde $J$ é a Jacobiana (não escalada) das restrições e $h$ é o valor da
restrições no ponto atual.
Esse passo é aceito se o decréscimo real for até uma certa fração do decréscimo
previsto.

Além disso, se o decréscimo do passo não for suficiente, no sentido de que a
norma do vetor de restrições no ponto encontrado não é suficientemente menor que
a norma anterior, consideramos a iteração uma falha. Se houverem \texttt{nfailv}
falhas consecutivas, verificamos se o ponto atual satisfaz o critério de
ponto estacionário da medida de infactibilidade. Se o ponto não for
estacionário, fazemos um passo diferente na tentativa de salvar a iteração.

Se o parâmetro \verb+use_normal_safe_guard+ for 1, fazemos um passo do tipo
$$ \delta = -\alpha AA^Th, $$
onde $A$ é a matriz Jacobiana escalada.

Se o parâmetro \verb+normal_fail_reboot+ for 1, e o ponto continuar fora do
cilindro após o \emph{safe guard} anterior, fazemos um reinicío das variáveis de
folga. Para cada restrição de desigualdade $c_i$ com limitantes $\ell_i$ e
$u_i$, e dado o ponto atual $x_c$, se $\ell_i < c_i(x) < u_i$, então a variável
de folga $s$ associada é definida como $s = c_i(x)$, de modo que $h_i(z) = 0$.

Mudando alguns dos parâmetros, usando 5 falhas como limite, obtivemos
convergência em 118 dos 126 problemas HS* da coleção CUTEst.
Os outros problemas foram
\begin{description}
  \item[HS54] Falha com número máximo de iterações. Não obtém factibilidade
    dual. Deixando mais iterações piora o algoritmo. Resolve se usa o
    escalamento das restrições.
  \item[HS87] Falha com número máximo de iterações. Não obtém factibilidade
    dual. Mais iterações não ajudam.
  \item[HS88] Ponto estacionário para a infactibilidade. Se não utilizar a
    salvaguarda normal, converge. Condicionamento péssimo.
  \item[HS89] Mesmo caso do HS88, sem problema no condicionamento, mas a matriz
    é da ordem de $10^{-16}$.
  \item[HS90] Ponto estacionário para a infactibilidade. Se usar \verb+eta3+
    menor ou igual a 0.2, converge. Matriz da ordem de $10^{-12}$.
  \item[HS92] Ponto estacionário para a infactibilidade. Matriz da ordem de
    $10^{-12}$.
  \item[HS99EXP] Número máximo de iterações. Condicionamento $10^6$.
  \item[HS109] Rhomax muito pequeno. Condicionamento péssimo.
\end{description}

\subsection{O escalamento das restrições}

O escalamento das restrições deveria ser feito de maneira a ajudar aqueles
problemas que precisam de escalamento sem danificar os problemas que já
funcionam bem. No entanto não é isso que está acontecendo. Para os problemas HS*
do repositório CUTEst, 110 de 126 problemas convergem. Em comparação ao caso sem
escalamento, 10 problemas deixam de convergir e 4 passam a convergir.
Um desses problemas é o HS32.

\bibliographystyle{siam}
\bibliography{bib}

\end{document}
