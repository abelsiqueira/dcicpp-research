\section{Converg\^encia Global}\visiblelbl{sec:conv-global}

Apresentamos aqui os resultados de convergência global do método. Veremos que,
com condições relativamente fracas, podemos obter resultados satisfatórios de
convergência. A seguir estão as nossas hipóteses para a prova de convergência do
Algoritmo \ref{alg:outline}.
\begin{hypoenv}\visiblelbl{hip:global.continuidade} 
  As funções $f$, $c_E$ e $c_I$ s\~ao
$C^2$.  
\end{hypoenv} 
\begin{hypoenv}\visiblelbl{hip:global.seq.limitadas} 
  As sequ\^encias $\{\zc{k}\}$ e $\{z^k\}$, as aproxima\c{c}\~oes $B^k$ e os
  multiplicadores $\{\lk{k}\}$ permanecem uniformemente limitados.
\end{hypoenv} 
\begin{hypoenv}\visiblelbl{hip:global.regularidade} A
  restaura\c{c}\~ao n\~ao falha, no sentido de que o algoritmo normal sempre
  encontra um ponto dentro do cilindro, e $\mathcal{Z} = \{\zc{k}\}$ permanece
  longe do conjunto singular de $h$, no sentido que $h$ \'e regular no fecho de
  $\mathcal{Z}$.  Al\'em disso, se a sequ\^encia gerada $\{\zc{k}\}$ \'e
  infinita, ent\~ao 
\begin{equation}\visiblelbl{nnormal} 
  \norma{\zc{k+1}-z^k} = \bigo(\norma{h(z^k)}).
\end{equation} 
\end{hypoenv}
\begin{hypoenv}\visiblelbl{hip:global.dsoc} 
  $\norma{\dsoc^k} = \bigo(\norma{\delta_t^k}^2)$ 
\end{hypoenv} 
A Hipótese \ref{hip:global.continuidade} é natural e esperada, já que canônica,
considerando que, no método, utilizamos essas derivadas, ou aproximações para as
mesmas. 
A Hipótese \ref{hip:global.seq.limitadas} é necessária pois podemos tomar alguma
direção de descida ilimitada. 
As aproximações podem ser definidas de modo a permanecerem
uniformemente limitadas. 
A Hipótese \ref{hip:global.regularidade} é importante pois a restauração pode
falhar. Note que na Seção \ref{sec:conv-infactivel} comentamos sobre a
convergência do método para pontos estacionários da infactibilidade.
A Hipótese \ref{hip:global.dsoc} é tradicional para os passos de correção de
segunda ordem.

Apresentamos agora algumas propriedades provenientes do algoritmo.
Começamos por aquelas relacionadas aos cilindros.
Pela maneira que calculamos os raios dos cilindros no passo
\ref{alg:normal-call} do Algoritmo \ref{alg:outline}, pelas condições dos
iterandos nos passos \ref{alg:normal-call} e \ref{alg:tangente-call} do Algoritmo
\ref{alg:outline}, e pela maneira que atualizamos $\rhomax$ nos passos
\ref{alg:update-rhomax-start}-\ref{alg:update-rhomax-end} do Algoritmo
\ref{alg:update.rhomax}, temos
\begin{eqnarray}
  \rho^k \quad \leq & \rhomax^{k-1}\norma{g_p^k} & \leq \quad
    2\rhomax^k\norma{g_p^k} \visiblelbl{limrho}, \\
  \rhomax^k \quad \leq & \rhomax^{k-1} & \leq \quad
    10^4\rho^k\frac{\norma{g(\zc{k},\mu^k)} + 1} {\norma{\gpk{k}}},
    \visiblelbl{limrhomax} \\
  \norma{h(\zc{k})} \quad \leq & \norma{h(z^{k-1})} & \leq \quad 2\rho^{k-1}.
    \visiblelbl{limhhornormal}
\end{eqnarray}
Como indicado previamente em \eqref{eq:slim} e visto no algoritmo, nossas
iterações seguem uma regra de fração-para-a-fronteira, de modo que são válidas
as inequações
\begin{eqnarray}\visiblelbl{zbound}
 \begin{array}{rcl}
  s_c^{k} & \geq & \epsmu s^{k-1}, \\
  s^{k} & \geq & \epsmu s_c^{k}, \\
  s^+ & \geq & \epsmu s_c^k.
 \end{array}
\end{eqnarray}
Alem disso, pela definição de $\mu^k$ no passo \ref{alg:mudef} do Algoritmo
\ref{alg:etapa.normal}, temos
\begin{equation}\visiblelbl{limmu}
  \mu^k \leq \alpha_{\rho}\min\{\rho^k,(\rho^k)^2\}.
\end{equation}
Deste ponto em diante, supomos que as sequências $\{\zc{k}\}$ e $\{z^k\}$,
geradas pelo algoritmo, satisfazem
\ref{hip:global.continuidade}-\ref{hip:global.dsoc}. Denotando por $\delta_N^k$
o passo normal e por $\delta_T^k$ o passo tangente na iteração $k$, temos
\begin{equation*} \delta_N^k = z_c^k - z^{k-1} \qquad \mbox{e} \qquad \delta_T^k
= z^k - \zc{k} = \delta_t^k + \dsoc^k.  \end{equation*} As hipóteses
\ref{hip:global.continuidade}-\ref{hip:global.dsoc} nos permitem escolher uma
constante $\dmax > 0$, tal que, para todo $k$,
\begin{equation}\visiblelbl{limdmax} 
  \norma{\delta_t^k} + \norma{\dsoc^k} + \norma{\delta_N^k} \leq \dmax.  
\end{equation} As
hipóteses tamb\'em nos permitem definir $\xi_0 > 0$, tal que, $\forall k$, se
$\norma{z - \zc{k}} \leq \dmax$ e $\mu \leq \mu_{0}$, ent\~ao 
\begin{eqnarray}
  \norma{A_j(z)} & \leq & \xi_0, \qquad j = 1,\dots,m \visiblelbl{limjacob} \\
  \norma{\nabla^2 h_j(z)} & \leq & \xi_0, \qquad j = 1,\dots,m
  \visiblelbl{limhessh} \\
  \norma{\nabla f(x)} & \leq & \xi_0, \visiblelbl{limgrad} \\
  \norma{\nabla^2 f(x)} & \leq & \xi_0, \visiblelbl{limhess} \\
  \norma{g(z,\mu)} & \leq & \xi_0, \visiblelbl{limgamma} \\
  \norma{\Gamma(z,\mu)} & \leq & \xi_0, \visiblelbl{limGamma} \\
  \norma{B^k} & \leq & \xi_0, \visiblelbl{limB} \\
  \norma{\lk{k}} & \leq & \xi_0, \visiblelbl{limlambda} \\
  \norma{\dsoc^k} & \leq & \xi_0\norma{\delta_t^k}^2 \visiblelbl{limsoc} \\
  \norma{\Lac{k}} & \leq & \xi_0 \visiblelbl{limLac}, 
\end{eqnarray}
  \begin{eqnarray}\visiblelbl{zcompact}
\begin{array}{rcl} s_c^{k} & \leq & \xi_0 s^{k-1}, \\
  s^k & \leq & \xi_0 s_c^{k}. 
\end{array} 
\end{eqnarray} 
Também supomos que (\ref{nnormal}) pode
ser reescrito como 
\begin{equation}\visiblelbl{nnormalxi} 
  \norma{\zc{k+1} - z^k}
\leq \xi_0\norma{h(z^k)}.  
\end{equation} 
Definimos as matrizes $\Lac{k} = \Lambda(\zc{k})$, $\Lak{k} = \Lambda(z^k)$ e
$\Lambda^+ = \Lambda(\zp)$, para facilitar a notação. 
  Nosso resultado de convergência global é dado no Teorema
  \ref{teo:conv-global}. Esse resultado depende dos próximos cinco lemas. O
  Lema \ref{lemma:31} a seguir dá um limitante para o aumento da infactibilidade
  do causada pelo passo tangente em relação ao iterando obtido no passo normal.

\begin{lemma}\visiblelbl{lemma:31} 
  A tentativa de itera\c{c}\~ao $\zp$ gerada na
  linha \ref{alg:tangente-plus} do Algoritmo \ref{alg:etapa.tangente} satisfaz
  \begin{equation}\visiblelbl{limvarh} 
    \norma{h(\zp) - h(\zc{k})} \leq \bxi_0\norma{\delta_t}^2,
  \end{equation} 
  em que $\barra{\xi}_0$ é uma constante positiva.
\end{lemma} 
\begin{proof} Iremos
  omitir os \'indices $k$ nessa demonstra\c{c}\~ao. A itera\c{c}\~ao \'e
  definida por $\zp = z_c + \dplus = z_c + \Lambda_c(\delta_t + \dsoc)$.  Usando
  uma expansão de Taylor, o fato de que $A_j(z_c)\delta_t = 0$ e
  (\ref{limhessh}), podemos garantir que existe $\zxi^j = \eta_j\zp +
  (1-\eta_j)z_c$, tal que
\begin{eqnarray} 
  \modulo{h_j(\zp) - h_j(z_c)} & = & \modulo{\nabla h_j(z_c)^T\dplus +
    \meio(\dplus)^T\hess h_j(\zxi^j)\dplus} \nonumber \\
  & = & \modulo{\nabla h_j(z_c)^T\Lac{k}(\dt+\dsoc) +
      \meio(\delta_t+\dsoc)^T\Lac{k}\hess h_j(\zxi^j)\Lac{k}(\delta_t +
    \dsoc)} \nonumber \\
  & \leq & \modulo{A_j(z_c)^T(\delta_t+\dsoc)} +
    \meio\norma{\delta_t+\dsoc}^2\norma{\Lac{k}}^2\norma{\hess h_j(\zxi^j)}
    \nonumber \\
  & \leq & \modulo{A_j(z_c)^T\dsoc} + \meio\norma{\delta_t+\dsoc}^2\xi_0^3.
    \label{lemma31.aux}
\end{eqnarray}
Agora, por \eqref{limjacob} e \eqref{limsoc}, temos
$$ \modulo{A_j(z_c)^T\dsoc} \leq \xi_0^2\norma{\delta_t}^2, $$
e pela desigualdade $\norma{v+w}^2 \leq 2(\norma{v}^2 + \norma{w}^2)$,
\eqref{limsoc} e \eqref{limdmax},
$$ \meio\norma{\delta_t + \dsoc}^2 \leq \norma{\delta_t}^2 + \norma{\dsoc}^2
\leq \norma{\delta_t}^2 + \dmax\xi_0\norma{\delta_t}^2. $$
Substituindo essas duas desigualdades em \eqref{lemma31.aux}, obtemos
\begin{eqnarray*}
 \modulo{h_j(\zp) - h_j(z_c)} 
  & \leq & \xi_0^2\norma{\delta_t}^2 + \xi_0^3(\norma{\delta_t}^2 +
    \xi_0\dmax\norma{\delta_t}^2) \\
  & \leq & \xi_0^2(1 + \xi_0 + \xi_0^2\dmax)\norma{\delta_t}^2.
\end{eqnarray*} 
  Definindo $\bxi_0 = \sqrt{m}\xi_0^2(1 + \xi_0 + \xi_0^2\dmax)$, temos o
  resultado desejado.
\end{proof}

O lema a seguir mostra que, se as hipóteses
\ref{hip:global.continuidade}-\ref{hip:global.dsoc} são satisfeitas, o passo
tangente não falha, e obtemos redução suficiente no Lagrangeano.
\begin{lemma}\visiblelbl{lemma:32} Se $x_c^k$ não é um ponto estacionário para o
  problema \eqref{prob:geral}, então $\zp$ \'e aceito com suficientes iterações.
  Al\'em disso,
  podemos definir constantes $\xi_1$, $\xi_2$ e $\xi_3$ tais que, para todo $k$,
\begin{equation}\visiblelbl{limDLHp} \DHp \leq
-\xi_1\norma{\gpk{k}}\min\{\xi_2\norma{\gpk{k}},\xi_3\sqrt{\rho^k},1-\epsmu\}.
\end{equation} \end{lemma} \begin{proof} Iremos omitir os \'indices $k$ nessa
  demonstra\c{c}\~ao. Seja $\zp = z_c + \dplus = z_c + \Lambda_c(\dt + \dsoc)$
  um candidato obtido na linha \ref{alg:tangente-plus} da $k$-\'esima
  itera\c{c}\~ao do Algoritmo \ref{alg:outline}.

  Utilizando uma expans\~ao de Taylor e \eqref{limvarh}, existe $z_\xi = \eta
  z^+ + (1-\eta)z_c$, para algum $\eta\in[0,1]$, tal que
\begin{eqnarray} 
  \DHp & = & L(\zp,\lambda,\mu) - L(z_c,\lambda,\mu) \nonumber \\
   & = & \varphi(\zp,\mu) - \varphi(z_c,\mu) + \lambda^T[h(\zp) - h(z_c)]
  \nonumber \\
  & \leq & \nabla \varphi(z_c,\mu)^T\dplus + \meio(\dplus)^T\hess
  \varphi(\zxi,\mu)\dplus + \xi_0\bxi_0\norma{\dt}^2. \visiblelbl{dlh:passo1}
\end{eqnarray}
Para o primeiro termo, usando \eqref{def:gradiente_escalado}, o fato de que
$g(z_c,\mu)^T\delta_t = (g_p^k)^T\delta_t$, a definição \eqref{prob:tangente},
e as condições \eqref{limsoc}, \eqref{limgamma} e
\eqref{limB}, temos
\begin{align}
  \nabla\varphi(z_c,\mu)^T\delta^+ & = \nabla\varphi(z_c,\mu)^T\Lambda_c
  (\delta_t+\dsoc) \nonumber \\
  & = g(z_c,\mu)^T(\delta_t + \dsoc) \nonumber \\
  & = q(\delta_t) - \meio\delta_t^TB\delta_t + g(z_c,\mu)^T\dsoc \nonumber \\
  & \leq q(\delta_t) + \bigg(\meio\xi_0 + \xi_0^2\bigg)\norma{\delta_t}^2.
  \visiblelbl{dlh:passo2}
\end{align}
Para o segundo termo de \eqref{dlh:passo1}, usando \eqref{def:hess_escalada},
\eqref{limGamma}, \eqref{zbound}, a desigualdade $\norma{v+w}^2 \leq 2(\norma{v}^2
+ \norma{w}^2)$, \eqref{limdmax} e \eqref{limsoc}, temos
\begin{align}
  \meio(\delta^+)^T\nabla^2\varphi(z_\xi,\mu)\delta^+ & =
    \meio(\delta_t+\dsoc)^T\Lambda_c\Lambda(z_{\xi})^{-1}\Gamma(z_{\xi},\mu)
    \Lambda(z_{\xi})^{-1}\Lambda_c(\delta_t + \dsoc) \nonumber \\
  & \leq \dfrac{1}{2}\norma{\Gamma(z_{\xi},\mu)} \norma{\Lambda(z_{\xi})^{-1}
    \Lambda_c}^2 \norma{\delta_t+\dsoc}^2  \nonumber \\
  & \leq \dfrac{1}{2} \dfrac{\xi_0}{\epsmu^2} \norma{\delta_t + \dsoc}^2
    \nonumber \\
  & \leq \frac{\xi_0}{\epsmu^2} (\norma{\delta_t}^2 + \norma{\dsoc}^2) \nonumber
    \\
  & \leq \frac{\xi_0}{\epsmu^2} (1+\xi_0\dmax)\norma{\delta_t}^2.
    \visiblelbl{dlh:passo3}
\end{align}
Assim, substituindo \eqref{dlh:passo2} e \eqref{dlh:passo3} em
\eqref{dlh:passo1}, temos
\begin{align}
  \DHp & \leq q(\delta_t) + \bigg(\frac{\xi_0}{2}+\xi_0^2\bigg)\norma{\delta_t}^2
  + \frac{\xi_0}{\epsmu^2}(1+\xi_0\dmax)\norma{\delta_t}^2
  + \xi_0\barra{\xi_0}\norma{\delta_t}^2 \nonumber \\
  & = q(\delta_t) + \barra{\xi}_1\norma{\delta_t}^2, \visiblelbl{auxdhp}
\end{align} 
onde $\barra{\xi}_1 = \frac{\xi_0}{2} + \xi_0^2 + \frac{\xi_0}{\epsmu^2} (1 +
\xi_0\dmax) + \xi_0\barra{\xi}_0$.

Pela defini\c{c}\~ao de $\dcp$ na linha \ref{alg:cauchy-point} do Algoritmo
\ref{alg:etapa.tangente}, temos $$\norma{\dcp} \geq
\min\left\{\frac{\norma{g_p}}{\norma{B}},\Delta,1-\epsmu \right\} \geq
\min\left\{\frac{\norma{g_p}}{\xi_0}, \Delta, 1 - \epsmu \right\}$$ e
\begin{equation}\visiblelbl{limqdcp} q(\dcp) \leq \meio\dcp^Tg_p \leq
-\meio\norma{g_p}\min\left\{ \frac{\norma{g_p}}{\xi_0}, \Delta, 1 - \epsmu
\right\}.  \end{equation}

Vamos mostrar que $\zp$ \'e aceito se $\Delta \leq \barra{\Delta}$, onde
\begin{equation}\visiblelbl{bdelta} \barra{\Delta} =
\min\left\{\frac{\norma{g_p}}{4\bxi_1}, 1-\epsmu, \sqrt{\frac{\rho}
{\bxi_0}}\right\}.  \end{equation} Note que $\xi_0 < 4\bxi_1$, logo
\begin{equation}\visiblelbl{limbdelta} \barra{\Delta} \leq
\frac{\norma{g_p}}{\xi_0}.  \end{equation} Como $\Delta \leq \barra{\Delta}$,
usando (\ref{limbdelta}), podemos simplificar (\ref{limqdcp}) para
\begin{equation}\visiblelbl{limqdcp2} 
  q(\dcp) \leq -\meio\norma{g_p}\Delta.
\end{equation}
Combinando $\norma{\dt} \leq \Delta$ e $q(\dt) \leq q(\dcp)$ da linha
\ref{alg:delta-t} do Algoritmo \ref{alg:etapa.tangente}, com (\ref{bdelta}) e
(\ref{limqdcp2}) obtemos 
\begin{eqnarray} 
  \bxi_1\norma{\dt}^2 & \leq & \bxi_1\Delta^2 \leq \bxi_1\barra{\Delta}\Delta
  \nonumber \\ 
  & \leq & \frac{1}{4}\norma{g_p}\Delta \leq -\meio q(\dcp) \leq -\meio q(\dt).
\visiblelbl{limxi1} 
\end{eqnarray} 
Agora, de (\ref{auxdhp}) e (\ref{limxi1}),
temos \begin{equation}\visiblelbl{negDHp} \DHp \leq \meio q(\dt) < 0,
\end{equation} de modo que 
\begin{equation}\visiblelbl{rgeqeta1} 
  r = \frac{\DHp}{q(\dt)} \geq \meio \geq \eta_1.  
\end{equation}
Como $\Delta \leq \barra{\Delta}$ e $\norma{h(z_c)} \leq \rho$, usando
(\ref{limvarh}) e (\ref{bdelta}), temos $$\norma{h(\zp)} \leq \rho + \bxi_0
\norma{\dt}^2 \leq \rho + \bxi_0\barra{\Delta}^2 \leq 2 \rho.$$ Portanto, ambas
condi\c{c}\~oes da linha \ref{alg:tangente-condicoes} do Algoritmo
\ref{alg:etapa.tangente} s\~ao satisfeitas e $\zp$ \'e aceito.

Para provar a segunda parte, vamos lembrar que cada vez que o passo \'e
rejeitado, multiplicamos o raio $\Delta$ por $\alpha_R$. Ent\~ao podemos assumir
que o raio aceito satisfaz $\Delta \geq \alpha_R\barra{\Delta}$, onde $0 <
\alpha_R < 1$. Combinando isto com (\ref{negDHp}), a condição $q(\dt) \leq
q(\dcp)$, (\ref{limqdcp}), (\ref{limbdelta}) e (\ref{bdelta}), obtemos 
\begin{eqnarray*}
  \DHp & \leq & \meio q(\dt) \leq \meio q(\dcp) \\ 
   & \leq & -\frac{1}{4}\norma{g_p}\min\left\{ \frac{\norma{g_p}}{\xi_0},
    \Delta, 1 - \epsmu\right\} \\ 
  & \leq & - \frac{1}{4}\norma{g_p}\alpha_R\barra{\Delta} \\ 
  & \leq & -\xi_1\norma{g_p}\min\{\xi_2\norma{g_p}, \xi_3\sqrt{\rho}, 1 -
    \epsmu\}, 
\end{eqnarray*} 
com $\xi_1 = \alpha_R/4$, $\xi_2 = 1/(4\bxi_1)$ e
$\xi_3 = 1/\sqrt{\bxi_0}$.  \end{proof}

O próximo lema define um limitante superior para a variação normal do
Lagrangeano. Note que essa variação pode ser positiva.
\begin{lemma}\visiblelbl{lemma:33} 
  Existe uma constante positiva $\xi_4$ tal que, para $k$ suficientemente
  grande,
  $$\DLV{k+1} \leq \xi_4\rhomax^k\norma{\gpk{k}}.  $$ 
\end{lemma} 
\begin{proof}
    Usando o Teorema do Valor Médio, temos
\begin{eqnarray*} 
    \DLV{k+1} & = & L(\zc{k+1},\lk{k+1},\mu^{k+1}) - L(z^k,\lk{k},\mu^k) \\ 
    & = & \varphi(\zc{k+1},\mu^{k+1}) - \varphi(z^k,\mu^k) + h(\zc{k+1})^T\lk{k+1} -
      h(z^k)^T\lk{k} \\ 
    & = & \varphi(\zc{k+1},\mu^k) - \varphi(z^k,\mu^k) + h(\zc{k+1})^T\lk{k+1} -
      h(z^k)^T\lk{k} + (\mu^{k+1}-\mu^k)\beta(z_c^{k+1}) \\
    & = & \nabla \varphi (\zxi,\mu^k)^T(\zc{k+1}-z^k) + h(\zc{k+1})^T\lk{k+1} -
      h(z^k)^T\lk{k} + (\mu^{k+1}-\mu^k)\beta(z_c^{k+1}) \\
    & =  & \nabla f(x_{\xi})^T(x_c^{k+1}-x^k) +
      \mu^k\nabla\beta(z_{\xi})^T(z_c^{k+1}-z^k) + h(\zc{k+1})^T\lk{k+1} -
      h(z^k)^T\lk{k} \\
    & & + (\mu^{k+1}-\mu^k)\beta(z_c^{k+1}),
\end{eqnarray*} 
  com $\zxi = \eta z^k + (1-\eta)\zc{k+1}$, para algum $\eta \in [0,1]$. 
  Pela hipótese \ref{hip:global.seq.limitadas}, existe $M >
  0$ tal que $s_i^k,s_{c_i}^k \leq M$. Usando isso, \eqref{limgrad},
(\ref{zbound}), (\ref{limlambda}), (\ref{nnormalxi}), (\ref{limhhornormal}), (\ref{limmu}) e
(\ref{limrho}) temos 
\begin{eqnarray*} 
  \DLV{k+1} & \leq & \xi_0\norma{\xc{k+1}-x^k} + 
  \mu^k\sum_{i=1}^{m_I} \frac{s_i^k - s_{c_i}^{k+1}}{\eta s_i^k +
    (1-\eta)s_{c_i}^{k+1}} + 
    \xi_0\norma{h(\zc{k+1})} + \xi_0\norma{h(z^k)} + (\mu^k - \mu^{k+1})m_IM \\ 
  & \leq & \xi_0^2\norma{h(z^k)} + \xi_0\norma{h(z^k)} + \xi_0\norma{h(z^k)} +
    \mu^km_I\bigg(\frac{1 - \epsmu}{\eta + (1-\eta)\epsmu}+M\bigg) \\ 
  & \leq & (\xi_0^2 + 2\xi_0)2\rho^k + 
    \rho^km_I\bigg(\frac{1 - \epsmu}{\eta + (1-\eta)\epsmu}+M\bigg) \\ 
  & \leq & \xi_4\rhomax^k\norma{\gpk{k}}, 
\end{eqnarray*}
onde $\xi_4 = 4(\xi_0^2 + 2\xi_0) + 2m_I\dfrac{1-\epsmu}{\eta+(1-\eta)\epsmu}+2M$.  
\end{proof}

O Lema \ref{lemma:34} mostra que entre iterações sucessivas em que $\rhomax$ não
muda, o Lagrangeano decresce proporcionalmente à variação do Lagrangeano nos
passos tangentes.  \begin{lemma}\visiblelbl{lemma:34} Se
  $\rhomax^{k+1}=\rhomax^{k+2}=\dots=\rhomax^{k+j}$, para $j \geq 1$, ent\~ao
\begin{equation}\visiblelbl{difL} 
  L_c^{k+j}-L_c^k = \sum_{i = k+1}^{k+j}\Delta
L_c^i \leq \frac{1}{4}\sum_{i = k}^{k+j-1}\DLH{i} + r^k, 
\end{equation} 
onde
$r^k = \meio[\Lref^k - L_c^k]$.  \end{lemma} \begin{proof} Suponha que $\Lref$
  n\~ao muda entre as itera\c{c}\~oes $k+1$ e $k+j_1-1$, onde $0<j_1\leq j+1$.
  Neste caso, por (\ref{DLC}) e pelo crit\'erio da linha \ref{alg:condicao-dlv}
  do Algoritmo \ref{alg:update.rhomax}, temos 
\begin{equation}\visiblelbl{limaux1}
  L_c^{k+j_1-1} - L_c^k = \sum_{i = k+1}^{k+j_1-1}(\DLV{i}+\DLH{i-1}) \leq
  \meio\sum_{i = k}^{k + j_1-2}\DLH{i}.  
\end{equation} 
Por outro lado, se $\Lref$
muda na itera\c{c}\~ao $k+j_1$, ent\~ao a condi\c{c}\~ao da linha
\ref{alg:condicao-dlv} \'e satisfeita. Neste caso, como $\rhomax$ n\~ao muda
nesta itera\c{c}\~ao, de modo que a condição da linha
\ref{alg:update-rhomax-start} não \'e satisfeita, e como $\DLH{k} \leq 0$ para
todo $k$, temos
\begin{eqnarray} 
  L_c^{k+j_1}-L_c^k & \leq & \DLV{k+j_1}+L(z^{k+j_1-1}, \lk{k+j_1-1},
    \mu^{k+j_1-1}) - \Lref^k + [\Lref^k - L_c^k] \nonumber \\ 
  & \leq & \meio[L(z^{k+j_1-1}, \lk{k+j_1-1}, \mu^{k+j_1-1}) - \Lref^k] +
    [\Lref^k - L_c^k] \nonumber \\ 
  & \leq & \meio[\DLH{k+j_1-1}+L_c^{k+j_1-1} - L_c^k] + \meio[\Lref^k-L_c^k]
    \nonumber \\ 
  & \leq & \frac{1}{4}\sum_{i = k}^{k + j_1-1}\DLH{i} + r^k.
    \visiblelbl{limaux2} 
\end{eqnarray} 
Se $j_1 \geq j$,
ent\~ao (\ref{limaux1}) e (\ref{limaux2}) implicam (\ref{difL}).

Por outro lado, se $\Lref$ \'e atualizado nas itera\c{c}\~oes
$k+j_1,\dots,k+j_s$, onde $0<j_1<j_2<\dots<j_s\leq j$, ent\~ao $r^{k+j_1} =
r^{k+j_2} = \dots = r^{k+j_s} = 0$. Portanto, aplicando o mesmo processo
descrito acima v\'arias vezes, e definindo $j_0 = 0$, obtemos 
\begin{equation*}
  L_c^{k+j} - L_c^k = \sum_{i = 1}^s[L_c^{k+j_i} - L_c^{k+j_i-1}] + L_c^{k+j} -
  L_c^{k+j_s} \leq \frac{1}{4}\sum_{i = k}^{k + j-1}\DLH{i} + r^k .
\end{equation*}
\end{proof}

O próximo lema estabelece a existência de \emph{espaço normal suficiente} nos
cilindros de confiança para garantir que o Lagrangeano possa ter decréscimo
suficiente. A base desse lema é que a inequação \eqref{limDLHp} garante que,
assintoticamente, $\modulo{\Delta L_H^k}$ é maior que uma fração de
$\sqrt{\rho^k}$, enquanto que, na demonstração do Lema \ref{lemma:34}, vemos que
$\norma{\Delta L_V^k} = \bigo(\rho^k)$. Isso significa que, no limite, a
restauração não irá destruir o descréscimo do Lagrangeano, o que evita que sejam
feitas alterações excessivas de $\rhomax$.  
\begin{lemma}\visiblelbl{lemma:35} 
  Se CDI gera uma
  sequ\^encia infinita $\{z^k\}$, ent\~ao 
\begin{description} 
  \item[(i)] Existem constantes positivas $\xi_5$ e $\xi_6$ tais que, se
\begin{equation}\visiblelbl{hip:limrhomax} 
  \rhomax^k < \min\{\xi_5\norma{g_p(\zc{k},\mu^k)},\xi_6\}, 
\end{equation}
    então $\rhomax$ n\~ao muda na itera\c{c}\~ao $k + 1$.  
  \item[(ii)] Al\'em disso, se $\lim\inf \norma{\gpk{k}} > 0$, ent\~ao existe
    $k_0 > 0$ tal que, para todo $k \geq k_0$
\begin{equation}\visiblelbl{rhomaxeq} 
  \rhomax^k = \rhomax^{k_0}.
\end{equation}
  \item[(iii)] Se o passo tangente e o vetor de multiplicadores satisfazem
  \begin{eqnarray} 
    \norma{z^k - \zc{k}} & = & \bigo(\norma{g_p(\zc{k},\mu^k)})
      \visiblelbl{hip:difz}, \\ 
    \norma{\lk{k} - \lls(\zc{k},\mu^k)} & = & \bigo(\norma{g_p(\zc{k},\mu^k)})
      \visiblelbl{hip:difl}, \\ 
    (\lambda^{k+1}_I)^T(s_c^{k+1}-s^k) & = &
      \bigo(\norma{g_p(\zc{k},\mu^k)}\rho^k), \visiblelbl{hip:gap}
\end{eqnarray} 
    ent\~ao (\ref{rhomaxeq}) \'e satisfeito, independentemente do
    valor de $\lim\inf\norma{g_p(\zc{k},\mu^k)}$. Em outras palavras,
    $\rhomax^k$ permanece suficientemente afastado de zero.  
  \end{description} 
\end{lemma}
\begin{proof} 
  \begin{description} 
    \item[(i)] Para provar que $\rhomax$ n\~ao muda na itera\c{c}\~ao $k+1$,
      s\'o precisamos mostrar que $\DLV{k+1} < -\DLH{k}/2$, o que, segundo o
      Lema \ref{lemma:32}, é obtido quando
  \begin{equation} 
    \DLV{k+1} \leq
    \frac{\xi_1}{2}\norma{g_p(\zc{k},\mu^k)}\min\{\xi_2\norma{g_p(\zc{k},
    \mu^k)},\xi_3\sqrt{\rho^k},1-\epsmu\}.\visiblelbl{eq:DLVlema2}
\end{equation} 
Para isso, tomamos 
\begin{equation} \visiblelbl{xi56}
  \xi_5\leq\min\bigg\{\frac{\xi_1\xi_2}{2\xi_4},\frac{10^{-4}\xi_1^2\xi_3^2}
  {4\xi_4^2(\xi_0+1)}\bigg\} \qquad \mbox{ e } \qquad \xi_6 \leq
  \frac{\xi_1(1-\epsmu)} {2\xi_4}, 
\end{equation} 
onde $\xi_4$ é a constante definida no Lema \ref{lemma:33}. Assim, a partir do Lema
\ref{lemma:33}, de (\ref{hip:limrhomax}) e (\ref{xi56}), obtemos
\begin{equation}\visiblelbl{dlvaux1} 
  \DLV{k+1} \leq \xi_4\rhomax^k\norma{\gpk{k}} \leq
    \xi_4\xi_5\norma{\gpk{k}}^2 \leq \meio\xi_1\xi_2\norma{\gpk{k}}^2.  
\end{equation} 
Além disso, usando (\ref{limrhomax}) e (\ref{limgamma}), temos
\begin{equation}\visiblelbl{sqrtrhomax} 
  \sqrt{\rhomax^k} \leq 10^2\sqrt{\rho^k(\xi_0+1)} \norma{\gpk{k}}^{-1/2}.  
\end{equation}
Extraindo a raiz quadrada dos dois lados de (\ref{hip:limrhomax}) e combinando o
resultado com (\ref{sqrtrhomax}), obtemos
\begin{equation}\visiblelbl{rhomax35} 
  \rhomax^k \leq \sqrt{\xi_5}10^2\sqrt{\xi_0+1}\sqrt{\rho^k}.  
\end{equation} 
  Usando o Lema \ref{lemma:33}, (\ref{rhomax35}) e (\ref{xi56}), temos
\begin{equation}\visiblelbl{dlvaux2} 
  \DLV{k+1} \leq \xi_4\rhomax^k\norma{\gpk{k}} \leq
  \xi_4\sqrt{\xi_5}10^2\sqrt{\xi_0+1}\sqrt{\rho^k}\norma{\gpk{k}} \leq
\meio\xi_1\xi_3\norma{\gpk{k}}\sqrt{\rho^k}. 
\end{equation} 
Finalmente, usando o Lema \ref{lemma:33}, (\ref{hip:limrhomax}) e (\ref{xi56}),
temos 
\begin{equation}\visiblelbl{dlvaux3} 
  \DLV{k+1} \leq \xi_4\rhomax^k\norma{\gpk{k}} \leq \xi_4\xi_6\norma{\gpk{k}}
  \leq \meio\xi_1(1-\epsmu)\norma{\gpk{k}}.  
\end{equation} 
O resultado segue de (\ref{dlvaux1}), (\ref{dlvaux2}) e (\ref{dlvaux3}).
\item[(ii)] Sejam $b = \lim\inf(\norma{\gpk{k}})$ e $\barra{k}_0$ um índice tal
  que $\norma{\gpk{k}} > b/2$, $\forall k \geq  \barra{k}_0$. Tome $k \geq
  \barra{k}_0$. Nesse caso, existem duas situações possíveis: ou existe $k$ tal
  que $\rhomax^k < \min\{\xi_5b/2,\xi_6\}$, ou $\rhomax^k \geq
  \min\{\xi_5b/2,\xi_6\}, \forall k$.
  Na primeira situação, pelo item (i), $\rhomax^{k+1} = \rhomax^{k}$ e, a partir
  deste $k$, $\rhomax$ não muda mais. Desta forma, basta definirmos este $k$
  como $k_0$. Na segunda situação, $\rhomax$ permance limitado inferiormente.
  Portanto, como $\rhomax^k = \rhomax^02^{-j}$, para algum $j\in\mathbb{N}$, a
  partir de uma certa iteração $k_0$, $\rhomax$ não irá descrescer mais.

\item[(iii)] Note que, pelas defini\c{c}\~oes de $\lls$ e $g_p$, e pelas
  hip\'oteses \ref{hip:global.continuidade}-\ref{hip:global.regularidade},
  $\lls(z,\mu)$ e $g_p(z,\mu)$ est\~ao bem definidos e são de classe $C^1$ em
  uma vizinhan\c{c}a compacta de $\barra{\mathcal{Z}}$, o fecho de $\mathcal{Z}
  = \{z_c^k\}$. Portanto, $\lls$ e $g_p$ s\~ao Lipschitz cont\'inuas no sentido
  que 
\begin{subequations} 
  \begin{eqnarray} 
    \norma{\lls(\zc{k+1},\mu) - \lls(\zc{k},\mu)} & = &
      \bigo(\norma{\zc{k+1}-\zc{k}}), \visiblelbl{llslip.a} \\
    \norma{\lls(\zc{k},\mu_1) - \lls(\zc{k},\mu_2)} & = &
      \bigo(\modulo{\mu_1-\mu_2}), \visiblelbl{llslip.b} 
\end{eqnarray}
\end{subequations}
  e
  \begin{subequations} 
  \begin{eqnarray}
    \norma{g_p(\zc{k+1},\mu) - g_p(\zc{k},\mu)} & = &
      \bigo(\norma{\zc{k+1}-\zc{k}}), \visiblelbl{gplip.a} \\
    \norma{g_p(\zc{k},\mu_1) - g_p(\zc{k},\mu_2)} & = &
      \bigo(\modulo{\mu_1-\mu_2}), \visiblelbl{gplip.b} 
\end{eqnarray}
  \end{subequations} 
  De (\ref{nnormal}), (\ref{hip:difz}), (\ref{limhhornormal}) e (\ref{limrho}),
  temos 
  \begin{eqnarray} 
    \norma{\zc{k+1}-\zc{k}} & \leq & \norma{\zc{k+1}-z^k} + \norma{z^k -
    \zc{k}} \nonumber \\ 
    & = & \bigo(\norma{h(z^k)}) + \bigo(\norma{g_p(\zc{k},\mu^k)}) \nonumber \\
    & = & \bigo(\rho^k) + \bigo(\norma{g_p(\zc{k},\mu^k)}) \nonumber \\ 
    & = & \bigo(\norma{g_p(\zc{k},\mu^k)}), \visiblelbl{difzck} 
\end{eqnarray} 
  e de (\ref{gplip.a}) e (\ref{difzck}), obtemos 
  \begin{eqnarray}
    \norma{g_p(\zc{k+1},\mu^k)} & \leq & \norma{g_p(\zc{k},\mu^k)} +
      \bigo(\norma{\zc{k+1}-\zc{k}}) \nonumber \\ 
    & = & \bigo (\norma{g_p(\zc{k},\mu^k)}). \visiblelbl{gpkplus} 
\end{eqnarray} 
  Agora, usando (\ref{gplip.b}), \eqref{limmu}, \eqref{gpkplus} e \eqref{limrho},
  obtemos
  \begin{eqnarray} 
    \norma{g_p(\zc{k+1},\mu^{k+1})} & \leq & \norma{g_p(\zc{k+1},\mu^k)} +
      \bigo(\modulo{\mu^k - \mu^{k+1}}) \nonumber \\ 
    & = & \bigo(\norma{g_p(\zc{k+1},\mu^k)}) + \bigo(\rho^k) \nonumber \\ 
    & = & \bigo(\norma{g_p(\zc{k},\mu^k)}). \visiblelbl{gpmukplus}
\end{eqnarray} 
Pelas definições \eqref{def:lagrangeano}, \eqref{def:phi}, \eqref{def:h} e
\eqref{def:lagrgeral}, a variação do Lagrangeano no passo normal da iteração $k$
pode ser escrita como
\begin{eqnarray}
  \DLV{k+1} 
    & = & L(\zc{k+1},\lk{k+1},\mu^{k+1}) - L(z^{k}, \lk{k}, \mu^k) \nonumber \\
    & = & f(\xc{k+1}) + \mu^{k+1}\beta(\zc{k+1}) + (\lk{k+1})^Th(\zc{k+1}) -
      \nonumber \\
    & & f(x^k) - \mu^k\beta(z^k) - (\lk{k})^T h(z^k) \nonumber \\
    & = & \mathcal{L}(\xc{k+1}, \lk{k+1}) - \mathcal{L}(x^k,\lk{k+1}) +
      \mu^{k+1}\beta(\zc{k+1}) - \mu^k\beta(z^k) + \nonumber \\
    & & [\lambda^{k+1} - \lambda^k]^Th(z^k) - (\lk{k+1}_I)^T(s_c^{k+1} - s^k).
      \visiblelbl{dlvaux4}
\end{eqnarray}
Usando uma expansão de Taylor, 
\eqref{def:gpk}, a Hipótese \ref{hip:global.regularidade},
\eqref{gradproj}, \eqref{hip:difl}, \eqref{gpmukplus}, 
\eqref{limhess}, \eqref{limhessh}, \eqref{limlambda},
\eqref{nnormal}, \eqref{limhhornormal} e \eqref{limrho},
garantimos que existe $x_\xi$ tal que
\begin{eqnarray}
  \mathcal{L}(\xc{k+1},\lk{k+1}) - \mathcal{L}(x^k,\lk{k+1})
  & = &\nabla_x \mathcal{L}(\xc{k+1},\lk{k+1})^T(\xc{k+1} - x^k) + \nonumber \\ 
  & & \meio (\xc{k+1}-x^k)^T\nabla_{xx}^2\mathcal{L} (x_{\xi},\lk{k+1})
    (\xc{k+1} - x^k) \nonumber \\
  & \leq & \norma{\vetord{I}{0}g_p^{k+1}}\norma{\xc{k+1}-x^k} +
    \meio(\xi_0+m_I\xi_0^2)\norma{\xc{k+1}-x^k}^2 \nonumber \\
    & = & \bigo(\norma{g_p(\zc{k},\mu^k)}^2) \visiblelbl{termo1.dlvaux4}
\end{eqnarray}

Usando uma expansão de Taylor,
\eqref{def:beta},
\eqref{zcompact}, \eqref{zbound}, \eqref{limmu} e \eqref{limrho}, temos
\begin{eqnarray}
  \mu^{k+1}\beta(\zc{k+1}) - \mu^k\beta(z^k) 
  & = & \mu^{k+1}[\beta(z^k) + \nabla \beta(z_{\xi})^T(\zc{k+1}-z^k)] 
    - \mu^k\beta(z^k) \nonumber \\
  & = & (\mu^{k}-\mu^{k+1})\sum_{i=1}^{m_I}\log(s_i^k) - \nonumber \\
  & & \mu^{k+1} e^T [\eta S^k + (1-\eta) S_c^{k+1}]^{-1}(s_c^{k+1}-s^k) \nonumber \\
  & = & \bigo(\mu^k) + \mu^{k+1}\sum_{i=1}^{m_I} \frac{s_i^k - s_{c_i}^{k+1}}
    {\eta s_i^k + (1-\eta)s_{c_i}^{k+1}} \nonumber \\
  & \leq & \bigo(\mu^k) + \mu^{k+1}m_I\frac{1-\epsmu}{\eta + (1-\eta)\epsmu}
    \nonumber \\
    & = & \bigo(\mu^k) = \bigo(\norma{g_p(\zc{k},\mu^k)}\rho^k).
    \visiblelbl{termo2.dlvaux4}
\end{eqnarray}

Usamos \eqref{hip:difl}, \eqref{gpmukplus},
\eqref{llslip.b}, \eqref{limmu}, \eqref{limrho},
\eqref{llslip.a},
\eqref{difzck} e \eqref{limhhornormal},
obtendo 
\begin{eqnarray}
  [\lk{k+1} - \lk{k}]^Th(z^k) & \leq & 
  \norma{\lk{k+1} - \lls(\zc{k+1},\mu^{k+1})}\norma{h(z^k)} + \nonumber \\
  & & \norma{\lls(\zc{k+1},\mu^{k+1}) - \lls(\zc{k+1},\mu^k)}\norma{h(z^k)} +
    \nonumber \\
  & & \norma{\lls(\zc{k+1},\mu^k) - \lls(\zc{k},\mu^k)}\norma{h(z^k)} + \nonumber \\
  & & \norma{\lls(\zc{k},\mu^k) - \lk{k}} \norma{h(z^k)} \nonumber \\
  & = & \bigo(\norma{g_p(\zc{k+1},\mu^{k+1})}\rho^k) +
    \bigo(\mu^k\rho^k) + 
    \bigo(\norma{g_p(\zc{k},\mu^k)}\rho^k) \nonumber \\
    & = & \bigo(\norma{g_p(\zc{k},\mu^k)}\rho^k). \visiblelbl{termo3.dlvaux4}
\end{eqnarray}

Assim, substituindo \eqref{termo1.dlvaux4}, \eqref{termo2.dlvaux4} e 
\eqref{termo3.dlvaux4} em \eqref{dlvaux4}, obtemos 
\begin{eqnarray*} \DLV{k+1} & = &
\bigo(\norma{g_p(\zc{k},\mu^k)}\rho^k) + \bigo(\rho^{k^2}) \\
& = & \bigo(\norma{g_p(\zc{k},\mu^k)}\rho^k) + \bigo(\rhomax^k\norma{g_p(\zc{k},
\mu^k)}\rho^k) \\
& = & \bigo(\norma{g_p(\zc{k},\mu^k)}\rho^k) =
\bigo(\norma{g_p(\zc{k},\mu^k)}^2\rhomax^k). 
\end{eqnarray*} 
Portanto, existe $\xi_7 > 0$ tal que 
\begin{eqnarray} \DLV{k+1} \leq
\xi_7\rhomax^k\norma{g_p(\zc{k},\mu^k)}^2.\visiblelbl{limdlvp} \end{eqnarray}
Vamos, agora, mostrar que, também no caso (iii), \eqref{eq:DLVlema2} é
satisfeito. Para tanto,
definimos $\bxi_7 = \xi_0 + \mu_{\max}m_I + (n+m_I)\xi_0^2$ e
\begin{eqnarray} \brm =
  \min\bigg\{\frac{\xi_1\xi_2}{2\xi_7},\frac{10^{-4}}{4\bxi_7(\xi_0 + 1)}
  \bigg(\frac{\xi_1\xi_3} {\xi_7}\bigg)^2,
  \frac{\xi_1(1-\epsmu)}{2\xi_7\bxi_7}\bigg\}.\visiblelbl{def:brm} 
\end{eqnarray}
Assim, tomando $k \geq k_0$ tal que $\rhomax^{k_0} \leq \brm$, e usando
\eqref{limdlvp} e \eqref{def:brm}, temos 
\begin{eqnarray*} 
  \DLV{k+1} & < &
  \xi_7\bigg(\frac{\xi_1\xi_2}{2\xi_7}\bigg)\norma{g_p(\zc{k},\mu^k)}^2 \\
  & = & \frac{\xi_1\xi_2}{2}\norma{g_p(\zc{k},\mu^k)}^2.  
\end{eqnarray*} 
Além disso, note que, por \eqref{limgamma}, \eqref{limjacob} e
\eqref{limlambda}, temos
\begin{eqnarray*} 
  \norma{g_p(\zc{k},\mu^k)} & = & \norma{g(\zc{k}) +
    A(\zc{k})^T\lls(\zc{k},\mu^k)} \\
  & \leq & \norma{g(\zc{k})} +
    \sum_{i=1}^{n+m_I}\norma{\lambda_{LS_i}(\zc{k},\mu^k) A_i(\zc{k})} \\
  & \leq & \xi_0 + \mu m_I + (n + m_I) \xi_0^2  \\
  & \leq & \xi_0 + \mu_{\max}m_I +
    (n+m_I)\xi_0^2 = \barra{\xi}_7.
\end{eqnarray*} 
Assim, usando \eqref{limdlvp}, \eqref{def:brm},
\eqref{limrhomax} e \eqref{limgamma}, temos 
\begin{eqnarray*} 
  \DLV{k+1} & = & \xi_7\sqrt{\rhomax^k}
    \sqrt{\rhomax^k}\norma{g_p(\zc{k},\mu^k)}^2 \\ 
  & \leq & \xi_7\bigg(\frac{\xi_1\xi_3}{\xi_7}
    \frac{10^{-2}\norma{g_p(\zc{k},\mu^k)}^2} {2\sqrt{\barra{\xi}_7(\xi_0+1)} } \bigg)
    \bigg(10^2\sqrt{\rho^k}\frac{\sqrt{\norma{g(\zc{k},\mu^k)} + 1}}
    {\sqrt{\norma{\gpk{k}}}}\bigg) \\ 
  & \leq & \frac{\xi_1\xi_3}{2} \sqrt{\rho^k}
    \frac{\norma{g_p(\zc{k},\mu^k)}^{3/2}} { \sqrt{\barra{\xi}_7} } \\ 
  & \leq & \frac{\xi_1\xi_3}{2}\sqrt{\rho^k}\norma{g_p(\zc{k},\mu^k)}. 
\end{eqnarray*}
    Finalmente, usando \eqref{limdlvp} e \eqref{def:brm}, temos
\begin{eqnarray*} 
  \DLV{k+1} & < & \xi_7\frac{\xi_1(1 - \epsmu)}
    {2\xi_7\bxi_7}\norma{g_p(\zc{k},\mu^k)}^2 \\ 
  & \leq & \frac{\xi_1(1 - \epsmu)}{2}\norma{g_p(\zc{k},\mu^k)}. 
\end{eqnarray*} 
Desse modo, temos
  $\DLV{k+1} < -\meio\DLH{k}$. Logo $\rhomax^k$ não muda depois de $k_0$.
  \end{description} \end{proof}

Apresentamos agora o teorema de convergência global de nosso algoritmo.
\begin{theorem}\label{teo:conv-global} 
  Sob as Hipóteses \ref{hip:global.continuidade}-\ref{hip:global.dsoc}, se o
  método CDI gera uma sequência infinita então existe uma subsequência
  convergente a um ponto estacionário para \eqref{prob:geral}.
  Se as condições (\ref{hip:difz}), (\ref{hip:difl}) e (\ref{hip:gap})
  também são satisfeitas, então toda subsequência convergente de $\{\xc{k}\}$ tem
  ponto limite estacionário para (\ref{prob:geral}).  
\end{theorem} 
\begin{proof} Suponha, por
  contradição, que $\lim\inf(\norma{g_p(\zc{k},\mu^k)}) = 2b > 0$. Seja
  $\barra{k}_0 \in \mathbb{N}$ tal que $\norma{g_p(\zc{k},\mu^k)} > b$ para
  qualquer $k > \barra{k}_0$. Assim, pelo item (ii) do Lema \ref{lemma:35},
  existe $k_0 \geq \barra{k}_0$ tal que, para todo $k \geq k_0$, $\rhomax^k =
  \rhomax^{k_0}$. Junto com \eqref{limrhomax} e \eqref{limgamma}, isto implica
  que $$\rho^k \geq 10^{-4}\frac{\rhomax^{k_0}b}{\xi_0 + 1},$$ e, portanto, para
  qualquer $i \geq k_0$, usando \eqref{limDLHp}, temos $\DLH{i} \leq -\theta$,
  onde \begin{eqnarray} \theta =
    \xi_1b\min\bigg\{\xi_2b,10^{-2}\xi_3\sqrt{\frac{\rhomax^{k_0}b}
  {\xi_0+1}},1-\epsmu\bigg\} > 0. \visiblelbl{def:theta} \end{eqnarray} Agora,
  usando \eqref{difL} e \eqref{def:theta}, podemos garantir que, para $k > k_0$,
  \begin{eqnarray*} L(\zc{k},\lk{k},\mu^k) - L(\zc{k_0},\lk{k_0},\mu^{k_0}) &
  = & \sum_{i = k_0+1}^k\Delta L_c^i \\ & \leq &
  \frac{1}{4}\sum_{i=k_0}^{k-1}\DLH{i} + r^{k_0} \\ & \leq & -\frac{1}{4}(k -
  k_0)\theta + r^{k_0} \longrightarrow -\infty.  \end{eqnarray*} Isso implica
  que $f$ é descontínua, ou $h$ é descontínua, ou uma das sequências,
  $\{\zc{k}\}$ ou $\{\lk{k}\}$, não é limitada, contrariando
  \ref{hip:global.continuidade}-\ref{hip:global.seq.limitadas}. Portanto,
  $\lim\inf(\norma{g_p(\zc{k}, \mu^k)}) = 0$.

Para a segunda parte do teorema, suponhamos válidas (\ref{hip:difz}),
(\ref{hip:difl}) e (\ref{hip:gap}). Então, pelo Lema \ref{lemma:35}, existe
$k_0$ tal que $\rhomax^k = \rhomax^{k_0}$, para todo $k \geq k_0$.

Suponha, por absurdo, que $\norma{g_p(\zc{k_l},\mu^{k_l})} \geq b > 0$ para uma
subsequência infinita $\{k_l\}$. Neste caso, usando \eqref{difL} e
\eqref{def:theta} novamente, e fazendo $n_k \longrightarrow \infty$, temos
$$L(\zc{k},\lk{k},\mu^k)-L(\zc{k_0},\lk{k_0},\mu^{k_0}) \leq
-\frac{1}{4}n_k\theta + r^{k_0} \longrightarrow -\infty,$$ onde $\theta$ é dado
por (\ref{def:theta}). Isso também contradiz
\ref{hip:global.continuidade}-\ref{hip:global.seq.limitadas}, implicando que
$\norma{g_p(\zc{k_l},\mu^{k_l})} \longrightarrow 0$ para toda subsequência de
$\{\zc{k}\}$.

Seja $\{k_l\}$ uma subsequência convergente tal que
$\norma{g_p(\zc{k_l},\mu^{k_l})} \longrightarrow 0$. Pelas propriedades do
algoritmo descritas em \eqref{limrho}-\eqref{limhhornormal} e \eqref{limmu}, temos
$\rho^{k_l} \longrightarrow 0$, $\norma{h(\zc{k_l})} \longrightarrow 0$ e
$\mu^{k_l} \longrightarrow 0$. Pela definição de $g_p(\zc{k_l},\mu^{k_l})$ temos
$-\mu^{k_l}e - S_c^{k_l}\lambda_I^{k_l} \longrightarrow 0$, de modo que
$S_c^{k_l}\lambda_I^{k_l} \longrightarrow 0$, e $\nabla f(\xc{k_l}) + \nabla
c(\xc{k_l})^T\lk{k_l} \longrightarrow 0$. Além disso, $\lambda_I^{k_l} \leq
\alpha(\mu^{k_l})^n$, de modo que $\lim \lambda_I^{k_l} \leq 0$. Portanto, o ponto
limite dessa subsequência é estacionário.  \end{proof}
