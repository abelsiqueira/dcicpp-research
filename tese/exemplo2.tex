\newpage
Para verificar a aplicação do método no caso com inequações, vamos apresentar
outro exemplo. Note, no entanto, que um problema de duas variáveis e uma
inequação precisaria ser analisado em 3 dimensões. Decidimos, para facilitar a
visualização, mostrar apenas as variáveis originais do problema, e utilizar a
definição da variável de folga para mostrar a restrição deslocada.
Pelo mesmo motivo, vamos mostrar apenas uma parte dos cilindros de confiança.

O problema que consideramos é
\begin{eqnarray}
  \min & f(x) = \meio[x_1^2 + (x_2+1)^2] \nonumber \\
  \mbox{suj. a} & x_2 \geq x_1^2, \nonumber \\
               & x_1 + x_2 = 1, \nonumber
\end{eqnarray}
cuja solução é o ponto $(\frac{-1+\sqrt{5}}{2},\frac{3-\sqrt{5}}{2}) \approx
(0.618,0.382)$.
Ao adicionarmos a variável de folga $s$, obtemos o problema
\begin{eqnarray}
  \min & f(x) = \meio[x_1^2 + (x_2+1)^2] \nonumber \\
  \mbox{suj. a} & x_2 - x_1^2 = s, \nonumber \\
               & x_1 + x_2 = 1, \nonumber \\
               & s \geq 0. \nonumber
\end{eqnarray}
Na solução, o valor de $s$ é $0$.

Os cilindros de confiança para esse problema serão os conjuntos da forma
$$ \mathcal{C}(\rho) = \{(x,s) \in \mathbb{R}^3: (x_2 - x_1^2 - s)^2 +
  (x_1+x_2-1)^2 \leq \rho^2\}. $$
Para mostrar alguma informação dessa região no plano original do problema,
decidimos tomar o caso com $s$ fixo. Dessa maneira, podemos ter alguma
informação do cilindro para as variáveis $x$. Note que essa visualização do
cilindro pode mudar de uma iteração para outra mesmo que o raio do cilindro
permaneça o mesmo.

\newcommand{\addexdois}[1]{
  \includegraphics[scale=1.0]{ex2_plots/ex2_img#1.pdf}
}

\begin{center}
  \addexdois{1}

  $s^0 = 0.01$
\end{center}

\begin{center}
  \begin{minipage}{0.9\textwidth}
    Começamos este exemplo pelo ponto $x^0 = (1,-1)$, e $s^0 = 0.01$, um valor
    pequeno e positivo.
    No gráfico, a reta corresponde à segunda equação; a região hachurada
    correspondente ao conjunto factível da desigualdade; e a parábola sólida
    corresponde à equação $x_2 - x_1^2 = s^0$.
    A curva tracejada denota o corte do cilindro pequeno em $s = s^0$, e a
    pontilhada o corte do cilindro grande pelo mesmo plano.
\end{minipage}
\end{center}

\begin{center}
  \addexdois{2}

  $s_c^0 = 0.01$
\end{center}

\begin{center}
  \begin{minipage}{0.9\textwidth}
    O ponto inicial já está dentro do cilindro pequeno, então o ponto é aceito
    sem necessidade de calcular nenhum passo normal.
  \end{minipage}
\end{center}

\begin{center}
  \addexdois{3}

  $s^+ = 14.15$
\end{center}

\begin{center}
  \begin{minipage}{0.9\textwidth}
    Tentamos agora fazer um passo tangente. O ponto encontrado fica fora do cilindro
    maior e não obtém decréscimo suficiente. Note que o cilindro não é
    visualizado, pois o corte é feito com o valor de $s$ encontrado resulta no
    conjunto vazio. Note que a parábola sólida também não é visualizada por
    causa do valor de $s$.
  \end{minipage}
\end{center}

\begin{center}
  \addexdois{4}

  $s^+ = 3.54$
\end{center}

\begin{center}
  \begin{minipage}{0.9\textwidth}
  O segundo passo tangente obtém decréscimo suficiente e está dentro do cilindro
  maior, que agora pode ser visualizado. Note que a parábola sólida também já
  pode ser visualizada.
\end{minipage}
\end{center}

\begin{center}
  \addexdois{5}

  $s^1 = 3.54$
\end{center}

\begin{center}
  \begin{minipage}{0.9\textwidth}
    O ponto é aceito como iteração tangente.
  \end{minipage}
\end{center}

\begin{center}
  \addexdois{6}

  $s_c = 1.29$
\end{center}

\begin{center}
  \begin{minipage}{0.9\textwidth}
    Atualizamos o raio do cilindro e fazemos um passo normal, levando o ponto
    para dentro do cilindro menor.
  \end{minipage}
\end{center}

\begin{center}
  \addexdois{7}

  $s_c^2 = 1.29$
\end{center}

\begin{center}
  \begin{minipage}{0.9\textwidth}
    Atualizamos novamente o raio do cilindro e o ponto encontrado permance dentro do
    cilindro menor. Portanto encerramos o passo normal.
  \end{minipage}
\end{center}

\begin{center}
  \addexdois{8}

  $s^+ = 1.03$
\end{center}

\begin{center}
  \begin{minipage}{0.9\textwidth}
    Fazemos um passo tangente, que fica dentro do cilindro e também tem
    decréscimo suficiente.
\end{minipage}
\end{center}

\begin{center}
  \addexdois{9}

  $s^2 = 1.03$
\end{center}

\begin{center}
  \begin{minipage}{0.9\textwidth}
    Aceitamos o ponto.
  \end{minipage}
\end{center}

\begin{center}
  \addexdois{10}

  $s_c = 0.67$
\end{center}

\begin{center}
  \begin{minipage}{0.9\textwidth}
    Fazemos um passo normal.
  \end{minipage}
\end{center}

\begin{center}
  \addexdois{11}

  $s_c^3 = 0.67$
\end{center}

\begin{center}
  \begin{minipage}{0.9\textwidth}
    Aceitamos a iteração normal.
  \end{minipage}
\end{center}

\begin{center}
  \addexdois{12}

  $s^+ = 0.16$
\end{center}

\begin{center}
  \begin{minipage}{0.9\textwidth}
    Fazemos um passo tangente.
  \end{minipage}
\end{center}

\begin{center}
  \addexdois{13}

  $s^3 = 0.16$
\end{center}

\begin{center}
  \begin{minipage}{0.9\textwidth}
    Aceitamos a iteração tangente.
  \end{minipage}
\end{center}

\begin{center}
  \addexdois{14}

  $s_c = 0.06$
\end{center}

\begin{center}
  \begin{minipage}{0.9\textwidth}
    Fazemos um passo normal.
  \end{minipage}
\end{center}

\begin{center}
  \addexdois{15}

  $s_c^4 = 0.06$
\end{center}

\begin{center}
  \begin{minipage}{0.9\textwidth}
    Aceitamos a iteração normal. O algoritmo oscila mais algumas iterações ao
    redor da solução até chegar à precisão necessária para declarar
    convergência.
  \end{minipage}
\end{center}

