\chapter{Conclusões}
\visiblelbl{chap:conclusoes}

Apresentamos nesta tese o método DCICPP, um método para o problema
de otimização não linear contínua. 
Mostramos uma teoria satisfatória de convergência global e local e dentro das
expectativas dos métodos usuais. Além disso, o algoritmo se mostrou competitivo
em comparação ao IPOPT e ao ALGENCAN, dois dos melhores algoritmos gratuitos
para programação não linear.

Ainda existe bastante espaço para trabalho tanto na teoria, quanto no algoritmo.
Consideramos a extensão da teoria para condições mais gerais de passos tangentes
e normais, possibilitando a implementação de algoritmos diferentes e mais
robustos para os subproblemas. Uma possibilidade é utilizar algum pacote pronto
para o subproblema.
Também consideramos a possibilidade de implementar a resolução dos sistemas
lineares utilizando métodos iterativos e/ou aproximados. Esperamos estudar os
problemas com Jacobianas singulares e quase singulares.
Também vale mencionar que o tratamento de variáveis fixas está sendo
implementado, e esperamos obter uma versão estável para próximos trabalhos.

Para finalizar, vale mencionar que a escolha dos parâmetros também pode fazer
muita diferença nos resultados, e que escolhemos apenas parâmetros canônicos ou
sugeridos na literatura, de modo que é possível obter alguma melhoria no
desempenho através da seleção criteriosa desses parâmetros.

Em trabalhos futuros, esperamos fazer novas comparações com outros algoritmos
notáveis e implementar as mudanças sugeridas acima. 
Também gostaríamos de obter hipóteses mais fracas para a teoria de convergência
do método, e estudar a estratégia de cilindros de confiança com outras
aplicações.
