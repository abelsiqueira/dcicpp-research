\chapter{Convergência}\label{chap:teoria}

Apresentaremos agora os resultados de convergência do método. Inicialmente
mostraremos que, com condições pouco restritivas, a sequência gerada pelo método
converge para um ponto estacionário. A seguir, mostraremos que, numa vizinhança
do ponto estacionário, se este for um minimizador sob condições satisfatórias, o
método converge superlinearmente em dois passos. Não obstante, mostraremos ainda
que o método converge para pontos estacionários do problema de infactibilidade, de
modo que, se o problema for infactível, não ficará iterando infinitamente.
Foi possível aproveitar todas as propriedades de convergência do método
original \cite{bib:chico-dci}, sendo necessário apenas alguns ajustes para o caso geral com
desigualdades. Para a convergência global, precisamos apenas de algumas
modificações para incluir o parâmetro de barreira. Para a convergência local,
precisamos modificar um pouco os teoremas para tratar apenas das restrições
ativas na solução. 

\section{Converg\^encia Global}\visiblelbl{sec:conv-global}

Apresentamos aqui os resultados de convergência global do método. Veremos que,
com condições relativamente fracas, podemos obter resultados satisfatórios de
convergência. A seguir estão as nossas hipóteses para a prova de convergência do
Algoritmo \ref{alg:outline}.
\begin{hypoenv}\visiblelbl{hip:global.continuidade} 
  As funções $f$, $c_E$ e $c_I$ s\~ao
$C^2$.  
\end{hypoenv} 
\begin{hypoenv}\visiblelbl{hip:global.seq.limitadas} 
  As sequ\^encias $\{\zc{k}\}$ e $\{z^k\}$, as aproxima\c{c}\~oes $B^k$ e os
  multiplicadores $\{\lk{k}\}$ permanecem uniformemente limitados.
\end{hypoenv} 
\begin{hypoenv}\visiblelbl{hip:global.regularidade} A
  restaura\c{c}\~ao n\~ao falha, no sentido de que o algoritmo normal sempre
  encontra um ponto dentro do cilindro, e $\mathcal{Z} = \{\zc{k}\}$ permanece
  longe do conjunto singular de $h$, no sentido que $h$ \'e regular no fecho de
  $\mathcal{Z}$.  Al\'em disso, se a sequ\^encia gerada $\{\zc{k}\}$ \'e
  infinita, ent\~ao 
\begin{equation}\visiblelbl{nnormal} 
  \norma{\zc{k+1}-z^k} = \bigo(\norma{h(z^k)}).
\end{equation} 
\end{hypoenv}
\begin{hypoenv}\visiblelbl{hip:global.dsoc} 
  $\norma{\dsoc^k} = \bigo(\norma{\delta_t^k}^2)$ 
\end{hypoenv} 
A Hipótese \ref{hip:global.continuidade} é natural e esperada, já que canônica,
considerando que, no método, utilizamos essas derivadas, ou aproximações para as
mesmas. 
A Hipótese \ref{hip:global.seq.limitadas} é necessária pois podemos tomar alguma
direção de descida ilimitada. 
As aproximações podem ser definidas de modo a permanecerem
uniformemente limitadas. 
A Hipótese \ref{hip:global.regularidade} é importante pois a restauração pode
falhar. Note que na Seção \ref{sec:conv-infactivel} comentamos sobre a
convergência do método para pontos estacionários da infactibilidade.
A Hipótese \ref{hip:global.dsoc} é tradicional para os passos de correção de
segunda ordem.

Apresentamos agora algumas propriedades provenientes do algoritmo.
Começamos por aquelas relacionadas aos cilindros.
Pela maneira que calculamos os raios dos cilindros no passo
\ref{alg:normal-call} do Algoritmo \ref{alg:outline}, pelas condições dos
iterandos nos passos \ref{alg:normal-call} e \ref{alg:tangente-call} do Algoritmo
\ref{alg:outline}, e pela maneira que atualizamos $\rhomax$ nos passos
\ref{alg:update-rhomax-start}-\ref{alg:update-rhomax-end} do Algoritmo
\ref{alg:update.rhomax}, temos
\begin{eqnarray}
  \rho^k \quad \leq & \rhomax^{k-1}\norma{g_p^k} & \leq \quad
    2\rhomax^k\norma{g_p^k} \visiblelbl{limrho}, \\
  \rhomax^k \quad \leq & \rhomax^{k-1} & \leq \quad
    10^4\rho^k\frac{\norma{g(\zc{k},\mu^k)} + 1} {\norma{\gpk{k}}},
    \visiblelbl{limrhomax} \\
  \norma{h(\zc{k})} \quad \leq & \norma{h(z^{k-1})} & \leq \quad 2\rho^{k-1}.
    \visiblelbl{limhhornormal}
\end{eqnarray}
Como indicado previamente em \eqref{eq:slim} e visto no algoritmo, nossas
iterações seguem uma regra de fração-para-a-fronteira, de modo que são válidas
as inequações
\begin{eqnarray}\visiblelbl{zbound}
 \begin{array}{rcl}
  s_c^{k} & \geq & \epsmu s^{k-1}, \\
  s^{k} & \geq & \epsmu s_c^{k}, \\
  s^+ & \geq & \epsmu s_c^k.
 \end{array}
\end{eqnarray}
Alem disso, pela definição de $\mu^k$ no passo \ref{alg:mudef} do Algoritmo
\ref{alg:etapa.normal}, temos
\begin{equation}\visiblelbl{limmu}
  \mu^k \leq \alpha_{\rho}\min\{\rho^k,(\rho^k)^2\}.
\end{equation}
Deste ponto em diante, supomos que as sequências $\{\zc{k}\}$ e $\{z^k\}$,
geradas pelo algoritmo, satisfazem
\ref{hip:global.continuidade}-\ref{hip:global.dsoc}. Denotando por $\delta_N^k$
o passo normal e por $\delta_T^k$ o passo tangente na iteração $k$, temos
\begin{equation*} \delta_N^k = z_c^k - z^{k-1} \qquad \mbox{e} \qquad \delta_T^k
= z^k - \zc{k} = \delta_t^k + \dsoc^k.  \end{equation*} As hipóteses
\ref{hip:global.continuidade}-\ref{hip:global.dsoc} nos permitem escolher uma
constante $\dmax > 0$, tal que, para todo $k$,
\begin{equation}\visiblelbl{limdmax} 
  \norma{\delta_t^k} + \norma{\dsoc^k} + \norma{\delta_N^k} \leq \dmax.  
\end{equation} As
hipóteses tamb\'em nos permitem definir $\xi_0 > 0$, tal que, $\forall k$, se
$\norma{z - \zc{k}} \leq \dmax$ e $\mu \leq \mu_{0}$, ent\~ao 
\begin{eqnarray}
  \norma{A_j(z)} & \leq & \xi_0, \qquad j = 1,\dots,m \visiblelbl{limjacob} \\
  \norma{\nabla^2 h_j(z)} & \leq & \xi_0, \qquad j = 1,\dots,m
  \visiblelbl{limhessh} \\
  \norma{\nabla f(x)} & \leq & \xi_0, \visiblelbl{limgrad} \\
  \norma{\nabla^2 f(x)} & \leq & \xi_0, \visiblelbl{limhess} \\
  \norma{g(z,\mu)} & \leq & \xi_0, \visiblelbl{limgamma} \\
  \norma{\Gamma(z,\mu)} & \leq & \xi_0, \visiblelbl{limGamma} \\
  \norma{B^k} & \leq & \xi_0, \visiblelbl{limB} \\
  \norma{\lk{k}} & \leq & \xi_0, \visiblelbl{limlambda} \\
  \norma{\dsoc^k} & \leq & \xi_0\norma{\delta_t^k}^2 \visiblelbl{limsoc} \\
  \norma{\Lac{k}} & \leq & \xi_0 \visiblelbl{limLac}, 
\end{eqnarray}
  \begin{eqnarray}\visiblelbl{zcompact}
\begin{array}{rcl} s_c^{k} & \leq & \xi_0 s^{k-1}, \\
  s^k & \leq & \xi_0 s_c^{k}. 
\end{array} 
\end{eqnarray} 
Também supomos que (\ref{nnormal}) pode
ser reescrito como 
\begin{equation}\visiblelbl{nnormalxi} 
  \norma{\zc{k+1} - z^k}
\leq \xi_0\norma{h(z^k)}.  
\end{equation} 
Definimos as matrizes $\Lac{k} = \Lambda(\zc{k})$, $\Lak{k} = \Lambda(z^k)$ e
$\Lambda^+ = \Lambda(\zp)$, para facilitar a notação. 
  Nosso resultado de convergência global é dado no Teorema
  \ref{teo:conv-global}. Esse resultado depende dos próximos cinco lemas. O
  Lema \ref{lemma:31} a seguir dá um limitante para o aumento da infactibilidade
  do causada pelo passo tangente em relação ao iterando obtido no passo normal.

\begin{lemma}\visiblelbl{lemma:31} 
  A tentativa de itera\c{c}\~ao $\zp$ gerada na
  linha \ref{alg:tangente-plus} do Algoritmo \ref{alg:etapa.tangente} satisfaz
  \begin{equation}\visiblelbl{limvarh} 
    \norma{h(\zp) - h(\zc{k})} \leq \bxi_0\norma{\delta_t}^2,
  \end{equation} 
  em que $\barra{\xi}_0$ é uma constante positiva.
\end{lemma} 
\begin{proof} Iremos
  omitir os \'indices $k$ nessa demonstra\c{c}\~ao. A itera\c{c}\~ao \'e
  definida por $\zp = z_c + \dplus = z_c + \Lambda_c(\delta_t + \dsoc)$.  Usando
  uma expansão de Taylor, o fato de que $A_j(z_c)\delta_t = 0$ e
  (\ref{limhessh}), podemos garantir que existe $\zxi^j = \eta_j\zp +
  (1-\eta_j)z_c$, tal que
\begin{eqnarray} 
  \modulo{h_j(\zp) - h_j(z_c)} & = & \modulo{\nabla h_j(z_c)^T\dplus +
    \meio(\dplus)^T\hess h_j(\zxi^j)\dplus} \nonumber \\
  & = & \modulo{\nabla h_j(z_c)^T\Lac{k}(\dt+\dsoc) +
      \meio(\delta_t+\dsoc)^T\Lac{k}\hess h_j(\zxi^j)\Lac{k}(\delta_t +
    \dsoc)} \nonumber \\
  & \leq & \modulo{A_j(z_c)^T(\delta_t+\dsoc)} +
    \meio\norma{\delta_t+\dsoc}^2\norma{\Lac{k}}^2\norma{\hess h_j(\zxi^j)}
    \nonumber \\
  & \leq & \modulo{A_j(z_c)^T\dsoc} + \meio\norma{\delta_t+\dsoc}^2\xi_0^3.
    \label{lemma31.aux}
\end{eqnarray}
Agora, por \eqref{limjacob} e \eqref{limsoc}, temos
$$ \modulo{A_j(z_c)^T\dsoc} \leq \xi_0^2\norma{\delta_t}^2, $$
e pela desigualdade $\norma{v+w}^2 \leq 2(\norma{v}^2 + \norma{w}^2)$,
\eqref{limsoc} e \eqref{limdmax},
$$ \meio\norma{\delta_t + \dsoc}^2 \leq \norma{\delta_t}^2 + \norma{\dsoc}^2
\leq \norma{\delta_t}^2 + \dmax\xi_0\norma{\delta_t}^2. $$
Substituindo essas duas desigualdades em \eqref{lemma31.aux}, obtemos
\begin{eqnarray*}
 \modulo{h_j(\zp) - h_j(z_c)} 
  & \leq & \xi_0^2\norma{\delta_t}^2 + \xi_0^3(\norma{\delta_t}^2 +
    \xi_0\dmax\norma{\delta_t}^2) \\
  & \leq & \xi_0^2(1 + \xi_0 + \xi_0^2\dmax)\norma{\delta_t}^2.
\end{eqnarray*} 
  Definindo $\bxi_0 = \sqrt{m}\xi_0^2(1 + \xi_0 + \xi_0^2\dmax)$, temos o
  resultado desejado.
\end{proof}

O lema a seguir mostra que, se as hipóteses
\ref{hip:global.continuidade}-\ref{hip:global.dsoc} são satisfeitas, o passo
tangente não falha, e obtemos redução suficiente no Lagrangeano.
\begin{lemma}\visiblelbl{lemma:32} Se $x_c^k$ não é um ponto estacionário para o
  problema \eqref{prob:geral}, então $\zp$ \'e aceito com suficientes iterações.
  Al\'em disso,
  podemos definir constantes $\xi_1$, $\xi_2$ e $\xi_3$ tais que, para todo $k$,
\begin{equation}\visiblelbl{limDLHp} \DHp \leq
-\xi_1\norma{\gpk{k}}\min\{\xi_2\norma{\gpk{k}},\xi_3\sqrt{\rho^k},1-\epsmu\}.
\end{equation} \end{lemma} \begin{proof} Iremos omitir os \'indices $k$ nessa
  demonstra\c{c}\~ao. Seja $\zp = z_c + \dplus = z_c + \Lambda_c(\dt + \dsoc)$
  um candidato obtido na linha \ref{alg:tangente-plus} da $k$-\'esima
  itera\c{c}\~ao do Algoritmo \ref{alg:outline}.

  Utilizando uma expans\~ao de Taylor e \eqref{limvarh}, existe $z_\xi = \eta
  z^+ + (1-\eta)z_c$, para algum $\eta\in[0,1]$, tal que
\begin{eqnarray} 
  \DHp & = & L(\zp,\lambda,\mu) - L(z_c,\lambda,\mu) \nonumber \\
   & = & \varphi(\zp,\mu) - \varphi(z_c,\mu) + \lambda^T[h(\zp) - h(z_c)]
  \nonumber \\
  & \leq & \nabla \varphi(z_c,\mu)^T\dplus + \meio(\dplus)^T\hess
  \varphi(\zxi,\mu)\dplus + \xi_0\bxi_0\norma{\dt}^2. \visiblelbl{dlh:passo1}
\end{eqnarray}
Para o primeiro termo, usando \eqref{def:gradiente_escalado}, o fato de que
$g(z_c,\mu)^T\delta_t = (g_p^k)^T\delta_t$, a definição \eqref{prob:tangente},
e as condições \eqref{limsoc}, \eqref{limgamma} e
\eqref{limB}, temos
\begin{align}
  \nabla\varphi(z_c,\mu)^T\delta^+ & = \nabla\varphi(z_c,\mu)^T\Lambda_c
  (\delta_t+\dsoc) \nonumber \\
  & = g(z_c,\mu)^T(\delta_t + \dsoc) \nonumber \\
  & = q(\delta_t) - \meio\delta_t^TB\delta_t + g(z_c,\mu)^T\dsoc \nonumber \\
  & \leq q(\delta_t) + \bigg(\meio\xi_0 + \xi_0^2\bigg)\norma{\delta_t}^2.
  \visiblelbl{dlh:passo2}
\end{align}
Para o segundo termo de \eqref{dlh:passo1}, usando \eqref{def:hess_escalada},
\eqref{limGamma}, \eqref{zbound}, a desigualdade $\norma{v+w}^2 \leq 2(\norma{v}^2
+ \norma{w}^2)$, \eqref{limdmax} e \eqref{limsoc}, temos
\begin{align}
  \meio(\delta^+)^T\nabla^2\varphi(z_\xi,\mu)\delta^+ & =
    \meio(\delta_t+\dsoc)^T\Lambda_c\Lambda(z_{\xi})^{-1}\Gamma(z_{\xi},\mu)
    \Lambda(z_{\xi})^{-1}\Lambda_c(\delta_t + \dsoc) \nonumber \\
  & \leq \dfrac{1}{2}\norma{\Gamma(z_{\xi},\mu)} \norma{\Lambda(z_{\xi})^{-1}
    \Lambda_c}^2 \norma{\delta_t+\dsoc}^2  \nonumber \\
  & \leq \dfrac{1}{2} \dfrac{\xi_0}{\epsmu^2} \norma{\delta_t + \dsoc}^2
    \nonumber \\
  & \leq \frac{\xi_0}{\epsmu^2} (\norma{\delta_t}^2 + \norma{\dsoc}^2) \nonumber
    \\
  & \leq \frac{\xi_0}{\epsmu^2} (1+\xi_0\dmax)\norma{\delta_t}^2.
    \visiblelbl{dlh:passo3}
\end{align}
Assim, substituindo \eqref{dlh:passo2} e \eqref{dlh:passo3} em
\eqref{dlh:passo1}, temos
\begin{align}
  \DHp & \leq q(\delta_t) + \bigg(\frac{\xi_0}{2}+\xi_0^2\bigg)\norma{\delta_t}^2
  + \frac{\xi_0}{\epsmu^2}(1+\xi_0\dmax)\norma{\delta_t}^2
  + \xi_0\barra{\xi_0}\norma{\delta_t}^2 \nonumber \\
  & = q(\delta_t) + \barra{\xi}_1\norma{\delta_t}^2, \visiblelbl{auxdhp}
\end{align} 
onde $\barra{\xi}_1 = \frac{\xi_0}{2} + \xi_0^2 + \frac{\xi_0}{\epsmu^2} (1 +
\xi_0\dmax) + \xi_0\barra{\xi}_0$.

Pela defini\c{c}\~ao de $\dcp$ na linha \ref{alg:cauchy-point} do Algoritmo
\ref{alg:etapa.tangente}, temos $$\norma{\dcp} \geq
\min\left\{\frac{\norma{g_p}}{\norma{B}},\Delta,1-\epsmu \right\} \geq
\min\left\{\frac{\norma{g_p}}{\xi_0}, \Delta, 1 - \epsmu \right\}$$ e
\begin{equation}\visiblelbl{limqdcp} q(\dcp) \leq \meio\dcp^Tg_p \leq
-\meio\norma{g_p}\min\left\{ \frac{\norma{g_p}}{\xi_0}, \Delta, 1 - \epsmu
\right\}.  \end{equation}

Vamos mostrar que $\zp$ \'e aceito se $\Delta \leq \barra{\Delta}$, onde
\begin{equation}\visiblelbl{bdelta} \barra{\Delta} =
\min\left\{\frac{\norma{g_p}}{4\bxi_1}, 1-\epsmu, \sqrt{\frac{\rho}
{\bxi_0}}\right\}.  \end{equation} Note que $\xi_0 < 4\bxi_1$, logo
\begin{equation}\visiblelbl{limbdelta} \barra{\Delta} \leq
\frac{\norma{g_p}}{\xi_0}.  \end{equation} Como $\Delta \leq \barra{\Delta}$,
usando (\ref{limbdelta}), podemos simplificar (\ref{limqdcp}) para
\begin{equation}\visiblelbl{limqdcp2} 
  q(\dcp) \leq -\meio\norma{g_p}\Delta.
\end{equation}
Combinando $\norma{\dt} \leq \Delta$ e $q(\dt) \leq q(\dcp)$ da linha
\ref{alg:delta-t} do Algoritmo \ref{alg:etapa.tangente}, com (\ref{bdelta}) e
(\ref{limqdcp2}) obtemos 
\begin{eqnarray} 
  \bxi_1\norma{\dt}^2 & \leq & \bxi_1\Delta^2 \leq \bxi_1\barra{\Delta}\Delta
  \nonumber \\ 
  & \leq & \frac{1}{4}\norma{g_p}\Delta \leq -\meio q(\dcp) \leq -\meio q(\dt).
\visiblelbl{limxi1} 
\end{eqnarray} 
Agora, de (\ref{auxdhp}) e (\ref{limxi1}),
temos \begin{equation}\visiblelbl{negDHp} \DHp \leq \meio q(\dt) < 0,
\end{equation} de modo que 
\begin{equation}\visiblelbl{rgeqeta1} 
  r = \frac{\DHp}{q(\dt)} \geq \meio \geq \eta_1.  
\end{equation}
Como $\Delta \leq \barra{\Delta}$ e $\norma{h(z_c)} \leq \rho$, usando
(\ref{limvarh}) e (\ref{bdelta}), temos $$\norma{h(\zp)} \leq \rho + \bxi_0
\norma{\dt}^2 \leq \rho + \bxi_0\barra{\Delta}^2 \leq 2 \rho.$$ Portanto, ambas
condi\c{c}\~oes da linha \ref{alg:tangente-condicoes} do Algoritmo
\ref{alg:etapa.tangente} s\~ao satisfeitas e $\zp$ \'e aceito.

Para provar a segunda parte, vamos lembrar que cada vez que o passo \'e
rejeitado, multiplicamos o raio $\Delta$ por $\alpha_R$. Ent\~ao podemos assumir
que o raio aceito satisfaz $\Delta \geq \alpha_R\barra{\Delta}$, onde $0 <
\alpha_R < 1$. Combinando isto com (\ref{negDHp}), a condição $q(\dt) \leq
q(\dcp)$, (\ref{limqdcp}), (\ref{limbdelta}) e (\ref{bdelta}), obtemos 
\begin{eqnarray*}
  \DHp & \leq & \meio q(\dt) \leq \meio q(\dcp) \\ 
   & \leq & -\frac{1}{4}\norma{g_p}\min\left\{ \frac{\norma{g_p}}{\xi_0},
    \Delta, 1 - \epsmu\right\} \\ 
  & \leq & - \frac{1}{4}\norma{g_p}\alpha_R\barra{\Delta} \\ 
  & \leq & -\xi_1\norma{g_p}\min\{\xi_2\norma{g_p}, \xi_3\sqrt{\rho}, 1 -
    \epsmu\}, 
\end{eqnarray*} 
com $\xi_1 = \alpha_R/4$, $\xi_2 = 1/(4\bxi_1)$ e
$\xi_3 = 1/\sqrt{\bxi_0}$.  \end{proof}

O próximo lema define um limitante superior para a variação normal do
Lagrangeano. Note que essa variação pode ser positiva.
\begin{lemma}\visiblelbl{lemma:33} 
  Existe uma constante positiva $\xi_4$ tal que, para $k$ suficientemente
  grande,
  $$\DLV{k+1} \leq \xi_4\rhomax^k\norma{\gpk{k}}.  $$ 
\end{lemma} 
\begin{proof}
    Usando o Teorema do Valor Médio, temos
\begin{eqnarray*} 
    \DLV{k+1} & = & L(\zc{k+1},\lk{k+1},\mu^{k+1}) - L(z^k,\lk{k},\mu^k) \\ 
    & = & \varphi(\zc{k+1},\mu^{k+1}) - \varphi(z^k,\mu^k) + h(\zc{k+1})^T\lk{k+1} -
      h(z^k)^T\lk{k} \\ 
    & = & \varphi(\zc{k+1},\mu^k) - \varphi(z^k,\mu^k) + h(\zc{k+1})^T\lk{k+1} -
      h(z^k)^T\lk{k} + (\mu^{k+1}-\mu^k)\beta(z_c^{k+1}) \\
    & = & \nabla \varphi (\zxi,\mu^k)^T(\zc{k+1}-z^k) + h(\zc{k+1})^T\lk{k+1} -
      h(z^k)^T\lk{k} + (\mu^{k+1}-\mu^k)\beta(z_c^{k+1}) \\
    & =  & \nabla f(x_{\xi})^T(x_c^{k+1}-x^k) +
      \mu^k\nabla\beta(z_{\xi})^T(z_c^{k+1}-z^k) + h(\zc{k+1})^T\lk{k+1} -
      h(z^k)^T\lk{k} \\
    & & + (\mu^{k+1}-\mu^k)\beta(z_c^{k+1}),
\end{eqnarray*} 
  com $\zxi = \eta z^k + (1-\eta)\zc{k+1}$, para algum $\eta \in [0,1]$. 
  Pela hipótese \ref{hip:global.seq.limitadas}, existe $M >
  0$ tal que $s_i^k,s_{c_i}^k \leq M$. Usando isso, \eqref{limgrad},
(\ref{zbound}), (\ref{limlambda}), (\ref{nnormalxi}), (\ref{limhhornormal}), (\ref{limmu}) e
(\ref{limrho}) temos 
\begin{eqnarray*} 
  \DLV{k+1} & \leq & \xi_0\norma{\xc{k+1}-x^k} + 
  \mu^k\sum_{i=1}^{m_I} \frac{s_i^k - s_{c_i}^{k+1}}{\eta s_i^k +
    (1-\eta)s_{c_i}^{k+1}} + 
    \xi_0\norma{h(\zc{k+1})} + \xi_0\norma{h(z^k)} + (\mu^k - \mu^{k+1})m_IM \\ 
  & \leq & \xi_0^2\norma{h(z^k)} + \xi_0\norma{h(z^k)} + \xi_0\norma{h(z^k)} +
    \mu^km_I\bigg(\frac{1 - \epsmu}{\eta + (1-\eta)\epsmu}+M\bigg) \\ 
  & \leq & (\xi_0^2 + 2\xi_0)2\rho^k + 
    \rho^km_I\bigg(\frac{1 - \epsmu}{\eta + (1-\eta)\epsmu}+M\bigg) \\ 
  & \leq & \xi_4\rhomax^k\norma{\gpk{k}}, 
\end{eqnarray*}
onde $\xi_4 = 4(\xi_0^2 + 2\xi_0) + 2m_I\dfrac{1-\epsmu}{\eta+(1-\eta)\epsmu}+2M$.  
\end{proof}

O Lema \ref{lemma:34} mostra que entre iterações sucessivas em que $\rhomax$ não
muda, o Lagrangeano decresce proporcionalmente à variação do Lagrangeano nos
passos tangentes.  \begin{lemma}\visiblelbl{lemma:34} Se
  $\rhomax^{k+1}=\rhomax^{k+2}=\dots=\rhomax^{k+j}$, para $j \geq 1$, ent\~ao
\begin{equation}\visiblelbl{difL} 
  L_c^{k+j}-L_c^k = \sum_{i = k+1}^{k+j}\Delta
L_c^i \leq \frac{1}{4}\sum_{i = k}^{k+j-1}\DLH{i} + r^k, 
\end{equation} 
onde
$r^k = \meio[\Lref^k - L_c^k]$.  \end{lemma} \begin{proof} Suponha que $\Lref$
  n\~ao muda entre as itera\c{c}\~oes $k+1$ e $k+j_1-1$, onde $0<j_1\leq j+1$.
  Neste caso, por (\ref{DLC}) e pelo crit\'erio da linha \ref{alg:condicao-dlv}
  do Algoritmo \ref{alg:update.rhomax}, temos 
\begin{equation}\visiblelbl{limaux1}
  L_c^{k+j_1-1} - L_c^k = \sum_{i = k+1}^{k+j_1-1}(\DLV{i}+\DLH{i-1}) \leq
  \meio\sum_{i = k}^{k + j_1-2}\DLH{i}.  
\end{equation} 
Por outro lado, se $\Lref$
muda na itera\c{c}\~ao $k+j_1$, ent\~ao a condi\c{c}\~ao da linha
\ref{alg:condicao-dlv} \'e satisfeita. Neste caso, como $\rhomax$ n\~ao muda
nesta itera\c{c}\~ao, de modo que a condição da linha
\ref{alg:update-rhomax-start} não \'e satisfeita, e como $\DLH{k} \leq 0$ para
todo $k$, temos
\begin{eqnarray} 
  L_c^{k+j_1}-L_c^k & \leq & \DLV{k+j_1}+L(z^{k+j_1-1}, \lk{k+j_1-1},
    \mu^{k+j_1-1}) - \Lref^k + [\Lref^k - L_c^k] \nonumber \\ 
  & \leq & \meio[L(z^{k+j_1-1}, \lk{k+j_1-1}, \mu^{k+j_1-1}) - \Lref^k] +
    [\Lref^k - L_c^k] \nonumber \\ 
  & \leq & \meio[\DLH{k+j_1-1}+L_c^{k+j_1-1} - L_c^k] + \meio[\Lref^k-L_c^k]
    \nonumber \\ 
  & \leq & \frac{1}{4}\sum_{i = k}^{k + j_1-1}\DLH{i} + r^k.
    \visiblelbl{limaux2} 
\end{eqnarray} 
Se $j_1 \geq j$,
ent\~ao (\ref{limaux1}) e (\ref{limaux2}) implicam (\ref{difL}).

Por outro lado, se $\Lref$ \'e atualizado nas itera\c{c}\~oes
$k+j_1,\dots,k+j_s$, onde $0<j_1<j_2<\dots<j_s\leq j$, ent\~ao $r^{k+j_1} =
r^{k+j_2} = \dots = r^{k+j_s} = 0$. Portanto, aplicando o mesmo processo
descrito acima v\'arias vezes, e definindo $j_0 = 0$, obtemos 
\begin{equation*}
  L_c^{k+j} - L_c^k = \sum_{i = 1}^s[L_c^{k+j_i} - L_c^{k+j_i-1}] + L_c^{k+j} -
  L_c^{k+j_s} \leq \frac{1}{4}\sum_{i = k}^{k + j-1}\DLH{i} + r^k .
\end{equation*}
\end{proof}

O próximo lema estabelece a existência de \emph{espaço normal suficiente} nos
cilindros de confiança para garantir que o Lagrangeano possa ter decréscimo
suficiente. A base desse lema é que a inequação \eqref{limDLHp} garante que,
assintoticamente, $\modulo{\Delta L_H^k}$ é maior que uma fração de
$\sqrt{\rho^k}$, enquanto que, na demonstração do Lema \ref{lemma:34}, vemos que
$\norma{\Delta L_V^k} = \bigo(\rho^k)$. Isso significa que, no limite, a
restauração não irá destruir o descréscimo do Lagrangeano, o que evita que sejam
feitas alterações excessivas de $\rhomax$.  
\begin{lemma}\visiblelbl{lemma:35} 
  Se CDI gera uma
  sequ\^encia infinita $\{z^k\}$, ent\~ao 
\begin{description} 
  \item[(i)] Existem constantes positivas $\xi_5$ e $\xi_6$ tais que, se
\begin{equation}\visiblelbl{hip:limrhomax} 
  \rhomax^k < \min\{\xi_5\norma{g_p(\zc{k},\mu^k)},\xi_6\}, 
\end{equation}
    então $\rhomax$ n\~ao muda na itera\c{c}\~ao $k + 1$.  
  \item[(ii)] Al\'em disso, se $\lim\inf \norma{\gpk{k}} > 0$, ent\~ao existe
    $k_0 > 0$ tal que, para todo $k \geq k_0$
\begin{equation}\visiblelbl{rhomaxeq} 
  \rhomax^k = \rhomax^{k_0}.
\end{equation}
  \item[(iii)] Se o passo tangente e o vetor de multiplicadores satisfazem
  \begin{eqnarray} 
    \norma{z^k - \zc{k}} & = & \bigo(\norma{g_p(\zc{k},\mu^k)})
      \visiblelbl{hip:difz}, \\ 
    \norma{\lk{k} - \lls(\zc{k},\mu^k)} & = & \bigo(\norma{g_p(\zc{k},\mu^k)})
      \visiblelbl{hip:difl}, \\ 
    (\lambda^{k+1}_I)^T(s_c^{k+1}-s^k) & = &
      \bigo(\norma{g_p(\zc{k},\mu^k)}\rho^k), \visiblelbl{hip:gap}
\end{eqnarray} 
    ent\~ao (\ref{rhomaxeq}) \'e satisfeito, independentemente do
    valor de $\lim\inf\norma{g_p(\zc{k},\mu^k)}$. Em outras palavras,
    $\rhomax^k$ permanece suficientemente afastado de zero.  
  \end{description} 
\end{lemma}
\begin{proof} 
  \begin{description} 
    \item[(i)] Para provar que $\rhomax$ n\~ao muda na itera\c{c}\~ao $k+1$,
      s\'o precisamos mostrar que $\DLV{k+1} < -\DLH{k}/2$, o que, segundo o
      Lema \ref{lemma:32}, é obtido quando
  \begin{equation} 
    \DLV{k+1} \leq
    \frac{\xi_1}{2}\norma{g_p(\zc{k},\mu^k)}\min\{\xi_2\norma{g_p(\zc{k},
    \mu^k)},\xi_3\sqrt{\rho^k},1-\epsmu\}.\visiblelbl{eq:DLVlema2}
\end{equation} 
Para isso, tomamos 
\begin{equation} \visiblelbl{xi56}
  \xi_5\leq\min\bigg\{\frac{\xi_1\xi_2}{2\xi_4},\frac{10^{-4}\xi_1^2\xi_3^2}
  {4\xi_4^2(\xi_0+1)}\bigg\} \qquad \mbox{ e } \qquad \xi_6 \leq
  \frac{\xi_1(1-\epsmu)} {2\xi_4}, 
\end{equation} 
onde $\xi_4$ é a constante definida no Lema \ref{lemma:33}. Assim, a partir do Lema
\ref{lemma:33}, de (\ref{hip:limrhomax}) e (\ref{xi56}), obtemos
\begin{equation}\visiblelbl{dlvaux1} 
  \DLV{k+1} \leq \xi_4\rhomax^k\norma{\gpk{k}} \leq
    \xi_4\xi_5\norma{\gpk{k}}^2 \leq \meio\xi_1\xi_2\norma{\gpk{k}}^2.  
\end{equation} 
Além disso, usando (\ref{limrhomax}) e (\ref{limgamma}), temos
\begin{equation}\visiblelbl{sqrtrhomax} 
  \sqrt{\rhomax^k} \leq 10^2\sqrt{\rho^k(\xi_0+1)} \norma{\gpk{k}}^{-1/2}.  
\end{equation}
Extraindo a raiz quadrada dos dois lados de (\ref{hip:limrhomax}) e combinando o
resultado com (\ref{sqrtrhomax}), obtemos
\begin{equation}\visiblelbl{rhomax35} 
  \rhomax^k \leq \sqrt{\xi_5}10^2\sqrt{\xi_0+1}\sqrt{\rho^k}.  
\end{equation} 
  Usando o Lema \ref{lemma:33}, (\ref{rhomax35}) e (\ref{xi56}), temos
\begin{equation}\visiblelbl{dlvaux2} 
  \DLV{k+1} \leq \xi_4\rhomax^k\norma{\gpk{k}} \leq
  \xi_4\sqrt{\xi_5}10^2\sqrt{\xi_0+1}\sqrt{\rho^k}\norma{\gpk{k}} \leq
\meio\xi_1\xi_3\norma{\gpk{k}}\sqrt{\rho^k}. 
\end{equation} 
Finalmente, usando o Lema \ref{lemma:33}, (\ref{hip:limrhomax}) e (\ref{xi56}),
temos 
\begin{equation}\visiblelbl{dlvaux3} 
  \DLV{k+1} \leq \xi_4\rhomax^k\norma{\gpk{k}} \leq \xi_4\xi_6\norma{\gpk{k}}
  \leq \meio\xi_1(1-\epsmu)\norma{\gpk{k}}.  
\end{equation} 
O resultado segue de (\ref{dlvaux1}), (\ref{dlvaux2}) e (\ref{dlvaux3}).
\item[(ii)] Sejam $b = \lim\inf(\norma{\gpk{k}})$ e $\barra{k}_0$ um índice tal
  que $\norma{\gpk{k}} > b/2$, $\forall k \geq  \barra{k}_0$. Tome $k \geq
  \barra{k}_0$. Nesse caso, existem duas situações possíveis: ou existe $k$ tal
  que $\rhomax^k < \min\{\xi_5b/2,\xi_6\}$, ou $\rhomax^k \geq
  \min\{\xi_5b/2,\xi_6\}, \forall k$.
  Na primeira situação, pelo item (i), $\rhomax^{k+1} = \rhomax^{k}$ e, a partir
  deste $k$, $\rhomax$ não muda mais. Desta forma, basta definirmos este $k$
  como $k_0$. Na segunda situação, $\rhomax$ permance limitado inferiormente.
  Portanto, como $\rhomax^k = \rhomax^02^{-j}$, para algum $j\in\mathbb{N}$, a
  partir de uma certa iteração $k_0$, $\rhomax$ não irá descrescer mais.

\item[(iii)] Note que, pelas defini\c{c}\~oes de $\lls$ e $g_p$, e pelas
  hip\'oteses \ref{hip:global.continuidade}-\ref{hip:global.regularidade},
  $\lls(z,\mu)$ e $g_p(z,\mu)$ est\~ao bem definidos e são de classe $C^1$ em
  uma vizinhan\c{c}a compacta de $\barra{\mathcal{Z}}$, o fecho de $\mathcal{Z}
  = \{z_c^k\}$. Portanto, $\lls$ e $g_p$ s\~ao Lipschitz cont\'inuas no sentido
  que 
\begin{subequations} 
  \begin{eqnarray} 
    \norma{\lls(\zc{k+1},\mu) - \lls(\zc{k},\mu)} & = &
      \bigo(\norma{\zc{k+1}-\zc{k}}), \visiblelbl{llslip.a} \\
    \norma{\lls(\zc{k},\mu_1) - \lls(\zc{k},\mu_2)} & = &
      \bigo(\modulo{\mu_1-\mu_2}), \visiblelbl{llslip.b} 
\end{eqnarray}
\end{subequations}
  e
  \begin{subequations} 
  \begin{eqnarray}
    \norma{g_p(\zc{k+1},\mu) - g_p(\zc{k},\mu)} & = &
      \bigo(\norma{\zc{k+1}-\zc{k}}), \visiblelbl{gplip.a} \\
    \norma{g_p(\zc{k},\mu_1) - g_p(\zc{k},\mu_2)} & = &
      \bigo(\modulo{\mu_1-\mu_2}), \visiblelbl{gplip.b} 
\end{eqnarray}
  \end{subequations} 
  De (\ref{nnormal}), (\ref{hip:difz}), (\ref{limhhornormal}) e (\ref{limrho}),
  temos 
  \begin{eqnarray} 
    \norma{\zc{k+1}-\zc{k}} & \leq & \norma{\zc{k+1}-z^k} + \norma{z^k -
    \zc{k}} \nonumber \\ 
    & = & \bigo(\norma{h(z^k)}) + \bigo(\norma{g_p(\zc{k},\mu^k)}) \nonumber \\
    & = & \bigo(\rho^k) + \bigo(\norma{g_p(\zc{k},\mu^k)}) \nonumber \\ 
    & = & \bigo(\norma{g_p(\zc{k},\mu^k)}), \visiblelbl{difzck} 
\end{eqnarray} 
  e de (\ref{gplip.a}) e (\ref{difzck}), obtemos 
  \begin{eqnarray}
    \norma{g_p(\zc{k+1},\mu^k)} & \leq & \norma{g_p(\zc{k},\mu^k)} +
      \bigo(\norma{\zc{k+1}-\zc{k}}) \nonumber \\ 
    & = & \bigo (\norma{g_p(\zc{k},\mu^k)}). \visiblelbl{gpkplus} 
\end{eqnarray} 
  Agora, usando (\ref{gplip.b}), \eqref{limmu}, \eqref{gpkplus} e \eqref{limrho},
  obtemos
  \begin{eqnarray} 
    \norma{g_p(\zc{k+1},\mu^{k+1})} & \leq & \norma{g_p(\zc{k+1},\mu^k)} +
      \bigo(\modulo{\mu^k - \mu^{k+1}}) \nonumber \\ 
    & = & \bigo(\norma{g_p(\zc{k+1},\mu^k)}) + \bigo(\rho^k) \nonumber \\ 
    & = & \bigo(\norma{g_p(\zc{k},\mu^k)}). \visiblelbl{gpmukplus}
\end{eqnarray} 
Pelas definições \eqref{def:lagrangeano}, \eqref{def:phi}, \eqref{def:h} e
\eqref{def:lagrgeral}, a variação do Lagrangeano no passo normal da iteração $k$
pode ser escrita como
\begin{eqnarray}
  \DLV{k+1} 
    & = & L(\zc{k+1},\lk{k+1},\mu^{k+1}) - L(z^{k}, \lk{k}, \mu^k) \nonumber \\
    & = & f(\xc{k+1}) + \mu^{k+1}\beta(\zc{k+1}) + (\lk{k+1})^Th(\zc{k+1}) -
      \nonumber \\
    & & f(x^k) - \mu^k\beta(z^k) - (\lk{k})^T h(z^k) \nonumber \\
    & = & \mathcal{L}(\xc{k+1}, \lk{k+1}) - \mathcal{L}(x^k,\lk{k+1}) +
      \mu^{k+1}\beta(\zc{k+1}) - \mu^k\beta(z^k) + \nonumber \\
    & & [\lambda^{k+1} - \lambda^k]^Th(z^k) - (\lk{k+1}_I)^T(s_c^{k+1} - s^k).
      \visiblelbl{dlvaux4}
\end{eqnarray}
Usando uma expansão de Taylor, 
\eqref{def:gpk}, a Hipótese \ref{hip:global.regularidade},
\eqref{gradproj}, \eqref{hip:difl}, \eqref{gpmukplus}, 
\eqref{limhess}, \eqref{limhessh}, \eqref{limlambda},
\eqref{nnormal}, \eqref{limhhornormal} e \eqref{limrho},
garantimos que existe $x_\xi$ tal que
\begin{eqnarray}
  \mathcal{L}(\xc{k+1},\lk{k+1}) - \mathcal{L}(x^k,\lk{k+1})
  & = &\nabla_x \mathcal{L}(\xc{k+1},\lk{k+1})^T(\xc{k+1} - x^k) + \nonumber \\ 
  & & \meio (\xc{k+1}-x^k)^T\nabla_{xx}^2\mathcal{L} (x_{\xi},\lk{k+1})
    (\xc{k+1} - x^k) \nonumber \\
  & \leq & \norma{\vetord{I}{0}g_p^{k+1}}\norma{\xc{k+1}-x^k} +
    \meio(\xi_0+m_I\xi_0^2)\norma{\xc{k+1}-x^k}^2 \nonumber \\
    & = & \bigo(\norma{g_p(\zc{k},\mu^k)}^2) \visiblelbl{termo1.dlvaux4}
\end{eqnarray}

Usando uma expansão de Taylor,
\eqref{def:beta},
\eqref{zcompact}, \eqref{zbound}, \eqref{limmu} e \eqref{limrho}, temos
\begin{eqnarray}
  \mu^{k+1}\beta(\zc{k+1}) - \mu^k\beta(z^k) 
  & = & \mu^{k+1}[\beta(z^k) + \nabla \beta(z_{\xi})^T(\zc{k+1}-z^k)] 
    - \mu^k\beta(z^k) \nonumber \\
  & = & (\mu^{k}-\mu^{k+1})\sum_{i=1}^{m_I}\log(s_i^k) - \nonumber \\
  & & \mu^{k+1} e^T [\eta S^k + (1-\eta) S_c^{k+1}]^{-1}(s_c^{k+1}-s^k) \nonumber \\
  & = & \bigo(\mu^k) + \mu^{k+1}\sum_{i=1}^{m_I} \frac{s_i^k - s_{c_i}^{k+1}}
    {\eta s_i^k + (1-\eta)s_{c_i}^{k+1}} \nonumber \\
  & \leq & \bigo(\mu^k) + \mu^{k+1}m_I\frac{1-\epsmu}{\eta + (1-\eta)\epsmu}
    \nonumber \\
    & = & \bigo(\mu^k) = \bigo(\norma{g_p(\zc{k},\mu^k)}\rho^k).
    \visiblelbl{termo2.dlvaux4}
\end{eqnarray}

Usamos \eqref{hip:difl}, \eqref{gpmukplus},
\eqref{llslip.b}, \eqref{limmu}, \eqref{limrho},
\eqref{llslip.a},
\eqref{difzck} e \eqref{limhhornormal},
obtendo 
\begin{eqnarray}
  [\lk{k+1} - \lk{k}]^Th(z^k) & \leq & 
  \norma{\lk{k+1} - \lls(\zc{k+1},\mu^{k+1})}\norma{h(z^k)} + \nonumber \\
  & & \norma{\lls(\zc{k+1},\mu^{k+1}) - \lls(\zc{k+1},\mu^k)}\norma{h(z^k)} +
    \nonumber \\
  & & \norma{\lls(\zc{k+1},\mu^k) - \lls(\zc{k},\mu^k)}\norma{h(z^k)} + \nonumber \\
  & & \norma{\lls(\zc{k},\mu^k) - \lk{k}} \norma{h(z^k)} \nonumber \\
  & = & \bigo(\norma{g_p(\zc{k+1},\mu^{k+1})}\rho^k) +
    \bigo(\mu^k\rho^k) + 
    \bigo(\norma{g_p(\zc{k},\mu^k)}\rho^k) \nonumber \\
    & = & \bigo(\norma{g_p(\zc{k},\mu^k)}\rho^k). \visiblelbl{termo3.dlvaux4}
\end{eqnarray}

Assim, substituindo \eqref{termo1.dlvaux4}, \eqref{termo2.dlvaux4} e 
\eqref{termo3.dlvaux4} em \eqref{dlvaux4}, obtemos 
\begin{eqnarray*} \DLV{k+1} & = &
\bigo(\norma{g_p(\zc{k},\mu^k)}\rho^k) + \bigo(\rho^{k^2}) \\
& = & \bigo(\norma{g_p(\zc{k},\mu^k)}\rho^k) + \bigo(\rhomax^k\norma{g_p(\zc{k},
\mu^k)}\rho^k) \\
& = & \bigo(\norma{g_p(\zc{k},\mu^k)}\rho^k) =
\bigo(\norma{g_p(\zc{k},\mu^k)}^2\rhomax^k). 
\end{eqnarray*} 
Portanto, existe $\xi_7 > 0$ tal que 
\begin{eqnarray} \DLV{k+1} \leq
\xi_7\rhomax^k\norma{g_p(\zc{k},\mu^k)}^2.\visiblelbl{limdlvp} \end{eqnarray}
Vamos, agora, mostrar que, também no caso (iii), \eqref{eq:DLVlema2} é
satisfeito. Para tanto,
definimos $\bxi_7 = \xi_0 + \mu_{\max}m_I + (n+m_I)\xi_0^2$ e
\begin{eqnarray} \brm =
  \min\bigg\{\frac{\xi_1\xi_2}{2\xi_7},\frac{10^{-4}}{4\bxi_7(\xi_0 + 1)}
  \bigg(\frac{\xi_1\xi_3} {\xi_7}\bigg)^2,
  \frac{\xi_1(1-\epsmu)}{2\xi_7\bxi_7}\bigg\}.\visiblelbl{def:brm} 
\end{eqnarray}
Assim, tomando $k \geq k_0$ tal que $\rhomax^{k_0} \leq \brm$, e usando
\eqref{limdlvp} e \eqref{def:brm}, temos 
\begin{eqnarray*} 
  \DLV{k+1} & < &
  \xi_7\bigg(\frac{\xi_1\xi_2}{2\xi_7}\bigg)\norma{g_p(\zc{k},\mu^k)}^2 \\
  & = & \frac{\xi_1\xi_2}{2}\norma{g_p(\zc{k},\mu^k)}^2.  
\end{eqnarray*} 
Além disso, note que, por \eqref{limgamma}, \eqref{limjacob} e
\eqref{limlambda}, temos
\begin{eqnarray*} 
  \norma{g_p(\zc{k},\mu^k)} & = & \norma{g(\zc{k}) +
    A(\zc{k})^T\lls(\zc{k},\mu^k)} \\
  & \leq & \norma{g(\zc{k})} +
    \sum_{i=1}^{n+m_I}\norma{\lambda_{LS_i}(\zc{k},\mu^k) A_i(\zc{k})} \\
  & \leq & \xi_0 + \mu m_I + (n + m_I) \xi_0^2  \\
  & \leq & \xi_0 + \mu_{\max}m_I +
    (n+m_I)\xi_0^2 = \barra{\xi}_7.
\end{eqnarray*} 
Assim, usando \eqref{limdlvp}, \eqref{def:brm},
\eqref{limrhomax} e \eqref{limgamma}, temos 
\begin{eqnarray*} 
  \DLV{k+1} & = & \xi_7\sqrt{\rhomax^k}
    \sqrt{\rhomax^k}\norma{g_p(\zc{k},\mu^k)}^2 \\ 
  & \leq & \xi_7\bigg(\frac{\xi_1\xi_3}{\xi_7}
    \frac{10^{-2}\norma{g_p(\zc{k},\mu^k)}^2} {2\sqrt{\barra{\xi}_7(\xi_0+1)} } \bigg)
    \bigg(10^2\sqrt{\rho^k}\frac{\sqrt{\norma{g(\zc{k},\mu^k)} + 1}}
    {\sqrt{\norma{\gpk{k}}}}\bigg) \\ 
  & \leq & \frac{\xi_1\xi_3}{2} \sqrt{\rho^k}
    \frac{\norma{g_p(\zc{k},\mu^k)}^{3/2}} { \sqrt{\barra{\xi}_7} } \\ 
  & \leq & \frac{\xi_1\xi_3}{2}\sqrt{\rho^k}\norma{g_p(\zc{k},\mu^k)}. 
\end{eqnarray*}
    Finalmente, usando \eqref{limdlvp} e \eqref{def:brm}, temos
\begin{eqnarray*} 
  \DLV{k+1} & < & \xi_7\frac{\xi_1(1 - \epsmu)}
    {2\xi_7\bxi_7}\norma{g_p(\zc{k},\mu^k)}^2 \\ 
  & \leq & \frac{\xi_1(1 - \epsmu)}{2}\norma{g_p(\zc{k},\mu^k)}. 
\end{eqnarray*} 
Desse modo, temos
  $\DLV{k+1} < -\meio\DLH{k}$. Logo $\rhomax^k$ não muda depois de $k_0$.
  \end{description} \end{proof}

Apresentamos agora o teorema de convergência global de nosso algoritmo.
\begin{theorem}\label{teo:conv-global} 
  Sob as Hipóteses \ref{hip:global.continuidade}-\ref{hip:global.dsoc}, se o
  método CDI gera uma sequência infinita então existe uma subsequência
  convergente a um ponto estacionário para \eqref{prob:geral}.
  Se as condições (\ref{hip:difz}), (\ref{hip:difl}) e (\ref{hip:gap})
  também são satisfeitas, então toda subsequência convergente de $\{\xc{k}\}$ tem
  ponto limite estacionário para (\ref{prob:geral}).  
\end{theorem} 
\begin{proof} Suponha, por
  contradição, que $\lim\inf(\norma{g_p(\zc{k},\mu^k)}) = 2b > 0$. Seja
  $\barra{k}_0 \in \mathbb{N}$ tal que $\norma{g_p(\zc{k},\mu^k)} > b$ para
  qualquer $k > \barra{k}_0$. Assim, pelo item (ii) do Lema \ref{lemma:35},
  existe $k_0 \geq \barra{k}_0$ tal que, para todo $k \geq k_0$, $\rhomax^k =
  \rhomax^{k_0}$. Junto com \eqref{limrhomax} e \eqref{limgamma}, isto implica
  que $$\rho^k \geq 10^{-4}\frac{\rhomax^{k_0}b}{\xi_0 + 1},$$ e, portanto, para
  qualquer $i \geq k_0$, usando \eqref{limDLHp}, temos $\DLH{i} \leq -\theta$,
  onde \begin{eqnarray} \theta =
    \xi_1b\min\bigg\{\xi_2b,10^{-2}\xi_3\sqrt{\frac{\rhomax^{k_0}b}
  {\xi_0+1}},1-\epsmu\bigg\} > 0. \visiblelbl{def:theta} \end{eqnarray} Agora,
  usando \eqref{difL} e \eqref{def:theta}, podemos garantir que, para $k > k_0$,
  \begin{eqnarray*} L(\zc{k},\lk{k},\mu^k) - L(\zc{k_0},\lk{k_0},\mu^{k_0}) &
  = & \sum_{i = k_0+1}^k\Delta L_c^i \\ & \leq &
  \frac{1}{4}\sum_{i=k_0}^{k-1}\DLH{i} + r^{k_0} \\ & \leq & -\frac{1}{4}(k -
  k_0)\theta + r^{k_0} \longrightarrow -\infty.  \end{eqnarray*} Isso implica
  que $f$ é descontínua, ou $h$ é descontínua, ou uma das sequências,
  $\{\zc{k}\}$ ou $\{\lk{k}\}$, não é limitada, contrariando
  \ref{hip:global.continuidade}-\ref{hip:global.seq.limitadas}. Portanto,
  $\lim\inf(\norma{g_p(\zc{k}, \mu^k)}) = 0$.

Para a segunda parte do teorema, suponhamos válidas (\ref{hip:difz}),
(\ref{hip:difl}) e (\ref{hip:gap}). Então, pelo Lema \ref{lemma:35}, existe
$k_0$ tal que $\rhomax^k = \rhomax^{k_0}$, para todo $k \geq k_0$.

Suponha, por absurdo, que $\norma{g_p(\zc{k_l},\mu^{k_l})} \geq b > 0$ para uma
subsequência infinita $\{k_l\}$. Neste caso, usando \eqref{difL} e
\eqref{def:theta} novamente, e fazendo $n_k \longrightarrow \infty$, temos
$$L(\zc{k},\lk{k},\mu^k)-L(\zc{k_0},\lk{k_0},\mu^{k_0}) \leq
-\frac{1}{4}n_k\theta + r^{k_0} \longrightarrow -\infty,$$ onde $\theta$ é dado
por (\ref{def:theta}). Isso também contradiz
\ref{hip:global.continuidade}-\ref{hip:global.seq.limitadas}, implicando que
$\norma{g_p(\zc{k_l},\mu^{k_l})} \longrightarrow 0$ para toda subsequência de
$\{\zc{k}\}$.

Seja $\{k_l\}$ uma subsequência convergente tal que
$\norma{g_p(\zc{k_l},\mu^{k_l})} \longrightarrow 0$. Pelas propriedades do
algoritmo descritas em \eqref{limrho}-\eqref{limhhornormal} e \eqref{limmu}, temos
$\rho^{k_l} \longrightarrow 0$, $\norma{h(\zc{k_l})} \longrightarrow 0$ e
$\mu^{k_l} \longrightarrow 0$. Pela definição de $g_p(\zc{k_l},\mu^{k_l})$ temos
$-\mu^{k_l}e - S_c^{k_l}\lambda_I^{k_l} \longrightarrow 0$, de modo que
$S_c^{k_l}\lambda_I^{k_l} \longrightarrow 0$, e $\nabla f(\xc{k_l}) + \nabla
c(\xc{k_l})^T\lk{k_l} \longrightarrow 0$. Além disso, $\lambda_I^{k_l} \leq
\alpha(\mu^{k_l})^n$, de modo que $\lim \lambda_I^{k_l} \leq 0$. Portanto, o ponto
limite dessa subsequência é estacionário.  \end{proof}

\section{Convergência Local}\visiblelbl{sec:conv-local}

Agora vamos analisar as propriedades do métodos quando obtemos uma sequência
convergente, com algumas condições adicionais. Mostraremos que o método tem
convergência superlinear em dois passos. Para isso, vamos analisar o método em
função das restrições ativas na solução. Pediremos que as condições de segunda
ordem tradicionais sejam satisfeitas e algumas condições extras, em geral
específicas para o método.

Sejam $\{z^k\}$ e $\{\zc{k}\}$ sequências geradas pelo algoritmo, convergentes
a $z^*$. Seja $\{\lk{k}\}$ convergente a $\lambda^* = \lls(z^*,0)$. Pelo
algoritmo, temos

\begin{eqnarray*}
 \left\{
\begin{array}{rcl}
 \nabla f(x^*) + \nabla c(x^*)^T\lambda^* & = & 0, \\
 c_E(x^*) & = & 0, \\
 c_I(x^*) & \geq & 0, \\
 c_I(x^*)^T\lambda_I^* & = & 0, \\
 \lambda_I^* & \leq & 0.
\end{array}
\right.
\end{eqnarray*}
Defina $\calA(x) = \{i \in E\cup I : c_i(x) = 0\}$, e $\calA^* = \calA(x^*)$.
Defina $\lA^k$ e $\lA^*$ como as componentes de $\lambda^k$ e $\lambda^*$,
respectivamente, correspondentes às restrições ativas.
Suponha que $x^*$ é um ``bom minimizador'' para \eqref{prob:geral}, no sentido
de que $V = \{\nabla c_i(x^*) : i \in \calA^*\}$ é L.I., e
\begin{eqnarray}
 y^T\bigg[\nabla^2 f(x^*) + \sum_{i \in \calA^*}\nabla^2c_i(x^*)\lk{*}_i\bigg]y =
 y^T\bigg[\nabla^2 f(x^*) + \sum_{i = 1}^m\nabla^2 c_i(x^*)\lambda_i^*\bigg]y \geq
 \theta_1\norma{y}^2, \visiblelbl{eq:defpos}
\end{eqnarray}
com $\theta_1 > 0$ e $y \in T = \{w : w^T\nabla c_i(x^*) = 0 : i \in E \cup J\}$, onde $J = \{i \in I:
 \lambda_i^* < 0\}$.
Defina a matriz $\nabla c_A(x^*)$, cujas linhas são os vetores de $V$, e a matriz $\nabla
 c_B(x^*)$, cujas linhas são os vetores que faltam para completar $\nabla c(x^*)$.
Como $\nabla c_A(x)$ é contínua, numa vizinhança de $x^*$, $\nabla c_A(x)$ tem posto linha
completo (ver \ref{app:prova-posto-vizinhanca}). Assim, podemos definir
$$\lA(x) = -[\nabla c_A(x)\nabla c_A(x)^T]^{-1}\nabla c_A(x) \nabla f(x),$$
$$g_A(x) = \nabla f(x) + \nabla c_A(x)^T\lA(x),$$
e 
$$H_A(x,\lambda) = \nabla^2 f(x) + \sum_{i \in \calAe}\nabla^2 c_i(x)\lambda_i.$$

Note que $\lambda_A(\xc{k})\tende\lambda_A^*$.
 Numa vizinhança $V^*$ de $x^*$, podemos definir o projetor ortogonal no núcleo de $\nabla
 c_A(x)$ como $$P(x) = I - \nabla c_A(x)^T[\nabla c_A(x)\nabla c_A(x)^T]^{-1}\nabla c_A(x),$$ e
 afirmamos que ele é Lipschitz contínuo, pois $c \in C^2$. Também definimos o passo completo
 $\delta_c^k = \zc{k+1}-\zc{k} = \delta_T^k + \delta_N^{k+1}$.

Além de considerar as hipóteses \ref{hip:global.continuidade}-\ref{hip:global.dsoc}, nossa
 análise da convergência local de $\xc{k}$ e $x^k$
 será baseada em cinco hipóteses locais. Quatro delas estão descritas a seguir.
 A quinta será descrita mais adiante.
\begin{hypoenv}\visiblelbl{hip:local.lls}
\begin{eqnarray*}
\norma{\lk{k} - \lls(\zc{k},\muc{k})} & = & \bigo(\norma{g_p(\zc{k},\muc{k})}), \\
  (\lk{k+1}_I)^T(s_c^{k+1} - s^k) & = & \bigo(\norma{g_p(\zc{k},\muc{k})}\rho^k)
\end{eqnarray*}
\end{hypoenv}
\begin{hypoenv}\visiblelbl{hip:local.bkdefpos}
 $B^k$ é assintoticamente uniformemente definida positiva em $\Nu(A(\zc{k}))$, isto é, em
alguma vizinhança de $z^*$, podemos redefinir $\theta_1$ de modo que
\begin{eqnarray}
 \theta_1\norma{y}^2 \leq y^TB^k y \leq \theta_2\norma{y}^2, \visiblelbl{eq:rayleigh}
\end{eqnarray}
para $y \in \Nu(A(\zc{k}))$, onde $\theta_2 = \xi_0$.
\end{hypoenv}
\begin{hypoenv}\visiblelbl{hip:local.horz.step}
 Seja $Z_A^k$ uma matriz cujas colunas formam uma base ortonormal para o núcleo de
 $\nabla c_A(\xc{k})$. Defina
\begin{eqnarray}
\delta_x^k = -Z_A^k[(Z_A^k)^T B_x^k Z_A^k]^{-1}(Z_A^k)^Tg_A(\xc{k}), \visiblelbl{eq:dhndefx} \\
\delta_{s_i}^k = \frac{1}{s_{c_i}^k}\nabla c_{I_i}(\xc{k})^T\delta_x^k, \visiblelbl{eq:dhndefs}
\end{eqnarray}
e
$$\delta_A^k = \vetor{\delta_x^k}{\delta_{s}^k}.$$
Assumimos que $\delta_A^k$ é o primeiro passo tentado pelo algoritmo se $\norma{\delta_A^k}
 \leq \Delta$ e $s_c^k + S_c^k\delta_s^k \geq \epsmu s_c^k$. Além disso, supomos que
\begin{eqnarray}
 P(\xc{k})[B_x^k - H_A(x^*,\lambda^*)]\delta_x^k = o(\norma{\delta_x^k}).
 \label{eq:proj_tangente}
\end{eqnarray}
\end{hypoenv}
Note que se $s_{c_i}^k \tende 0$, então $i \in \mathcal{A}^*$. Daí, $\nabla
c_{I_i}(\xc{k})^T$ é
uma das linhas de $\nabla c_A(\xc{k})$, de modo que $\nabla c_{I_i}(\xc{k})^T Z_A^k = 0$.
Logo, $\delta_{s_i}^k$ é zero. Caso contrário, $s_{c_i}^k$ permanece afastado de
zero. Portanto $\delta_s^k$ é limitado. Assim, definimos
 $\smin > 0$ tal que se $i \not\in \mathcal{A}^*$, então $s_{c_i}^k \geq \smin$.
\begin{hypoenv}\visiblelbl{hip:local.active-relations}
 Existem constantes positivas $\theta_A$ e $\theta_p$, tais que para $k$
 suficientemente grande, 
$$\norma{\gAk{k}} \leq \theta_A\norma{\gpk{k}},$$ 
$$\norma{\gpk{k}} \leq \theta_p\norma{\gAk{k}},$$ 
$$\norma{c_A(\xc{k})} = \Theta(\norma{h(\zc{k})}),$$
$$\norma{c_A(x^{k})} = \Theta(\norma{h(z^{k})}),$$
$$\norma{\xc{k+1} - x^k} = \bigo(\norma{c_A(x^k)}).$$
\end{hypoenv}

Como $\nabla c_A(x)$ e $H_A(x,\lambda)$ são contínuas e $\nabla c_A(x^*)$ tem
posto completo, nossas hipóteses implicam que existe $\theta_3 > 0$ e uma
vizinhança $V^*$ de $x^*$, tal que, para $x$, $\xc{k} \in V^*$,
\begin{eqnarray}
\norma{\nabla h(z)^T\lambda} & \geq & \theta_3\norma{\lambda}, \qquad \lambda \in \Rn{n+m_I},
\ s \in \Rn{m_I}_+. \visiblelbl{eq:svd}
\end{eqnarray}
Usando \eqref{eq:proj_tangente} e a continuidade de $H_A$, obtemos
\begin{eqnarray}
P(\xc{k})[B_x^k - H_A(\xc{k},\lA(\xc{k}))]\delta_x^k & = &
 P(\xc{k})[B_x^k - H_A(\xc{k},\lA(\xc{k})) + H_A(x^*,\lk{*}) - H_A(x^*,\lk{*})] 
\delta_x^k \nonumber \\
& = & o(\norma{\delta_x^k}) + P(\xc{k})[H_A(x^*,\lk{*}) - H_A(\xc{k},\lA(\xc{k}))]\delta_x^k
\nonumber \\
& = & o(\norma{\delta_x^k}) + P(\xc{k})[o(1)]\delta_x^k \nonumber \\
& = & o(\norma{\delta_x^k}). \visiblelbl{eq:PBWlls}
\end{eqnarray}
e
\begin{eqnarray}
P(\xc{k})[B_x^k - \nabla_{xx}^2 \mathcal{L}(\xc{k},\lk{k})]\delta_x^k & = &
P(\xc{k})[B_x^k - \nabla_{xx}^2 \mathcal{L}(\xc{k},\lk{k}) + H_A(x^*,\lk{*}) - H_A(x^*,\lk{*})]
\delta_x^k \nonumber \\
& = & o(\norma{\delta_x^k}) + P(\xc{k})[H_A(x^*,\lk{*}) - \nabla_{xx}^2 
\mathcal{L}(\xc{k},\lk{k})]\delta_x^k \nonumber \\
& = & o(\norma{\delta_x^k}) + P(\xc{k})[o(1)]\delta_x^k \nonumber \\
& = & o(\norma{\delta_x^k}). \visiblelbl{eq:PBWlk}
\end{eqnarray}

Podemos escrever $\delta_x^k = Z_A^k\nu^k \in \Nu(\nabla c_A(\xc{k}))$. Note que $\nu^k$ é
 minimizador do problema
$$\min \barra{q}_x^k(\nu) = q_x^k(Z_A^k\nu) = \meio\nu^T(Z_A^k)^TB_x^kZ_A^k\nu +
 \nu^T(Z_A^k)^T\nabla f(\xc{k}),$$
onde $q_x(\delta)$ é definido como
$$q_x^k(\delta) = \meio\delta^TB_x^k\delta + \delta^T \nabla f(x_c^k),$$
isto é, a parte de $q^k(\delta)$ que não envolve $s$ ou $\mu$.
Então, como $(Z_A^k)^T\nabla f(\xc{k}) = (Z_A^k)^Tg_A(\xc{k})$, temos
\begin{eqnarray}
 (Z_A^k)^T(B_x^k Z_A^k\nu^k + g_A(\xc{k})) = 0.\visiblelbl{eq:solnu}
\end{eqnarray}

Por \eqref{eq:rayleigh} e o fato de que $(Z_A^k)^TZ_A^k = I$, a matriz $[(Z_A^k)^T B_x^k
 Z_A^k]^{-1}$ satisfaz, na vizinhança $V^*$, para todo $u \in \Rn{\barra{n}-m_A}$,
 \begin{eqnarray}
 \frac{1}{\mu_2}\norma{u}^2 \leq u^T[(Z_A^k)^T B_x^k Z_A^k]^{-1}u \leq \frac{1}
{\mu_1}\norma{u}^2.\visiblelbl{eq:rayinversa}
\end{eqnarray}

No próximo lema, mostraremos que o passo $\delta_A^k$ é aceito pelo algoritmo.

\begin{lemma}\visiblelbl{lemma:horz.step.accepted}
 O passo $\delta_A^k$ é aceito pelo algoritmo \ref{alg:outline}, para $k$
 suficientemente grande.
\end{lemma}
\begin{proof}
 Como $\gAk{k} \in \Nu(\nabla c_A(\xc{k}))$, existe $\nu_p^k$ tal que $\gAk{k} =
 Z_A^k\nu_p^k$ com $\norma{\gAk{k}}=\norma{\nu_p^k}$. Combinando
 \eqref{eq:dhndefx} e \eqref{eq:rayinversa} temos
\begin{eqnarray}
 \norma{\delta_x^k} & = & \norma{Z_A^k[(Z_A^k)^TB_x^kZ_A^k]^{-1}(Z_A^k)^T\gAk{k}} =
 \norma{[(Z_A^k)^T B_x^k Z_A^k]^{-1}(Z_A^k)^T\gAk{k}} \nonumber \\
& \leq & \frac{1}{\mu_1}\norma{(Z_A^k)^T\gAk{k}} = \frac{1}{\mu_1}\norma{\nu_p^k} \nonumber \\ 
& = & \frac{1}{\mu_1}\norma{\gAk{k}},\visiblelbl{dxgAk}
\end{eqnarray}
Além disso, para $i \not\in \calAe$,
\begin{eqnarray*}
 \modulo{\delta_{A_i}^k} & = & \bigg|\frac{\nabla c_i (\xc{k})^T\delta_x^k}{s_i}\bigg| \leq
 \frac{\xi_0}{\smin}\norma{\delta_x^k} \leq \frac{\xi_0}{\mu_1\smin}\norma{\gAk{k}},
\end{eqnarray*}
de modo que
\begin{eqnarray}
 \norma{\delta_s^k} \leq \frac{m_A\xi_0}{\mu_1\smin}\norma{\gAk{k}}. \visiblelbl{dsgAk}
\end{eqnarray}
Portanto,
\begin{eqnarray}
 \norma{\delta_A^k} \leq \theta_4\norma{\gAk{k}}, \visiblelbl{eq:dhnlim}
\end{eqnarray}
onde $\theta_4 = \sqrt{1 + (m_A\xi_0/\smin)^2}/\mu_1$, para $k$ suficientemente grande. Como
 $\gAk{k}\tende 0$, para $k$ suficientemente grande, temos $\norma{\delta_A^k} < \Delta_{\min}$
 e $\delta_s^k \geq (\epsmu - 1)e$, de modo que, pela Hipótese
 \ref{hip:local.horz.step}, $\delta_A^k$ será o primeiro passo tentado pelo
 algoritmo, em alguma vizinhança $V^*$.

Para $k$ suficientemente grande, $\barra{q}_x(\nu)$ será uma quadrática definida positiva.
 Portanto, seu mínimo, $\barra{q}_x(\nu^k)$, satisfaz
\begin{eqnarray}
 q_x(\delta_x^k) & = & \barra{q}_x(\nu^k) \nonumber \\
& = & -\meio(\gAk{k})^TZ_A^k[(Z_A^k)^T B_x^k Z_A^k]^{-1}(Z_A^k)^T\gAk{k} \nonumber \\
& \leq & - \frac{1}{2\mu_2}\norma{\gAk{k}}^2, \visiblelbl{eq:qdhnx}
\end{eqnarray}
onde a última desigualdade vem de \eqref{eq:rayinversa}.
Além disso, temos
\begin{eqnarray}
 q_s(\delta_s^k) & = & \meio\delta_s^TB_s\delta_s + \delta_s^T(-\mu e) \nonumber \\
& = & \meio \sum_{i \not\in\mathcal{A}^*}B_{s_i}\delta_{s_i}^2 - 
\mu\sum_{i \not\in\mathcal{A}^*}\delta_{s_i} \nonumber \\
& = & \meio\sum_{i \not\in\mathcal{A}^*}\frac{B_{s_i}}{s_i^2} \nabla c_i(\xc{k})^T\delta_x -
\mu \sum_{i \not\in\mathcal{A}^*}\frac{ \nabla c_i(\xc{k})^T\delta_x }{s_i} \nonumber \\
& \leq & \meio m \frac{\xi_0^2}{s_{\min}^2}\norma{\delta_x}^2 + 
\alpha_{\mu}\rho^k m \frac{\xi_0}{s_{\min}} \norma{\delta_x} \nonumber \\
& \leq & \frac{m \xi_0^2}{2 s_{\min}}\norma{\delta_x}^2 + \frac { \alpha_{\mu}
\rhomax^{k-1} m} {s_{\min}} \norma{g_p^k}\norma{\delta_x} \nonumber \\
& \leq & \frac{m \xi_0^2}{2 s_{\min} \mu_1^2}\norma{ g_A(\xc{k}) }^2
+ \frac{\alpha_{\mu}\rhomax^0 m \theta_p}{s_{\min}\mu_1} \norma{g_A(\xc{k})}^2.
\visiblelbl{eq:qdhns}
\end{eqnarray}
Assim, juntando \eqref{eq:qdhnx} com \eqref{eq:qdhns}, temos
\begin{eqnarray}
 q(\delta_A) & = & q_x(\delta_x) + q_s(\delta_s) \nonumber \\
& \leq & \bigg[-\frac{1}{2\mu_2} + \frac{m\xi_0^2}{2s_{\min}\mu_1^2} +
\frac{\alpha_{\mu}\rhomax^0 m \theta_p}{s_{\min}\mu_1}\bigg]\norma{g_A(\xc{k})}^2 \nonumber \\
& = & \mu_3\norma{g_A(\xc{k})}^2, \visiblelbl{eq:qdhn}
\end{eqnarray}
onde $\mu_3$ é o termo entre colchetes na segunda linha.

Agora, usando uma expansão de Taylor, temos
\begin{eqnarray}
 \DLH{+} & = & L(\zc{k}+\Lac{k}\delta_A^k,\lk{k},\muc{k}) -
  L(\zc{k},\lk{k},\muc{k}) \nonumber \\
 & = & \nabla_z L(\zc{k},\lk{k},\muc{k})^T\Lac{k}\delta_A^k + \meio
  (\delta_A^k)^T \Lac{k}\nabla_{zz}^2 L(\zc{k},\lk{k},\muc{k})\Lac{k}\delta_A^k
  + o(\norma{\Lac{k}\delta_A^k}^2) \nonumber \\
 & = & (\gpk{k})^T\delta_A^k + \meio (\delta_A^k)^T
  W(\zc{k},\lk{k},\muc{k})\delta_A^k + o(\norma{\delta_A^k}^2).
  \visiblelbl{lemma.horz.aux1}
\end{eqnarray}
Lembrando que $A(z_c^k)\delta_A^k=0$, o primeiro termo de
\eqref{lemma.horz.aux1} se expande em
\begin{eqnarray}
  (g_p^k)^T\delta_A^k & = & [g(z_c^k,\mu^k) + A(z_c^k)^T\lambda^k]^T
  \delta_A^k \nonumber \\
  & = & g(z_c^k,\mu^k)^T\delta_A^k + (\lambda^k)^TA(z_c^k)\delta_A^k \nonumber
  \\
  & = & q(\delta_A^k) - \meio (\delta_A^k)^TB^k\delta_A^k. 
  \nonumber
\end{eqnarray}
Substituindo isso em \eqref{lemma.horz.aux1}, temos
\begin{eqnarray}
\DLH{+} 
& = & q(\delta_A^k) + \meio (\delta_A^k)^T[W(\zc{k}, \lk{k}, \muc{k}) -
  B^k]\delta_A^k + o(\norma{\delta_A^k}^2).
  \visiblelbl{lemma.horz.aux2}
\end{eqnarray}
Usando o fato de que $\delta_x^k = P(\xc{k})\delta_x^k$, \eqref{eq:PBWlk},
\eqref{dxgAk} e \eqref{dsgAk}, o segundo termo de \eqref{lemma.horz.aux2} vira
\begin{eqnarray}
  (\delta_A^k)^T [W(z_c^k,\lambda^k,\mu^k) - B^k] \delta_A^k 
  & = & (\delta_x^k)^T [\nabla_{xx}^2\mathcal{L}(x_c^k,\lambda^k) - B_x^k]
    \delta_x^k + (\delta_s^k)^T (\mu I - B_s^k) \delta_s^k \nonumber \\
  & = & (\delta_x^k)^TP(x_c^k) [\nabla_{xx}^2 \mathcal{L}(x_c^k,\lambda^k) -
  B_x^k] \delta_x^k + (\delta_s^k)^T(\mu I - B_s^k) \delta_s^k \nonumber \\
  & = & o(\norma{\delta_x^k}^2) + o(\norma{\delta_s^k}^2) \nonumber \\
  & = & o(\norma{\delta_A^k}^2).
  \visiblelbl{lemma.horz.aux3}
\end{eqnarray}
Agora, substituindo \eqref{lemma.horz.aux3} em \eqref{lemma.horz.aux2}, e usando
\eqref{eq:dhnlim} obtemos
\begin{eqnarray}
\DLH{+} 
& = & q(\delta_A^k) + o(\norma{\delta_A^k}^2) \nonumber \\
& = & q(\delta_A^k) + o(\norma{g_A(\xc{k})}^2). \visiblelbl{eq:DLHp}
\end{eqnarray}
Segue de \eqref{eq:qdhn} e \eqref{eq:DLHp} que
\begin{eqnarray*}
 r & = & \frac{ \DLH{+} }{ q(\delta_A^k) } = 1 + \frac{ o(\norma{g_A(\xc{k})}^2) }
{q(\delta_A^k)} \\
& \geq & 1 + \frac{1}{\mu_3}\frac{ o(\norma{g_A(\xc{k})}^2) }{ \norma{g_A(\xc{k})}^2 }.
\end{eqnarray*}
Lembrando que $\eta_1 \in(0,1)$, para $k$ suficientemente
grande, temos
$$ \bigg|\frac{ o(\norma{g_A(\xc{k})}^2) }{ \norma{g_A(\xc{k})}^2 }\bigg| \leq 
\mu_3(1 - \eta_1), $$
logo,
\begin{eqnarray*}
 r & \geq & 1 - \frac{1}{\mu_3}\mu_3(1 - \eta_1) = \eta_1,
\end{eqnarray*}
de modo que uma das condições da linha \ref{alg:tangente-condicoes} do Algoritmo \ref{alg:outline}
 é satisfeita para $k$ suficientemente grande.

Por \eqref{limLac}, \eqref{eq:dhnlim}, e a Hipótese \ref{hip:local.active-relations}, temos
\begin{eqnarray}
 \norma{\Lambda_c^k\delta_A^k} & \leq & \norma{\Lambda_c^k} \norma{\delta_A^k} \nonumber \\
& \leq & \xi_0\theta_4\norma{g_A(\xc{k})} \nonumber \\
& \leq & \xi_0\theta_4\theta_A\norma{g_p^k}. \visiblelbl{eq:aux.difz}
\end{eqnarray}
Então, de \eqref{eq:aux.difz} e a Hipótese \ref{hip:local.lls}, concluímos que
as hipóteses da terceira parte do Lema \ref{lemma:35} são satisfeitas, de modo
que existe $k_0$ suficientemente grande tal que $\rhomax^k = \rhomax^{k_0}$, $k
\geq k_0$. 
Assim, usando \eqref{limrhomax} e \eqref{limgamma}, para $k \geq k_0$, obtemos
\begin{eqnarray*}
 \norma{g_p^k} \leq 10^4\rho^k\frac{\norma{g(\zc{k},\muc{k})} + 1}{\rhomax^k}
\leq 10^4\rho^k\frac{\xi_0 + 1}{\rhomax^{k_0}} = \beta \rho^k,
\end{eqnarray*}
onde  $\beta = 10^4(1+\xi_0)/\rhomax^{k_0}$. 
Junto com \eqref{limvarh}, \eqref{eq:dhnlim}, a 
Hipótese \ref{hip:local.active-relations}, 
\eqref{limhhornormal}, isto implica que, para $k$ suficientemente grande,
\begin{eqnarray}
 \norma{h(\zc{k} + \Lac{k}\delta_A^k)} & \leq & \norma{h(\zc{k})} + 
\norma{h(\zc{k}+\Lac{k}\delta_A^k) - h(\zc{k})} \nonumber \\
& \leq & \norma{h(\zc{k})} + \bxi_0\norma{\delta_A^k}^2 \nonumber \\
& \leq & \norma{h(\zc{k})} + \bxi_0\theta_4^2\norma{\gAk{k}}^2 \nonumber \\
& \leq & \norma{h(\zc{k})} + \bxi_0\theta_4^2\theta_A^2\norma{\gpk{k}}^2 \nonumber \\
& \leq & \rho^k + \beta\bxi_0\theta_4^2\theta_A^2\norma{\gpk{k}}\rho^k \nonumber \\
& = & \rho^k (1 + \beta\bxi_0\theta_4^2\theta_A^2\norma{\gpk{k}} ). \nonumber
\end{eqnarray}

Assim para $k$ suficientemente grande, $\beta\bxi_0 \theta_4^2
\theta_A^2\norma{\gpk{k}} < 1$, de modo que o passo $\delta_A^k$ será aceito.
\end{proof}

Até agora, pudemos trabalhar com um passo normal genérico que satisfizesse
poucas condições. Doravante, vamos considerar um formato para o passo normal que
nos ajudará na demonstração da convergência local.

\begin{hypoenv}\visiblelbl{hip:local.normal.step}
Para $k$ suficientemente grande, cada passo normal $\delta_N^{k+1} =
\zc{k+1}-z^k$ é calculado tomando um ou mais passos da forma
\begin{eqnarray}
 \delta_N^+ = -J^+h(z_c) = -J^T(JJ^T)^{-1}h(z_c),\visiblelbl{def:normalstep}
\end{eqnarray}
onde $J$ satisfaz
\begin{eqnarray}
 \norma{J - \nabla h(z_c)} = \bigo(\norma{\gpk{k}}). \visiblelbl{eq:Abarra}
\end{eqnarray}
\end{hypoenv}
Note que, para $k$ suficientemente grande, usando \eqref{eq:svd}, podemos
redefinir $\theta_3$ de modo que 
$\norma{J^T\lambda} \geq \theta_3\norma{\lambda}$.

Usando uma expansão de Taylor, \eqref{eq:svd}, \eqref{def:normalstep}, \eqref{eq:Abarra},
\eqref{nnormalxi} e a 
continuidade de $A(z)$, é fácil mostrar que, se $\zc{k+1} \neq z^k$, para $k$
suficientemente grande, então o primeiro passo 
normal da iteração $k+1$, digamos $\delta_N^+$, satisfaz
\begin{eqnarray}
 \norma{\delta_N^+} & = & \norma{J^+ h(z_c)} \leq
  \frac{1}{\theta_3}\norma{h(z_c)} = \bigo(\norma{h(z_c)})
\end{eqnarray}
e
\begin{eqnarray}
 \norma{h(\zc{k+1})} & \leq & \norma{h(z^k + \delta_N^+)} \nonumber \\
& = & \norma{h(z^k) + \nabla h(z^k)\delta_N^+ + o(\norma{\delta_N^+})} \nonumber \\
& = & \norma{h(z^k) + J\delta_N^+ + [\nabla h(z^k)-J]\delta_N^+ + 
o(\norma{\delta_N^+})} \nonumber \\
& = &  \norma{ h(z^k) - JJ^+h(z^k) + [ \nabla h(z^k) - J] \delta_N^+ + o(\norma{\delta_N^+})} \nonumber \\
& \leq & \norma{\nabla h(z^k) - J}\norma{\delta_N^+} + o(\norma{\delta_N^+}) \nonumber \\
& = & \bigo (\norma{\gpk{k}}\norma{\delta_N^+}) + o(\norma{\delta_N^+}) \nonumber \\
& = & o(\norma{\delta_N^+}) \nonumber \\
& = & o(\norma{h(z^k)}). \visiblelbl{eq:hzckp}
\end{eqnarray}

No próximo lema, iremos mostrar que os valores de $\norma{g_A(x)}$ e $\norma{c_A(x)}$ definem 
uma medida de distância ao ponto $x^*$.
\begin{lemma}
 Numa vizinhança $V^*$ de $x^*$, temos
\begin{eqnarray}
 \norma{x - x^*} = \Theta(\norma{c_A(x)} + \norma{g_A(x)}). \visiblelbl{eq:difxTheta}
\visiblelbl{eq:difzTheta}
\end{eqnarray}
\end{lemma}
\begin{proof}
Como $g_A(x)$ e $c_A(x)$ são Lipschitz contínuas, temos
$$ \norma{g_A(x)} = \norma{g_A(x) - g_A(x^*)} = \bigo(\norma{x-x^*}),$$
e
$$ \norma{c_A(x)} = \norma{c_A(x) - c_A(x^*)} = \bigo(\norma{x-x^*}).$$
Então
  $$\norma{c_A(x)} + \norma{g_A(x)} = \bigo(\norma{x-x^*}).$$

Agora consideremos o Lagrangeano do problema de igualdade associado às
restrições ativas,
$$L_A(x,\lambda) = f(x) + c_A(x)^T\lambda,$$
cujas derivadas são
$$ \nabla L_A(x,\lambda) = 
\vetor{\nabla f(x) + \nabla c_A(x)^T\lambda}{c_A(x)}$$
e
$$\nabla^2 L_A(x,\lambda) = 
\matriz{H_A(x,\lambda)}{\nabla c_A(x)^T}{\nabla c_A(x)}{0}.$$
Nesse caso, $\nabla L_A(x^*,\lk{*}) = 0$ e $\nabla^2 L_A(x^*, \lk{*})$ é
não-singular.
Definamos $e_x = x^* - x, e_\lambda = \lambda^* - \lambda$ e
$e = \vetor{e_x}{e_\lambda}$.
Pelo Corolário \ref{app:teo-medida}, temos
$$ \norma{e} = 
\bigo(\norma{\nabla L_A(x,\lambda)}). $$
Daí,
\begin{align*}
\norma{x - x^*} 
  & \leq \bigg\Vert \vetor{x - x^*}
    {\lambda_A(x) - \lambda^*}\bigg\Vert \\
  & = \bigo(\norma{\nabla L_A(x,\lambda_A(x))}) \\
  & = \bigo(\norma{g_A(x)} + \norma{c_A(x)}).
\end{align*}
\end{proof}

Apresentamos agora o teorema de convergência local de nosso algoritmo. Obtivemos
os mesmos resultados que o artigo original.
\begin{theorem}
 Com as hipóteses \ref{hip:global.continuidade}-\ref{hip:local.normal.step}, $x^k$ e $\xc{k}$ são superlinearmente convergentes em dois 
passos para $x^*$. Se uma restauração é calculada a cada iteração, então $x^k$ converge 
superlinearmente para $x^*$.
\end{theorem}
\begin{proof}
 As expressões \eqref{dxgAk} e \eqref{eq:difzTheta} implicam que
\begin{eqnarray}
 \norma{x^{k+1}-x^*} & \leq & \norma{x^{k+1} - \xc{k+1}} + \norma{\xc{k+1}-x^*} \nonumber \\
 & = & \norma{\delta_x^{k+1}} + \norma{\xc{k+1} - x^*} \nonumber \\
 & = & \bigo(\norma{\gAk{k+1}}) + \norma{\xc{k+1}-x^*} \nonumber \\
 & = & \bigo(\norma{\xc{k+1} - x^*}). \visiblelbl{eq:limdifxkxe}
\end{eqnarray}
Além disso, pela Hipótese \ref{hip:local.active-relations} e \eqref{eq:difzTheta}, temos
\begin{eqnarray}
 \norma{\xc{k+1} - x^*} & \leq & \norma{x^k - \xc{k+1}} + \norma{x^k - x^*} \nonumber \\
 & = & \bigo(\norma{c_A(x^k)}) + \norma{x^k - x^*} \nonumber \\
 & = & \bigo(\norma{x^k - x^*}). \visiblelbl{eq:limdifxckxe}
\end{eqnarray}
Para mostrar a convergência local, precisamos das relações a seguir:
\begin{eqnarray}
 g_A(x^k) & = & o(\norma{\xc{k} - x^*}), \visiblelbl{eq:limgAxk} \\
 g_A(\xc{k+1}) & = & o(\norma{\xc{k-1} - x^*}), \visiblelbl{eq:limgAxck} \\
 c_A(x^k) & = & o(\norma{\xc{k-1} - x^*}), \visiblelbl{eq:limcxk} \\
 c_A(\xc{k+1}) & = & o(\norma{\xc{k} - x^*}). \visiblelbl{eq:limcxck}
\end{eqnarray}

Vamos prová-las, começando por \eqref{eq:limgAxk}. Para uma vizinhança $V^*$ de $x^*$, usando a 
expansão de Taylor, temos
\begin{eqnarray}
 \norma{g_A(x^k)} & = & \norma{P(x^k)g_A(x^k)} \nonumber \\
                  & = & \norma{P(x^k)[g_A(\xc{k}) + \nabla g_A(\xc{k})\delta_x^k]} + o(\norma{\delta_x^k}) 
\nonumber \\
& = & \norma{P(x^k)\zeta_k} + o(\norma{\delta_x^k}) \nonumber \\
& \leq & \norma{[P(x^k) - P(\xc{k})]\zeta_k} + \norma{P(\xc{k})\zeta_k} +
o(\norma{\delta_x}),
\visiblelbl{eq:gAxk.aux1}
\end{eqnarray}
onde $\zeta_k = g_A(\xc{k}) + \nabla g_A(\xc{k})\delta_x^k$.
Pela continuidade de $P(x)$ em $V^*$, \eqref{limjacob} e \eqref{dxgAk}, temos
\begin{eqnarray}
 \norma{[P(x^k) - P(\xc{k})]\zeta_k} & = & \bigo(\norma{\xc{k} - x^k}\norma{\zeta_k}) = 
o(\norma{\zeta_k}) \nonumber \\
 & = & o(\norma{g_A(\xc{k})}) + o(\norma{\nabla g_A(\xc{k})}\norma{\delta_x^k}) \nonumber \\
 & = & o(\norma{g_A(\xc{k})}) + o(\norma{\delta_x^k}) \nonumber \\
 & = & o(\norma{g_A(\xc{k})}).\visiblelbl{eq:gAxk.aux2}
\end{eqnarray}
Além disso, usando o fato de que $P(\xc{k})\nabla c_A(\xc{k})^T = 0$, \eqref{eq:solnu}, 
\eqref{eq:PBWlls} e \eqref{dxgAk}, temos
\begin{eqnarray}
 \norma{P(\xc{k})\zeta_k} & \leq & \norma{P(\xc{k})[g_A(\xc{k}) + B_x^k\delta_x^k]} + \nonumber 
\\
& & \norma{P(\xc{k})[H_A(\xc{k},\lambda_A(\xc{k})) - B_x^k]\delta_x^k} + \nonumber \\
& & \norma{P(\xc{k})[H_A(\xc{k},\lambda_A(\xc{k})) - \nabla g_A(\xc{k})]\delta_x^k} \nonumber 
\\
& = & o(\norma{\delta_x^k}) + \norma{P(\xc{k})\nabla c_A(\xc{k})^T\nabla 
\lambda_A(\xc{k})\delta_x^k} \nonumber \\
& = & o(\norma{\delta_x^k}) = o(\norma{g_A(\xc{k})}). \visiblelbl{eq:gAxk.aux3}
\end{eqnarray}
Assim, substituindo \eqref{eq:gAxk.aux2} e \eqref{eq:gAxk.aux3} em \eqref{eq:gAxk.aux1}, e 
usando \eqref{eq:difzTheta}, obtemos
\begin{eqnarray*}
 \norma{g_A(x^k)} = o(\norma{g_A(\xc{k})}) = o(\norma{\xc{k} - x^*}).
\end{eqnarray*}
Para provar \eqref{eq:limcxck}, precisamos considerar duas situações separadamente. Primeiro, 
suporemos que $\{k_i\}$ seja uma subsequência infinita em que nenhum passo normal é feito, i.e. 
$\zc{k_i+1} = z^{k_i}$. 
Nesse caso, a dinâmica do controle da factibilidade, \eqref{limrho}, a Hipótese 
\ref{hip:local.active-relations} e \eqref{eq:limgAxk} implicam que
\begin{eqnarray}
 \norma{c_A(\xc{k_i+1})} & = & \bigo(\rho^{k_i+1}) = \bigo(\norma{\gpk{k_i+1}}) \nonumber \\
 & = & \bigo(\norma{g_A(\xc{k_i+1})}) = \bigo(\norma{g_A(x^{k_i})}) = o(\norma{\xc{k_i}-
x^*}). \visiblelbl{eq:cxck.aux1}
\end{eqnarray}
Agora, vamos considerar uma subsequência infinita $\{k_j\}$, onde pelo menos um
passo normal $\delta_N^+$ satisfaz a Hipótese \ref{hip:local.normal.step}.
Nesse caso, as hipóteses \ref{hip:local.active-relations} e
\ref{hip:local.normal.step}, \eqref{eq:difzTheta} e \eqref{eq:limdifxkxe}
implicam que
\begin{eqnarray}
 \norma{c_A(\xc{k_j + 1})} & = & \bigo(h(\zc{k_j+1})) = \bigo(\norma{h(z^{k_j} +
\delta_N^+)}) 
\nonumber \\
& = & o(\norma{h(z^{k_j})}) = o(\norma{c_A(x^{k_j})}) \nonumber \\
& = & o(\norma{x^{k_j} - x^*}) = o(\norma{\xc{k_j} - x^*}). \visiblelbl{eq:cxck.aux2}
\end{eqnarray}
A expressão \eqref{eq:limcxck} segue de \eqref{eq:cxck.aux1} e \eqref{eq:cxck.aux2}.

Combinando a Hipótese \ref{hip:local.active-relations}, o Lema
\ref{lemma:horz.step.accepted}, \eqref{limvarh}, \eqref{eq:hzckp},
\eqref{eq:limdifxkxe}, \eqref{eq:difzTheta}, \eqref{eq:dhnlim} e
\eqref{gpmukplus}, podemos escrever
\begin{eqnarray*}
 \norma{c_A(x^k)} & = & \bigo(\norma{h(z^k)}) = \bigo(\norma{h(\zc{k})}) + \bigo(\norma{h(z^k) 
- h(\zc{k})}) \nonumber \\
& = & o(\norma{h(z^{k-1})}) + \bigo(\norma{\delta_A^k}^2) \nonumber \\
& = & o(\norma{c_A(x^{k-1})}) + \bigo(\norma{\delta_A^k}^2) \nonumber \\
& = & o(\norma{x^{k-1} - x^*}) + \bigo(\norma{\delta_A^k}^2) \nonumber \\
& = & o(\norma{\xc{k-1} - x^*}) + \bigo(\norma{\delta_A^k}^2) \nonumber \\
& = & o(\norma{\xc{k-1} - x^*}) + \bigo(\norma{\gAk{k}}^2) \nonumber \\
& = & o(\norma{\xc{k-1} - x^*}) + \bigo(\norma{\gAk{k-1}}^2) \nonumber \\
& = & o(\norma{\xc{k-1} - x^*}) + \bigo(\norma{\xc{k-1} - x^*}^2) \nonumber \\
& = & o(\norma{\xc{k-1} - x^*}).
\end{eqnarray*}
Finalmente, para obter \eqref{eq:limgAxck}, usamos uma expansão de Taylor, a Hipótese 
\ref{hip:local.active-relations}, \eqref{eq:difxTheta}, 
\eqref{eq:limgAxk}, \eqref{eq:limcxk}, \eqref{eq:limdifxckxe} e \eqref{eq:limdifxkxe},
de modo que
\begin{eqnarray*}
 \norma{g_A(\xc{k+1})} & = & \norma{g_A(x^k)} + \bigo(\norma{\xc{k+1} - x^k}) \\
 & = & \norma{g_A(x^k)} + \bigo(\norma{c_A(x^k)}) \\
 & = & o(\norma{\xc{k-1} - x^*}).
\end{eqnarray*}
A convergência em dois passos segue utilizando \eqref{eq:difzTheta}, \eqref{eq:limgAxk}, 
\eqref{eq:limcxk}, \eqref{eq:limdifxckxe}, \eqref{eq:limdifxkxe}, \eqref{eq:limgAxck} e 
\eqref{eq:limcxck}, pois
\begin{eqnarray}
 \norma{x^{k+1} - x^*} & = & \bigo(\norma{g_A(x^{k+1})} + \norma{c_A(x^{k+1})}) \nonumber \\
 & = & o(\norma{\xc{k+1} - x^*}) + o(\norma{\xc{k} - x^*}) \nonumber \\
 & = & o(\norma{x^{k-1} - x^*}),\visiblelbl{eq:2stepxk}
\end{eqnarray}
e
\begin{eqnarray}
 \norma{\xc{k+1} - x^*} & = & \bigo(\norma{g_A(\xc{k+1})} + \norma{c_A(\xc{k+1})}) \nonumber \\
 & = & o(\norma{\xc{k-1} - x^*}) + o(\norma{\xc{k} - x^*}) \nonumber \\
 & = & o(\norma{\xc{k-1} - x^*}).\visiblelbl{eq:2stepxck}
\end{eqnarray}

Para concluir a prova, vamos supor que uma restauração não-nula é feita em cada
iteração. Nesse caso, 
a Hipótese \ref{hip:local.active-relations}, \eqref{limvarh}, \eqref{eq:hzckp},
 \eqref{eq:difzTheta}, \eqref{eq:dhnlim} e 
\eqref{eq:limdifxckxe} nos permitem melhorar \eqref{eq:limcxk}, obtendo
\begin{eqnarray}
 \norma{c_A(x^k)} & \leq & \norma{c_A(\xc{k})} + \norma{c_A(x^k) - c_A(\xc{k})} \nonumber \\
 & = & o(\norma{c_A(x^{k-1})} + \bigo(\norma{\delta_A^k}^2) \nonumber \\
 & = & o(\norma{x^{k-1} - x^*}) + \bigo(\norma{g_A(\xc{k})}^2) \nonumber \\
 & = & o(\norma{x^{k-1} - x^*}) + \bigo(\norma{\xc{k} - x^*}^2) \nonumber \\
 & = & o(\norma{x^{k-1} - x^*}). \visiblelbl{eq:cA1step}
\end{eqnarray}
Substituindo \eqref{eq:limgAxk} e \eqref{eq:cA1step} em \eqref{eq:2stepxk}, temos
\begin{eqnarray*}
 \norma{x^{k+1} - x^*} & = & \bigo(\norma{g_A(x^{k+1})} + \norma{c_A(x^{k+1})}) \nonumber \\
 & = & o(\norma{\xc{k+1} - x^*}) + o(\norma{x^k - x^*}) = o(\norma{x^k - x^*}).
\end{eqnarray*}
\end{proof}

\section{Convergência para Pontos Estacionários da Infactibilidade}
\visiblelbl{sec:conv-infactivel}

Algumas vezes, em nosso algoritmo, não é possível encontrar um ponto dentro do cilindro
de confiança. Quando isso acontece, supomos que o problema é infactível, e 
desejamos que nosso método não faça iterações desnecessárias.
Os teoremas de convergência anteriores consideram que a restauração não falha,
isto é, que o passo normal consegue encontrar um ponto $z_c^k$ tal que
$\norma{h(z_c^k)} \leq \rho^k$. Nesta seção, vamos mostrar que, se isso não é
possível, então pelo menos o algoritmo deve convergir para um ponto estacionário
da infactibilidade do problema \eqref{prob:geral}.

Como apresentamos anteriormente, o problema normal é dado por
\begin{equation}\tag{\ref{prob:normal}}
\begin{array}{rl}
  \min & \meio\norma{h(z)}^2 \\
  \mbox{suj. a} & s \geq 0.
\end{array}
\end{equation}
Já o problema da infactibilidade é
\begin{eqnarray}
 \min & & \norma{c_E(x)}^2 + \norma{c_I^-(x)}^2, \visiblelbl{prob:infac}
\end{eqnarray}
onde $v^- = (\min\{0,v_1\}, \min\{0,v_2\}, \dots, \min\{0,v_n\})^T$.
\begin{theorem}\visiblelbl{teo:infac}
  Seja $\{\znj\}$ uma sequência gerada pela algoritmo normal com pontos de
  acumulação estacionários para o problema \eqref{prob:normal}, com
  $$\znj = \vetor{\xnj}{\snj}.$$
  Nesse caso, as componentes $\xnj$ de cada elemento dessa sequência formam uma
  sequência com pontos de acumulação estacionários para o problema da
  infactibilidade \eqref{prob:infac}.
\end{theorem}
\begin{proof}
  Seja $z^*$ um ponto de acumulação de $\{\znj\}$ estacionário para o problema
  \eqref{prob:normal}. Então, pelas condições KKT, existe $w^*\in\Rn{m_I}$ tal que
\begin{align*}
  \nabla h(z^*)^Th(z^*) - \vetor{0}{w^*} & = \vetor{0}{0}, \\
  w_i^*s_i^* & = 0, \\
  w^*, s^* & \geq 0.
\end{align*}
Como
\begin{eqnarray*}
  \nabla h(z)^T h(z) & = & \matriz{\nabla c_E(x)^T}{\nabla
    c_I(x)^T}{0}{-I}\vetor{c_E(x)}{c_I(x) - s} \\
  & = & \vetor{ \nabla c_E(x)^Tc_E(x) + \nabla c_I(x)^T[c_I(x) - s]}{s - c_I(x)},
\end{eqnarray*}
temos
\begin{align}
  \nabla c_E(x^*)c_E(x^*) + \nabla c_I(x^*)[c_I(x^*) - s^*] & = 0,
  \visiblelbl{eq:infac.kkt} \\
  s_i^* - c_{I_i}(x^*) & = w_i^*. \nonumber
\end{align}
Para mostrar que $x^*$ é um ponto estacionário do problema de
infactibilidade \eqref{prob:infac}, basta mostrar que 
$$c_{I_i}(x^*) - s_i^* = c_{I_i}^-(x^*), \qquad i = 1,\dots,m_I. $$
Para cada $s_i^*$, temos duas opções:
\begin{itemize}
 \item Se $s_i^* > 0$, então $w_i^* = 0$. Daí, $ c_{I_i}(x^*) = s_i^* > 0$, e
   portanto
   $$c_{I_i}(x^*) - s_i^* = 0 = \min\{ c_{I_i}(x^*), 0\}.$$
 \item Se $s_i^* = 0$, então $w_i^* \geq 0$. Daí, $c_{I_i}^* = -w_i^* \leq 0$, e
   portanto
   $$c_{I_i}(x^*) - s_i^* = c_{I_i}(x^*) = \min\{c_{I_i}(x^*),0\}.$$
\end{itemize}
De uma maneira ou de outra, temos
$$ \nabla c_I(x^*)^T[c_I(x^*) - s^*] = \nabla c_I(x^*)^Tc_I^-(x^*). $$
Substituindo essa expressão em \eqref{eq:infac.kkt}, obtemos
$$ \nabla c_E(x^*)^T c_E(x^*) + \nabla c_I(x^*)^T[c_I(x^*) - s_i^*] =
\nabla c_E(x^*)^T c_E(x^*) + \nabla c_I(x^*)^T c_I^-(x^*) = 0, $$
de modo que $x^*$ é ponto estacionário do problema \eqref{prob:infac}.
\end{proof}

