\chapter{Introdução}

Vários problemas práticos podem ser formulados como problemas de otimização não
linear. Das várias áreas de engenharia às teorias físicas, muitos problemas
requerem a obtenção de um minimizador ou maximizador. 
Em muitos casos, é possível considerar propriedades específicas do problema e
reduzir a complexidade do modelo, obtendo um problema de otimização com
características especiais, como são os problemas de programação linear, inteira,
quadrática; os problemas de quadrados mínimos, lineares e não lineares; os
problemas convexos, irrestritos; e assim por diante.
No entanto, quando não podemos fazer essa simplificação, é necessário enfrentar
o problema de otimização não linear geral diretamente. Nesta tese, iremos
apresentar um novo método para a solução desse problema.
Nosso método é uma extensão do Método de Controle Dinâmico da
Infactibilidade \cite{bib:chico-dci}, originalmente desenvolvido para o problema
com restrições apenas de igualdade.
O método CDI (do inglês ``Dynamic Control of Infeasibility'') obtém seus
iterandos através de passos normais e tangentes, utilizando aproximações
quadráticas, e técnicas de pontos interiores.
Nossa estratégia de globalização é baseada no que chamamos de \emph{cilindros de
confiança}, que são regiões ao redor do conjunto factível, com raio proporcional
à norma do gradiente projetado.
Esses cilindros limitam a infactibilidade dos iterandos. 
A cada iteração definimos um cilindro pequeno de raio $\rho$ e um cilindro
grande de raio $2\rho$.
Definimos o passo tangente de modo a diminuir o valor da norma do gradiente
projetado, limitando o iterando ao cilindro grande. Prosseguimos então com um ou
mais passos normais, até que o iterando fique dentro do cilindro pequeno. 

Foi preciso decidir como lidar com as desigualdades do problema com cuidado. A
maneira mais direta, considerando o cilindro como uma penalidade das restrições,
não herdava as propriedades do método. 
Escolhemos a estratégia de pontos interiores, e ela se mostrou muito eficaz. A
convergência global do método não teve muitas modificações em relação ao
original com essa estratégia. Para a convergência local, foi preciso considerar
apenas as restrições ativas na solução. Isso gerou algumas mudanças nas
hipóteses e nos teoremas, mas conseguimos obter os mesmos resultados que o
método original.
Uma grande parte deste trabalho foi a implementação do método.
Sempre trabalhamos para suplantar o método original na robustez e na eficiência.
Implementamos nossas estruturas de dados e interfaces tentando aproveitar ao
máximo as ferramentas utilizadas.

Iniciaremos este trabalho com uma revisão do problema de programação não linear,
e de estratégias clássicas no Capítulo \ref{chap:pnl}.
Os detalhes dos
passos tangentes e normais serão mostrados no Capítulo \ref{chap:metodo}, assim
como o pseudo-código para o algoritmo. A teoria de convergência global é
mostrada na seção \ref{sec:conv-global}, e a de convergência local está na seção
\ref{sec:conv-local}. Também vamos falar sobre o comportamento do algoritmo
com problemas infactíveis na Seção \ref{sec:conv-infactivel}. No Capítulo
\ref{chap:implementacao} são mostrados
detalhes da implementação do algoritmo, no Capítulo \ref{chap:resultados}
mostramos os resultados computacionais obtidos com essa implementação, e no
Capítulo \ref{chap:conclusoes} apresentamos nossas considerações finais.
