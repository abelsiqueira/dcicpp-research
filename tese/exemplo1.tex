Para deixar mais claro o funcionamento do método, vamos aplicar o algoritmo à
dois exemplos. O primeiro é o problema
\begin{eqnarray}
  \min & f(x) = \meio(x_1^2 + x_2^2) \nonumber \\
  \mbox{s.t} & x_2 = x_1^2 + 1, \nonumber
\end{eqnarray}
cuja solução é o ponto $(1,0)$.

\newcommand{\addexum}[1]{
  \includegraphics[scale=1.0]{ex1_plots/ex1_img#1.pdf}
}

\begin{center}
  \addexum{1}
\end{center}

\begin{center}
  \begin{minipage}{0.9\textwidth}
  Começamos pelo ponto $x_0 = \vetor{2}{3}$, denotado pelo círculo sólido na imagem.
  A curva sólida representa a região factível, as curvas tracejadas 
  denotam o cilindro menor e as curvas pontilhadas denotam o cilindro
  maior. As circunferências concêntricas denotam as curvas de nível da função objetivo, cujo
  menor valor ocorre na origem.
\end{minipage}
\end{center}

\begin{center}
  \addexum{2}
\end{center}

\begin{center}
  \begin{minipage}{0.9\textwidth}
  Iniciamos notando que o ponto não está dentro do cilindro menor. Então
  fazemos um passo normal. O ponto obtido, denotado $x_c$, é a tentativa de
  iterando. Como o ponto já está dentro do cilindro
  menor, o passo é interrompido.
\end{minipage}
\end{center}

\begin{center}
  \addexum{3}
\end{center}

\begin{center}
  \begin{minipage}{0.9\textwidth}
  Agora, atualizamos o raio dos cilindros. Como o $x_c$ obtido continuou
  dentro do cilindro menor após o raio deste ter sido atualizado, definimos
  $x_c^0$ como este iterando normal.
\end{minipage}
\end{center}

\begin{center}
  \addexum{4}
\end{center}

\begin{center}
  \begin{minipage}{0.9\textwidth}
  A partir deste ponto, tentamos realizar um passo tangente. O raio da região de
  confiança, não ilustrado na imagem, é grande o suficiente para permitir que o
  passo seja tomado completamente. O passo aqui é o minimizador da aproximação
  quadrática do Lagrangeano. $x^+$. Nesse caso, como esse ponto está fora do
  cilindro maior, rejeitamos o passo e reduzimos a
  região de confiança.
\end{minipage}
\end{center}

\begin{center}
  \addexum{5}
\end{center}

\begin{center}
  \begin{minipage}{0.9\textwidth}
  Calculamos um novo passo tangente, desta vez limitado pela região de
  confiança, indicada pelo segmento tracejado. Note que, na verdade, este
  segmento faz parte da caixa que é a região de confiança.
  O iterando $x^+$ obtido está dentro do cilindro maior e fornece decréscimo
  suficiente, sendo portanto aceito como o próximo iterando.
\end{minipage}
\end{center}

\begin{center}
  \addexum{6}
\end{center}

\begin{center}
  \begin{minipage}{0.9\textwidth}
  $x^1$ obtido pelo passo tangente.
\end{minipage}
\end{center}
