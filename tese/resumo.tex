\begin{center}
  \large{\textbf{Resumo}}
\end{center}

Uma maneira de resolver problemas gerais de programação não linear é utilizar
estratégias de passos compostos. Essas estratégias normalmente combinam um passo
tangente às restrições e um passo normal, alternando entre a diminuição da função
objetivo e da norma da infactibilidade. Esse tipo de método exige o controle
dos passos ou dos iterandos, para que não se perca o progresso de um passo no outro.
Apresentaremos uma extensão do método de Controle Dinâmico da Infactibilidade, que
utiliza uma estratégia de controle de passos chamado de Cilindros de Confiança.
Esse método foi desenvolvido para problemas com restrições apenas de igualdade,
e nossa extensão lida com restrições gerais. Mostraremos testes numéricos comparando
nosso método com um método do mesmo tipo.

\vspace{.2cm}
\textbf{Palavras-chave } Programação Não-linear, Controle Dinâmico da Infactibilidade,
Métodos de Passos Compostos, Experimentos Numéricos.
