\documentclass[compress,10 pt]{beamer}
\usepackage[utf8]{inputenc}
\usepackage[T1]{fontenc}
\newif\ifpt
%\pttrue
\ifpt
\usepackage[portuguese]{babel}
\else
\usepackage[english]{babel}
\fi
\usepackage{ulem}
\usepackage{ae}
\newcommand{\epsmu}{\varepsilon_{\mu}}
\newcommand{\lls}{\lambda_{LS}}
\newcommand{\suja}{\iflanguage{portuguese}{suj. a}{s. t.}}
\usepackage{slidesdef}

\newcommand{\pten}[2]{\iflanguage{portuguese}{#1}{#2}}

\newcommand{\mytitle}{
  \iflanguage{portuguese}{
    Controle Dinâmico de Inviabilidade para Programação Não Linear
  }{
    Dynamic Control of Infeasibility for Nonlinear Programming
  }
}
\newcommand{\myshorttitle}{
  \iflanguage{portuguese}{
    CDI para PNL
  }{
    DCI for NP
  }
}
\newcommand{\myauthor}{ Abel Soares Siqueira \\ Francisco A. M. Gomes}
\newcommand{\myshortauthor}{ Abel S. Siqueira \quad Francisco A. M. Gomes}

\newcommand{\mydate}{
\iflanguage{portuguese}{
  17 de Março de 2014
}{
  March, 17, 2014
}
}

\title{\myshorttitle}
\author{\myshortauthor}
\date{\mydate}

\begin{document}

\begin{frame}
  \begin{center}
  \begin{beamercolorbox}[center,rounded=true,shadow=true]{mybox}
    \bf \Large \mytitle
  \end{beamercolorbox}

  \vspace{0.5cm}

  \begin{beamercolorbox}[center,rounded=true,shadow=true]{mybox}
    \bf \myauthor
  \end{beamercolorbox}
  \mydate

  \vspace{0.5cm}
  \begin{center}
    {\color{mydarkcolor} \small
\iflanguage{portuguese}{
  Universidade Estadual de Campinas \\
    Campinas - São Paulo - Brasil
  }
  {
    State University of Campinas \\
    Campinas - São Paulo - Brazil
  }
  }
  \end{center}
  \vspace{0.3cm}

  \begin{center}
    {\color{mydarkcolor}
    {\scriptsize
      \iflanguage{portuguese}{
        Apoio financeiro \\
        CNPq (desde 12/2013) \\
        Fapesp (2009-2013). Processo 2009/17273-4
      }{
        Financial support \\
        CNPq (since 12/2013) \\
        Fapesp (2009-2013). Reference 2009/17273-4
      }
    }
  }
  \end{center}
  \end{center}
\end{frame}

\makesection{Introdução}{Introduction}

\makesubsection{Objetivos}{Objectives}

\myframe{
\iflanguage{portuguese}{
  \begin{itemize}
    \item Objetivos:
    \begin{itemize}
      \item Estender o Controle Dinâmico de Inviabilidade proposto por
        \citet*{bib:francisco} para problemas de igualdades para tratar das
        desigualdades;
      \item Implementar um software C++ com interface CUTEr, disponível online;
    \end{itemize}
  \end{itemize}
}{
  \begin{itemize}
    \item Objectives:
    \begin{itemize}
      \item Extend the Dynamic Control of Infeasibility method proposed by
        \citet*{bib:francisco} for equality problems to handle inequalities;
      \item Implement a C++ software with {\color{red}\sout{CUTEr}} CUTEst
        interface, available online;
    \end{itemize}
  \end{itemize}
}
}

\makesubsection{Problema}{Problem}

\myframetop{
  \ctr{}
    \begin{equation}\visiblelbl{prob:geral}\tag{P}
    \begin{array}{rrcl}
    \min & f(x) \\
  \iflanguage{portuguese}{ \mbox{suj. a}}{\mbox{s.t.} }
    & c_E(x) &   =  & 0, \\
                  & c_I(x) & \geq & 0,
    \end{array}
    \end{equation}
}

\myframetop{
\iflanguage{portuguese}{
  \ctr{Introduzindo variáveis de folga}
}{
  \ctr{Introducing slack variables}
}
 \begin{equation*}\visiblelbl{prob:igual}
 \begin{array}{rrcl}
  \min & f(x) \\
   \iflanguage{portuguese}{\mbox{suj. a}}{\mbox{s.t.} }
  & c_E(x) & =  & 0, \\
	      & c_I(x) - s & = & 0, \\
             & s   & \geq & 0.
 \end{array}
 \end{equation*}
 \pause
  \vspace{0.5 cm}
  $$ z = \vetor{x}{s} \qquad h(z) = \vetor{c_E(x)}{c_I(x) - s} $$

  \vspace{0.5 cm}
  $$ \beta(z) \mbox{
  \iflanguage{portuguese}{barreira da fronteira}{boundary barrier}} $$
  \pause
  $$ \beta(z) = -\sum_{i=1}^{m_I} \ln(s_i) $$
}

\myframetop{
\iflanguage{portuguese}{
  \ctr{ Problema de restrição de igualdade com barreira logarítmica }
}{
  \ctr{ Equality contrained problem with logarithmic barrier }
}
  \begin{equation*}
  \begin{array}{rl}
  \min & \varphi(z,\mu) = f(x) + \mu\beta(z) \\
  \mbox{s.t.} & h(z) = 0,
  \end{array}
  \end{equation*}
  \pause
  \vspace{0.5cm}
\iflanguage{portuguese}{
  \ctr{ Função Lagrangiana }
}{
  \ctr{ Lagrangian function }
}
  $$L(z,\lambda,\mu) = \varphi(z,\mu) + \sum_{i=1}^mh_i(z)\lambda_i$$
}

\makesubsection{ Passos Compostos }{Composite Steps}

\myframe{
\iflanguage{portuguese}{
  \ctr{Passos Compostos}
  \begin{itemize}
    \item Passo Normal $\longrightarrow$ Factibilidade Primal
    \item Passo Tangente $\longrightarrow$ Factibilidade Dual
  \end{itemize}
}{
  \ctr{Composite Steps}
  \begin{itemize}
    \item Normal Step $\longrightarrow$ Primal Feasibility
    \item Tangent Step $\longrightarrow$ Dual Feasibility
  \end{itemize}
}
}

\myframetop{
  \ctr{ \pten{ Escoço }{ Outline } }
  \vfill
  \includegraphics[width=\textwidth]{plots/cilindro3d.pdf}
  \vfill
}

\makesubsection{ Passo Normal }{Normal Step}

\myframetop {
  \iflanguage{portuguese}{
    \ctr{ Passo Normal }
  }{
    \ctr{ Normal Step }
  }
  \begin{eqnarray*}
    \min & \meio \norma{h(z)}^2 \\
    \mbox{s.t.} & s \geq 0
  \end{eqnarray*}
}

\makesubsection{ Passo Tangente}{Tangent Step}

\myframetop {
  \iflanguage{portuguese}{
    \ctr{ Passo Tangente \only<2>{Escalado} }
  }{
    \ctr{ \onslide<2>{Scaled} Tangent Step }
  }
  \begin{eqnarray*}
    \min_d & & L(z_c +
    \only<2>{ {\color{red} \Lambda_c } }
    d, \lambda, \mu) \\
    \mbox{\suja} & & h(z_c +
    \only<2>{ {\color{red} \Lambda_c } }
    d) = h(z_c)
  \end{eqnarray*}
}


\makesubsection{Definições}{Definitions}

\myframe{
  \ctr{
    \iflanguage{portuguese}{
      Matriz de escalamento
    }{
      Scaling Matrix
    }
  $\Lambda(z) = \matriz{I}{0}{0}{S} $ }
\pause
\begin{align*}
  g(z,\mu) & = \Lambda(z)\nabla\varphi(z,\mu) \\
  A(z) & = \nabla h(z) \Lambda(z),\\
  \Gamma(z,\mu) & = \Lambda(z)\nabla^2\varphi(z,\mu)\Lambda(z) \\
  W(z,\lambda,\mu) & = \Lambda(z)\nabla_{zz}^2 L(z,\lambda,\mu)\Lambda(z)
\end{align*}
}


\myframe{
  \iflanguage{portuguese}{
    \ctr{ Multiplicadores de Lagrange }
  }{
    \ctr{ Lagrange Multipliers }
  }
 $$\lls(z,\mu) = \arg\min_{\lambda}\bigg\{\meio\norma{g(z,\mu) + A(z)^T\lambda}^2\bigg\}$$
 \vspace{0.3 cm}
 $$\lambda^k_i = \left\{
\begin{array}{lll}
 \lls(z_c^k,\mu^k)_i, & \mbox{se } i \in E \\
 \min\{\lls(z_c^k,\mu^k)_i, \alpha(\mu^k)^n\}, & \mbox{se } i \in I
\end{array}
\right.$$
}

\myframe {
  \iflanguage{portuguese}{
    \ctr{ Gradiente Projetado }
  }{
    \ctr{ Projected Gradient }
  }
 \begin{align*}
   g_p(z,\mu) & = g(z,\mu) + A(z)^T\lls(z,\mu). \\
   g_p^k & = g(z_c^k,\mu^k) + A(z_c^k)^T\lambda^k
 \end{align*}
}

\myframe {
  \iflanguage{portuguese}{
    \ctr{ Cilindro de Confiança }
  }{
    \ctr{ Trust Cylinder }
  }
  $$\mathcal{C}(\rho) = \{z \in \Rn{N}:\norma{h(z)}\leq\rho\}$$
  $$\rho^k = \bigo(\norma{g_p^k})$$
  \begin{center}
  \begin{tikzpicture}[domain=0:9,scale=0.8]
    %\draw[fill,gray] plot (\x,{0.5*\x*(\x-10)*sin(\x)+1}) --
    %plot (9-\x,{0.5*(9-\x)*(-\x-1)*sin(9-\x)-1}) -- cycle;
    \draw[black] plot (\x,{0.5*\x*(\x-10)*sin(\x)}) node[right] {$h(z) = 0$};
    \draw[red] plot (\x,{0.5*\x*(\x-10)*sin(\x)+1});
    \draw[red] plot (\x,{0.5*\x*(\x-10)*sin(\x)-1});
    \draw[<->] (4,-12*0.07) -- node[right] {$\rho$} (4,-12*0.07+1);
    \draw[<->] (5,-12.5*0.087) -- node[right] {$\rho$} (5,-12.5*0.087-1);
  \end{tikzpicture}
  \end{center}
}

\makesection{Método}{Method}

\myframe{
  \iflanguage{portuguese}{
    \ctr{ Passo Normal }
  }{
    \ctr{ Normal Step }
  }
  \begin{center}
    \iflanguage{portuguese}{
      \includegraphics{plots/flux-normal-pt.pdf}
    }{
      \includegraphics{plots/flux-normal-en.pdf}
    }
  \end{center}
}

\myframetop{
  \iflanguage{portuguese}{
    \ctr{ Passo Normal Interno }
    $$ \mbox{Como encontrar } z_c \in \mathcal{C}(\rho)? $$
    \pause
    \begin{center}
    Método proposto por \citet{bib:francisco}.
  }{
    \ctr{ Internal Normal Step }
    $$ \mbox{How do we find} z_c \in \mathcal{C}(\rho)? $$
    \pause
    \begin{center}
    Method proposed by \citet{bib:francisco}.
  }
  \begin{tikzpicture}
    \begin{axis}[view={0}{90}, axis lines=none, xmin=-2,xmax=5, ymin=-4,ymax=3,
      height=6cm, width=10cm]
      \addplot[mark=none,green,dashed] plot coordinates {
        (-2,-2) (-2,2) (2,2) (2,-2) (-2,-2)
      };
      \only<2>{
        \addplot[->] plot coordinates { (0,0) (3,-1) }
        node[pos=1, pin=30: $d_{cp}$] {};
      }
      \only<3->{
        \addplot[dashed,->] plot coordinates { (0,0) (3,-1) }
        node[pos=1, pin=30: $d_{cp}$] {};
        \addplot[->] plot coordinates { (0,0) (3*0.63, -1*0.63) }
        node[pos=1, pin=30: $P(d_{cp})$] {};
      }
      \only<4>{
        \addplot[->] plot coordinates { (0,0) (-0.2,-3) }
        node[pos=1, pin=150: $d_N$] {};
      }
      \only<5->{
        \addplot[->,dashed] plot coordinates { (0,0) (-0.2,-3) }
        node[pos=1, pin=150: $d_N$] {};
        \addplot[->] plot coordinates { (0,0) (-0.2*0.63,-3*0.63) }
        node[pos=1, pin=150: $P(d_N)$] {};
      }
      \only<6->{
        \addplot[->,blue] plot coordinates { (1.89, -0.63) (-0.126,-1.89) }
        node[pos=1, pin=-30: $v$] {};
      }
      \only<7->{
        \addplot[->,red] plot coordinates { (0,0) (0.882,-1.26) };
        \addplot[->,blue] plot coordinates { (1.89, -0.63) (0.882,-1.26) }
        node[pos=1, pin=-30: $tv$] {};
      }
    \end{axis}
  \end{tikzpicture}
  \end{center}
  \uncover<8->{
  $$ \psi(t) = \frac{ m(P(d_{cp}) + tv) - m(0) }{ m(P(d_{cp})) - m(0) } $$}
}

\makesubsection{ Passo Tangente }{Tangent Step}
\myframe{
  \iflanguage{portuguese}{
    \ctr{ Passo Tangente }
  }{
    \ctr{ Tangent Step }
  }
  \begin{equation}\visiblelbl{prob:tangent}
    \tag{TS}
    \begin{array}{rl}
    \min & q_k(\delta) = \meio \delta^TB^k\delta +
    \delta^Tg_p(z_c^k,\mu^k), \\
    \mbox{\suja} & A(z_c^k)\delta = 0 \\
     & \tilde{l} \leq \delta \leq \tilde{u},
    \end{array}
  \end{equation}
  \pause
  \iflanguage{portuguese}{
    Resolvido por uma modificação do método de Steihaug
    \cite{bib:steihaug}.
  }{
    Solved with a modification of Steihaug's method
    \cite{bib:steihaug}.
  }
}

\myframe{
  \iflanguage{portuguese}{
    \ctr{ Passo Tangente }
    \includegraphics{plots/flux-tangent-pt.pdf}
  }{
    \ctr{ Tangent Step }
    \includegraphics{plots/flux-tangent-en.pdf}
  }
}

\makesubsection{Esboço}{Outline}

\myframe {
  \iflanguage{portuguese}{
    \ctr{Esboço}
    \begin{center}
      \includegraphics{plots/flux-outline-pt.pdf}
    \end{center}
  }{
    \ctr{Outline}
    \begin{center}
      \includegraphics{plots/flux-outline-en.pdf}
    \end{center}
  }
}

\makesubsection{Exemplo}{Example}

\myframe{
  \iflanguage{portuguese}{
    \ctr{Exemplo de Igualdade}
  }{
    \ctr{Equality Example}
  }
\begin{eqnarray*}
 \min & f(x) = \meio(x_1^2 + x_2^2) \\
 \mbox{\suja} & x_2 = x_1^2 + 1
\end{eqnarray*}
 \vspace{1 cm}
 $$x^0 = \vetor{2}{3}$$
}

\myframe{
  \ctr{\pten{Exemplo de Igualdade}{Equality Example}}
  \begin{center}
  \begin{overlayarea}{0.7\textwidth}{0.7\textheight}
    \only<1>{\includegraphics[height=0.7\textheight]{plots/ex1_img1.pdf}}
    \only<2>{\includegraphics[height=0.7\textheight]{plots/ex1_img2.pdf}}
    \only<3>{\includegraphics[height=0.7\textheight]{plots/ex1_img3.pdf}}
    \only<4>{\includegraphics[height=0.7\textheight]{plots/ex1_img4.pdf}}
    \only<5>{\includegraphics[height=0.7\textheight]{plots/ex1_img5.pdf}}
    \only<6>{\includegraphics[height=0.7\textheight]{plots/ex1_img6.pdf}}
  \end{overlayarea}
  \end{center}
}

\myframe{
  \iflanguage{portuguese}{
    \ctr{Exemplo de Desigualdade}
  }{
    \ctr{Inequality Example}
  }
\begin{eqnarray*}
  \min & f(x) = \meio[x_1^2 + (x_2+1)^2] \\
 \mbox{\suja} & x_2 \geq x_1^2, \\
              & x_1+x_2 = 1.
\end{eqnarray*}
 \vspace{1 cm}
 $$x^0 = \vetor{1}{-1}$$
}

\myframe{
  \ctr{\pten{Exemplo de Desigualdade}{Inequality Example}}
  \begin{center}
  \begin{overlayarea}{0.7\textwidth}{0.7\textheight}
    \only<1>{\includegraphics[height=0.7\textheight]{plots/ex2_img1.pdf}}
    \only<2>{\includegraphics[height=0.7\textheight]{plots/ex2_img2.pdf}}
    \only<3>{\includegraphics[height=0.7\textheight]{plots/ex2_img3.pdf}}
    \only<4>{\includegraphics[height=0.7\textheight]{plots/ex2_img4.pdf}}
    \only<5>{\includegraphics[height=0.7\textheight]{plots/ex2_img5.pdf}}
    \only<6>{\includegraphics[height=0.7\textheight]{plots/ex2_img6.pdf}}
    \only<7>{\includegraphics[height=0.7\textheight]{plots/ex2_img7.pdf}}
    \only<8>{\includegraphics[height=0.7\textheight]{plots/ex2_img8.pdf}}
    \only<9>{\includegraphics[height=0.7\textheight]{plots/ex2_img9.pdf}}
    \only<10>{\includegraphics[height=0.7\textheight]{plots/ex2_img10.pdf}}
    \only<11>{\includegraphics[height=0.7\textheight]{plots/ex2_img11.pdf}}
    \only<12>{\includegraphics[height=0.7\textheight]{plots/ex2_img12.pdf}}
    \only<13>{\includegraphics[height=0.7\textheight]{plots/ex2_img13.pdf}}
    \only<14>{\includegraphics[height=0.7\textheight]{plots/ex2_img14.pdf}}
    \only<15>{\includegraphics[height=0.7\textheight]{plots/ex2_img15.pdf}}
  \end{overlayarea}
  \end{center}
}

\makesection{Convergência}{Convergence}

\makesubsection{ Convergência Global }{Global Convergence}

\myframe{
  \iflanguage{portuguese}{
    \ctr{ Hipóteses para Convergência Global }
    \begin{enumerate}
      \item[H1] $f$, $c_E$ e $c_I$ são $C^2$.
      \item[H2] As sequências $\{z_c^{k}\}$ e $\{z^k\}$,
        as aproximações $B^k$ e os multiplicadores $\{\lambda^k\}$ permanecem
        uniformemente limitados.
      \item[H3] A restauração nunca falha e $\mathcal{Z} =\{z_c^k\}$ permanece
        longe do conjunto singular de $h$, i.e., $h$ é regular no fecho da
        $\mathcal{Z}$. Além disso, se a sequência gerada
        $\{z_c^k\}$ é infinita, então
      \begin{equation}\visiblelbl{nvert}
      \norma{z_c^{k+1}-z^k} = \bigo(\norma{h(z^k)})
      \end{equation}
      \item[H4] $\norma{\delta_{soc}^k} = \bigo(\norma{\delta_t^k}^2)$
    \end{enumerate}
  }{
    \ctr{ Hypothesis for Global Convergence }
    \begin{enumerate}
      \item[H1] $f$, $c_E$ and $c_I$ are $C^2$.
      \item[H2] The sequences $\{z_c^{k}\}$ and $\{z^k\}$,
        the approximations $B^k$ and the multipliers $\{\lambda^k\}$ remain
        uniformly bounded.
      \item[H3] The restoration never fails and $\mathcal{Z} =\{z_c^k\}$ remains
        away from the singular set of $h$, i.e., $h$ is regular in the closure
        of $\mathcal{Z}$. Furthermore, if the sequence
        $\{z_c^k\}$ is infinite, then
      \begin{equation}\visiblelbl{nvert}
      \norma{z_c^{k+1}-z^k} = \bigo(\norma{h(z^k)})
      \end{equation}
      \item[H4] $\norma{\delta_{soc}^k} = \bigo(\norma{\delta_t^k}^2)$
    \end{enumerate}
  }
}

\myframe{
  \iflanguage{portuguese}{
    \ctr{Teorema: Convergência Global }
    {\it Sob as hipóteses H1-H4, DCI para em um ponto estacionário de
      \eqref{prob:geral} em um número finito de iterações, ou gera uma sequência
      com pontos estacionários em seu conjunto de acumulação. Além disso, se as
      condições
    \begin{enumerate}
      \item[C1] $\norma{z^k - z_c^k} = \bigo(\norma{g_p(z_c^k,\mu_c^k)})$
      \item[C2] $\norma{\lambda^k - \lls(z_c^k, \mu_c^k)} = \bigo(\norma{g_p(z_c^k,\mu_c^k)})$
      \item[C3] $(\lambda_I^{k+1})^T(s_c^{k+1} - s^k) =
        \bigo(\norma{g_p(z_c^k,\mu_c^k)}\rho^k)$
    \end{enumerate}
    são satisfeitas, então todo ponto de acumulação de $z_c^k$ é estacionário
    para
      \eqref{prob:geral}. }
  }{
    \ctr{Theorem: Global Convergence }
    {\it Under the assumptions H1-H4, DCI stops at a stationary point for
      \eqref{prob:geral} in a finite number of iterations, or generates an
      sequence with stationary points in its accumulation set.
      In addition, if the conditions
    \begin{enumerate}
      \item[C1] $\norma{z^k - z_c^k} = \bigo(\norma{g_p(z_c^k,\mu_c^k)})$
      \item[C2] $\norma{\lambda^k - \lls(z_c^k, \mu_c^k)} = \bigo(\norma{g_p(z_c^k,\mu_c^k)})$
      \item[C3] $(\lambda_I^{k+1})^T(s_c^{k+1} - s^k) =
        \bigo(\norma{g_p(z_c^k,\mu_c^k)}\rho^k)$
    \end{enumerate}
    are satisfied, then every accumulation point of $\{z_c^k\}$ is stationary
    for \eqref{prob:geral}. }
  }
}

\makesubsection{Convergência Local}{Local Convergence}

\myframe{
  \iflanguage{portuguese}{
    \ctr{ Hipóteses de Convergência Local }
    Para convergência local consideramos apenas as restrições ativas na solução
    e hipóteses sobre
    \begin{itemize}
      \item o ponto estacionário sendo um ``bom minimizador'';
      \item limitantes nas aproximações dos multiplicadores de Lagrange;
      \item a positividade das aproximações para a Lagrangiana da Hessiana;
      \item a equivalência das condições de otimalidade nas restrições ativas;
      \item a forma dos passos normal e tangente perto da solução.
    \end{itemize}
  }{
    \ctr{ Hypothesis for Local Convergence }
    For the local convergence we consider only the active constraints in the
    solution, and hypothesis about
    \begin{itemize}
      \item the stationary point being a ``good minimizer'';
      \item bounds on the approximations to the Lagrangian multipliers;
      \item positivity of the approximations for the Hessian of the Lagrangian;
      \item equivalence between the optimality conditions at the active
        constraints;
      \item form of the normal and tangent steps near the solution.
    \end{itemize}
  }
}

\myframe{
  \iflanguage{portuguese}{
    \ctr{Teorema: Convergência Local}
    {\it Sob as hipóteses H1-H4 e A1-A5, $x^k$ e $x_c^k$ são superlinearmente
      convergentes em 2 passos para $x^*$. Se uma restauração é feita a cada
    $x^k$, então $\{x^k\}$ converge superlinearmente para $x^*$.  }
  }{
    \ctr{Theorem: Local Convergence}
    {\it Under hypothesis H1-H4 and A1-A5, $\{x^k\}$ and $\{x_c^k\}$ are
      superlinearly convergent in 2 steps to $x^*$. If a restoration is made
      every iteration, then $\{x^k\}$ is superlinearly convergent to $x^*$.
    }
  }
}

\makesubsection{Problemas Infactíveis}{Infeasible Problems}

\myframe{
  \iflanguage{portuguese}{
    \ctr{ Hipóteses para Convergência Infactível }
    \begin{enumerate}
      \item[I1] A sequência gerada pelo algoritmo do passo normal é limitada.
      \item[I2] Seja $L$ um conjunto convexo, limitado e aberto contendo todos
        os pontos tentados pelo algoritmo do passo normal. Para todo $x,y \in
        L$, temos
      $$\norma{\nabla h(x) - \nabla h(y)} \leq 2\gamma_0\norma{x-y}.$$
    \end{enumerate}
  }{
    \ctr{Hypothesis for Infeasibility Convergence}
    \begin{enumerate}
      \item[I1] The sequence generated by the algorithm of the normal step is
        limited.
      \item[I2] Let $L$ se a convex set, limited and open, containing every
        point tried by the algorithm of the normal step. For each $x, y \in L$,
        we have
      $$\norma{\nabla h(x) - \nabla h(y)} \leq 2\gamma_0\norma{x-y}.$$
    \end{enumerate}
  }
}

\myframe{
  \iflanguage{portuguese}{
    \ctr{Teorema: Convergência Infactível}
    Sob as hipóteses H1-H4, I1, I2, se o algoritmo para o passo normal gera uma
    sequência infinita, então todo ponto limite desta sequência é estacionária
    para o problema
  }{
    \ctr{Theorem: Infeasibility Convergence}
    Under hypothesis H1-H4, I1 and I2, if the algorithm for the normal step
    generates an infinite sequence, then every limit point of this sequence is a
    stationary point for the problem
  }
  \begin{equation*}
   \min \norma{c_E(x)}^2 + \norma{c_I^-(x)}^2.
  \end{equation*}
}

\makesection{ Experimentos Numéricos }{Numerical Experiments}

\makesubsection{Implementação do Algoritmo}{Algorithm Implementation}

\myframe{
  \iflanguage{portuguese}{
    \ctr{ Implementação do Algoritmo }
    \begin{itemize}
      \item Uma implementação C++ do método, chamada DCICPP, foi criada.
      \item DCICPP foi construída sobre a biblioteca de Cholesky.
      \item Disponível online no Github, sob a licença GPL.
      \item Usou as seguintes bibliotecas
      \begin{itemize}
        \item CHOLMOD \cite{bib:cholmod5} (Cholesky);
        \item METIS \cite{bib:metis} (biblioteca de permutações para Cholesky);
        \item GotoBLAS2 \cite{bib:gotoblas}, OpenBLAS \cite{bib:openblas};
        \item base\_matrices (embrulho C++ para Cholesky);
        \item CUTEr \cite{bib:cuter} (teste);
      \end{itemize}
    \end{itemize}
  }{
    \ctr{Algorithm Implementation}
    \begin{itemize}
      \item A C++ implementation was created, called DCICPP.
      \item DCICPP was built over a Cholesky library.
      \item Available online on GitHub, under the GPL.
      \item We used
        \begin{itemize}
          \item CHOLMOD
            \cite{bib:cholmod1,bib:cholmod2,bib:cholmod3,bib:cholmod4,bib:cholmod5}
            (Cholesky);
          \item METIS \cite{bib:metis} (permutation library used by CHOLMOD);
          \item OpenBLAS \cite{bib:openblas};
          \item base\_matrices (C++ wrapper for Cholesky);
          \item CUTEst \cite{bib:cute,bib:cuter} (testing);
          \item perprof-py \cite{bib:perprof} (a performance profile tool).
        \end{itemize}
    \end{itemize}
  }
}

\myframe {
  \ctr{ \pten{Experimentos Numéricos}{Numerical Experiments} }
  \begin{itemize}
    \item \pten{514 problemas do CUTEst}{514 CUTEst problems}.
    \item \pten{Problemas com variáveis fixas ou ``muito pequenos'' foram
      removidos.} {Problems with fixed variables or ``too small'' were removed.}
    \item \pten{XXX}{Subset of problems where the Jacobian has full rank, and
      small and medium problems}
  \end{itemize}
}

\subsection{ Perfil de Desempenho }

\myframe {
  \begin{center}
    \ctr{ \pten{Jacobianas não singulares}{Full rank Jacobians} }
    \includegraphics[height=0.7\textheight]{profs/nozero_cholok_nofix.pdf}
  \end{center}
}
\myframe {
  \begin{center}
    \ctr{ \pten{Problemas pequenos e médios com Jacobianas não singulares}
      {Small and medium problems with full rank Jacobians} }
    \includegraphics[height=0.7\textheight]{profs/nozero_cholok_small_medium_slacktrue.pdf}
  \end{center}
}
\myframetop {
  \begin{center}
    \ctr{ \pten{Impacto das Jacobianas singulares}
      {Comparison of small and medium problems regarding singular Jacobians} }
    \includegraphics[width=0.45\textwidth]{profs/nozero_cholok_small_medium_slacktrue.pdf}
    \includegraphics[width=0.45\textwidth]{profs/nozero_small_medium_slacktrue.pdf}
  \end{center}
}
\myframetop {
  \begin{center}
    \ctr{ \pten{Impacto das Jacobianas singulares}
      {Comparison of all problems regarding singular Jacobians} }
    \includegraphics[width=0.45\textwidth]{profs/nozero_cholok_nofix.pdf}
    \includegraphics[width=0.45\textwidth]{profs/nozero.pdf}
  \end{center}
}

\makesection{ Próximos Passos }{Next Steps}

\myframe{
  \ctr{ \pten{Próximos Passos}{Next Steps} }
  \begin{itemize}
    \item \pten{Implementar suporte para variáveis fixas;}{Implement support for
      fixed variables;}
    \item \pten{Melhorar estratégia para Jacobianas singulares;}{Improve
      strategies for singular Jacobians;}
    \item \pten{Implementar uma versão com o Método dos Gradientes Conjugados
      ao menos para problemas muito grandes;}{Implement a version with the
      Conjugate Gradient method at least for large problems;}
    \item \pten{Investigar cada problema com falha procurando por um possível solução
      geral;}{Investigate each failing problem for a possible general solution;}
    \item \pten{Investigar como torná-lo mais eficiente.}{Improve efficiency
      overall.}
  \end{itemize}
}

\makesection{Referências}{References}

\begin{frame}[allowframebreaks]
 \small
 \bibliographystyle{plainnat}
 \bibliography{relbib}
 \end{frame}

\myframe {
  %\fcolorbox{mydarkestcolor}{mydarkcolor}
  \begin{center}
  \begin{beamercolorbox}[center,rounded=true,shadow=true]{mybox}
    \bf \Huge \pten{Obrigado}{Thank you}
  \end{beamercolorbox}
  \end{center}
}

\end{document}
