\documentclass{article}
\usepackage[utf8]{inputenc}
\usepackage[T1]{fontenc}
\usepackage{float}
\usepackage{caption}
\usepackage{subcaption}

\usepackage{graphicx}
\usepackage[top=1cm,bottom=2cm,right=1cm,left=1cm]{geometry}

\renewcommand{\emph}[1]{\textbf{#1}}

\title{2014 - 03 - 06}
\author{}
\date{}

\begin{document}
\maketitle
\section{Atualização do trabalho}

Apresentamos bons resultados no brazopt, e reapresentamos eles aqui, com a
adição de algumas dicas obtidas no congresso.

Durante o congresso conversamos com algumas pessoas que aparentemente usaram o
CUTEst com o IPOPT. Na verdade, a comparação foi feita com a versão para CUTEr
do IPOPT. Eu não concordo muito com essa abordagem, mas também a fiz.

\begin{figure}[H]
  \centering
  \includegraphics[width=0.9\textwidth]{plots/all_2014_04_16.pdf}
  \caption{Todos os problemas utilizados (Sem variáveis fixas). Tempos menores
    que 0.005 foram considerados como 0.005. - 874 problemas}
\end{figure}
\begin{figure}[H]
  \centering
  \includegraphics[width=0.9\textwidth]{plots/set_2014_04_16.pdf}
  \caption{Sem variáveis fixas, nem problemas
    que tiveram 0.00 no ALGENCAN - 514 problemas}
\end{figure}
\begin{figure}[H]
  \centering
  \includegraphics[width=0.9\textwidth]{plots/fullrank_2014_04_16.pdf}
  \caption{Sem variáveis fixas, nem problemas
    que tiveram 0.00 no ALGENCAN, e com Jacobiana de posto completo - 364
    problemas}
\end{figure}
\begin{figure}[H]
  \centering
  \includegraphics[width=0.9\textwidth]{plots/nolarge_2014_04_16.pdf}
  \caption{Sem variáveis fixas, nem problemas
    que tiveram 0.00 no ALGENCAN, e sem problemas grandes - 444 problemas}
\end{figure}
\begin{figure}[H]
  \centering
  \includegraphics[width=0.9\textwidth]{plots/fullrank_nolarge_2014_04_16.pdf}
  \caption{Sem variáveis fixas, nem problemas
    que tiveram 0.00 no ALGENCAN, e sem problemas grandes e com Jacobiana de
    posto completo - 304 problemas}
\end{figure}

\end{document}
