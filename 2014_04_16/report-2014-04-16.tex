\documentclass{article}
\usepackage[utf8]{inputenc}
\usepackage[T1]{fontenc}
\usepackage{float}
\usepackage{caption}
\usepackage{subcaption}
\usepackage[numbers,sort&compress]{natbib}

\usepackage{graphicx}
\usepackage[top=1cm,bottom=2cm,right=1cm,left=1cm]{geometry}

\renewcommand{\emph}[1]{\textbf{#1}}

\title{2014 - 03 - 06}
\author{}
\date{}

\begin{document}
\maketitle
\section{Atualização do trabalho}

Apresentamos bons resultados no brazopt, e reapresentamos eles aqui, com a
adição de algumas dicas obtidas no congresso.

Durante o congresso conversamos com algumas pessoas que aparentemente usaram o
CUTEst com o IPOPT. Na verdade, a comparação foi feita com a versão para CUTEr
do IPOPT. Eu não concordo muito com essa abordagem, mas também a fiz.

\begin{figure}[H]
  \centering
  \includegraphics[width=0.9\textwidth]{plots/all_xf_2014_04_16.pdf}
  \caption{Todos os problemas utilizados (Sem variáveis fixas). Tempos menores
    que 0.005 foram considerados como 0.005. - 874 problemas}
\end{figure}
\begin{figure}[H]
  \centering
  \includegraphics[width=0.9\textwidth]{plots/set_xf_2014_04_16.pdf}
  \caption{Sem variáveis fixas, nem problemas
    que tiveram 0.00 no ALGENCAN - 514 problemas}
\end{figure}
\begin{figure}[H]
  \centering
  \includegraphics[width=0.9\textwidth]{plots/fullrank_xf_2014_04_16.pdf}
  \caption{Sem variáveis fixas, nem problemas
    que tiveram 0.00 no ALGENCAN, e com Jacobiana de posto completo - 364
    problemas}
\end{figure}
\begin{figure}[H]
  \centering
  \includegraphics[width=0.9\textwidth]{plots/nolarge_xf_2014_04_16.pdf}
  \caption{Sem variáveis fixas, nem problemas
    que tiveram 0.00 no ALGENCAN, e sem problemas grandes - 444 problemas}
\end{figure}
\begin{figure}[H]
  \centering
  \includegraphics[width=0.9\textwidth]{plots/fullrank_nolarge_xf_2014_04_16.pdf}
  \caption{Sem variáveis fixas, nem problemas
    que tiveram 0.00 no ALGENCAN, e sem problemas grandes e com Jacobiana de
    posto completo - 304 problemas}
\end{figure}

\subsection{Usando os valores ótimos de função e factibilidade}

Segundo recomendação em \cite{bib:compare-optimal-values}, vamos reavaliar os
problemas segundo um critério diferente.
Ao invés de considerarmos a saída do algoritmo segundo o critério adotado, vamos
considerar apenas os valores da função objetivo e das inviabilidades primal e
dual para determinar se um problema convergiu, em relação aos outros, e assim
determinar o tempo que levou para convergir.

Dado um conjunto $S$ de algoritmos, e um conjunto $P$ de problemas, determinamos
que um algoritmo $s \in S$ é factível para o problema $p \in P$ se o máximo
entre as inviabilidades primal e dual é menor que uma tolerância $\varepsilon$.
Neste caso, escolhemos $\varepsilon = 10^{-4}$.
Sendo $f_{s,p}$ o valor da função objetivo obtido na solução, definimos
\newcommand{\fbest}{f_{\mbox{best},p}}
$$ \fbest = \min_s\{f_{s,p} | s \mbox{é factível para } p\}. $$
Dizemos que $s$ encontrou uma solução para $p$ se, $s$ é factível para $p$, e
$$ f_{s,p} \leq \fbest + 10^{-3}\vert \fbest \vert + 10^{-6}. $$
O perfil de desempenho é feito como antes, mas agora considerando a definição de
convergência acima.

\begin{figure}[H]
  \centering
  \includegraphics[width=0.9\textwidth]{plots/all_ov_2014_04_16.pdf}
  \caption{Todos os problemas utilizados (Sem variáveis fixas). Tempos menores
    que 0.005 foram considerados como 0.005. - 874 problemas}
\end{figure}
\begin{figure}[H]
  \centering
  \includegraphics[width=0.9\textwidth]{plots/set_ov_2014_04_16.pdf}
  \caption{Sem variáveis fixas, nem problemas
    que tiveram 0.00 no ALGENCAN - 514 problemas}
\end{figure}
\begin{figure}[H]
  \centering
  \includegraphics[width=0.9\textwidth]{plots/fullrank_ov_2014_04_16.pdf}
  \caption{Sem variáveis fixas, nem problemas
    que tiveram 0.00 no ALGENCAN, e com Jacobiana de posto completo - 364
    problemas}
\end{figure}
\begin{figure}[H]
  \centering
  \includegraphics[width=0.9\textwidth]{plots/nolarge_ov_2014_04_16.pdf}
  \caption{Sem variáveis fixas, nem problemas
    que tiveram 0.00 no ALGENCAN, e sem problemas grandes - 444 problemas}
\end{figure}
\begin{figure}[H]
  \centering
  \includegraphics[width=0.9\textwidth]{plots/fullrank_nolarge_ov_2014_04_16.pdf}
  \caption{Sem variáveis fixas, nem problemas
    que tiveram 0.00 no ALGENCAN, e sem problemas grandes e com Jacobiana de
    posto completo - 304 problemas}
\end{figure}

\bibliographystyle{siam}
\bibliography{bib}

\end{document}
