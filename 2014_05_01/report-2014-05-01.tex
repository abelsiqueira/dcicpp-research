\documentclass{article}
\usepackage[utf8]{inputenc}
\usepackage[T1]{fontenc}
\usepackage{float}
\usepackage{caption}
\usepackage{subcaption}
\usepackage[numbers,sort&compress]{natbib}

\usepackage{graphicx}
\usepackage[top=1cm,bottom=2cm,right=1cm,left=1cm]{geometry}

\renewcommand{\emph}[1]{\textbf{#1}}

\title{2014 - 05 - 01}
\author{}
\date{}

\newcommand{\ftol}{f_{\mbox{tol}}}

\begin{document}
\maketitle
\section{Atualização do trabalho}

Aqui são apenas os resultados anteriores.

\subsection{Normal}

\begin{figure}[H]
  \centering
  \includegraphics[width=0.9\textwidth]{plots/all_xf_2014_04_16.pdf}
  \caption{Todos os problemas utilizados (Sem variáveis fixas). Tempos menores
    que 0.005 foram considerados como 0.005. - 874 problemas}
\end{figure}
\begin{figure}[H]
  \centering
  \begin{subfigure}{0.48\textwidth}
    \includegraphics[width=\textwidth]{plots/set_xf_2014_04_16.pdf}
    \caption{Sem variáveis fixas, nem problemas
      que tiveram 0.00 no ALGENCAN - 514 problemas}
  \end{subfigure}
  \begin{subfigure}{0.48\textwidth}
    \includegraphics[width=\textwidth]{plots/fullrank_xf_2014_04_16.pdf}
    \caption{Sem variáveis fixas, nem problemas
      que tiveram 0.00 no ALGENCAN, e com Jacobiana de posto completo - 364
      problemas}
  \end{subfigure}
  \begin{subfigure}{0.48\textwidth}
    \includegraphics[width=\textwidth]{plots/nolarge_xf_2014_04_16.pdf}
    \caption{Sem variáveis fixas, nem problemas
      que tiveram 0.00 no ALGENCAN, e sem problemas grandes - 444 problemas}
  \end{subfigure}
  \begin{subfigure}{0.48\textwidth}
    \includegraphics[width=\textwidth]{plots/fullrank_nolarge_xf_2014_04_16.pdf}
    \caption{Sem variáveis fixas, nem problemas
      que tiveram 0.00 no ALGENCAN, e sem problemas grandes e com Jacobiana de
      posto completo - 304 problemas}
  \end{subfigure}
\end{figure}

\subsection{Aceitando apenas aqueles que o mesmo $f$. }

Agora vamos separar os problemas que, quando convergem, convergem para o mesmo
valor de função.

Definindo $S$ como o conjunto de algoritmos e $P$ como o conjunto de problemas,
definimos como $f_{s,p}$ o valor da função objetivo do algoritmo $s \in S$ para
o problema $p \in P$ na solução obtida.
Definimos
$$ \overline{f}_{p} = \min\{f_{s,p} : s \mbox{ converge para } p\}. $$
Aceitamos um problema $p$ para comparação se, para cada algoritmo $s$ que
converge para $p$, temos
$$ f_{s,p} < \overline{f} + \ftol\vert\overline{f}\vert + f_0, $$
onde $\ftol$ é o erro percentual aceito, e $f_0$ é uma tolerância para quando o
valor ótimo deveria ser $0$.
Os valores sugeridos em \cite{bib:compare-optimal-values} são $\ftol=10^{-3}$ e
$f_0 = 10^{-6}$.

Segue abaixo os resultados específicos.

\subsection{Para $\ftol = 5e-03$ e $f_0 = 1e-05$}

\begin{figure}[H]
  \centering
  \begin{subfigure}{0.48\textwidth}
    \includegraphics[width=\textwidth]{plots/samef_5e-03_1e-05_set_xf_2014_06_25.pdf}
    \caption{samef-5e-03-1e-05-set - 622}
  \end{subfigure}
  \begin{subfigure}{0.48\textwidth}
    \includegraphics[width=\textwidth]{plots/samef_5e-03_1e-05_fullrank_xf_2014_06_25.pdf}
    \caption{samef-5e-03-1e-05-fullrank - 435}
  \end{subfigure}
  \begin{subfigure}{0.48\textwidth}
    \includegraphics[width=\textwidth]{plots/samef_5e-03_1e-05_nolarge_xf_2014_06_25.pdf}
    \caption{samef-5e-03-1e-05-nolarge - 516}
  \end{subfigure}
  \begin{subfigure}{0.48\textwidth}
    \includegraphics[width=\textwidth]{plots/samef_5e-03_1e-05_fullrank_nolarge_xf_2014_06_25.pdf}
    \caption{samef-5e-03-1e-05-fullrank-nolarge - 342}
  \end{subfigure}
\end{figure}

\subsubsection{Só fixas}
\begin{figure}[H]
  \centering
  \begin{subfigure}{0.48\textwidth}
    \includegraphics[width=\textwidth]{plots/samef_5e-03_1e-05_onlyfix_xf_2014_06_25.pdf}
    \caption{samef-5e-03-1e-05-onlyfix - 207}
  \end{subfigure}
  \begin{subfigure}{0.48\textwidth}
    \includegraphics[width=\textwidth]{plots/samef_5e-03_1e-05_fullrank_onlyfix_xf_2014_06_25.pdf}
    \caption{samef-5e-03-1e-05-fullrank-onlyfix - 152}
  \end{subfigure}
\end{figure}

\subsubsection{Sem fixas}
\begin{figure}[H]
  \centering
  \begin{subfigure}{0.48\textwidth}
    \includegraphics[width=\textwidth]{plots/samef_5e-03_1e-05_nofix_xf_2014_06_25.pdf}
    \caption{samef-5e-03-1e-05-nofix - 415}
  \end{subfigure}
  \begin{subfigure}{0.48\textwidth}
    \includegraphics[width=\textwidth]{plots/samef_5e-03_1e-05_fullrank_nofix_xf_2014_06_25.pdf}
    \caption{samef-5e-03-1e-05-fullrank-nofix - 283}
  \end{subfigure}
\end{figure}



\bibliographystyle{siam}
\bibliography{bib}

\end{document}
