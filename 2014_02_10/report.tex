\documentclass{article}
\usepackage[utf8]{inputenc}
\usepackage[T1]{fontenc}
\usepackage{float}

\usepackage{graphicx}
\usepackage[top=3cm,bottom=3cm,right=2cm,left=2cm]{geometry}

\begin{document}

\section{Atualização do trabalho}

Os objetivos das últimas duas semanas eram: instalar os softwares \verb+algencan+ e
\verb+ipopt+ com o CUTEst, fazer novos testes com estes dois pacotes e o
\verb+dcicpp+, e separar os problemas por dimensões.
Infelizmente, descobrimos que o IPOPT ainda não tem uma versão com o CUTEst.
Dessa maneira, fizemos a separação dos problemas, e utilizamos para fazer a
comparação da versão antiga dos testes.
Os resultados estão a seguir

Separamos, inicialmente, os números de variáveis e restrições em 3 categorias:
\begin{itemize}
  \item $[0,10^2]$,
  \item $(10^2,10^4]$,
  \item $(10^4,10^6]$.
\end{itemize}
Combinando essas categorias, temos 9 possibilidades. Essas possibilidades são
suficientes para todos os problemas do CUTEst.

Para facilitar a descrição, e como as categorias são diferentes, vamos denominar
por $nvar$ o número de variáveis e $ncon$ o número de restrições, e
redefinir os intervalos como
\begin{itemize}
  \item $[0,10^2] \rightarrow \leq 10^2$,
  \item $(10^2,10^4]$,
  \item $(10^4,10^6] \rightarrow > 10^4$.
\end{itemize}

A tabela a seguir mostra a quantidade de problemas para cada categoria:
\begin{center}
\begin{tabular}{|l|c|c|c|} \hline
& $ncon \leq 10^2$ & $ncon \in (10^2,10^4]$ & $ncon > 10^4$ \\ \hline
$nvar \leq 10^2$       & 479 &  38 &  2 \\ \hline
$nvar \in (10^2,10^4]$ & 187 & 329 &  0 \\ \hline
$nvar > 10^4$          &  14 &  25 & 74 \\ \hline
\end{tabular}
\end{center}
Mas temos que levar em conta que os testes iniciais só rodavam em 767 problemas.
Muitos dos problemas acima não são rodados. Os valores Com essa retrição são
\begin{center}
\begin{tabular}{|l|c|c|c|} \hline
& $ncon \leq 10^2$ & $ncon \in (10^2,10^4]$ & $ncon > 10^4$ \\ \hline
$nvar \leq 10^2$       & 432 &  31 &  1 \\ \hline
$nvar \in (10^2,10^4]$ & 124 & 146 &  0 \\ \hline
$nvar > 10^4$          &   9 &  16 &  8 \\ \hline
\end{tabular}
\end{center}

As Figuras \ref{fig:nvar_small}-\ref{fig:nvar_big} mostram os perfis de
desempenho para cada linha dessa tabela.

\begin{figure}[H]
\centering
\includegraphics[width=0.3\textwidth]{var1e0-1e2_con1e0-1e2_2014_02_10.pdf}
\includegraphics[width=0.3\textwidth]{var1e0-1e2_con1e2-1e4_2014_02_10.pdf}
\includegraphics[width=0.3\textwidth]{var1e0-1e2_con1e4-1e6_2014_02_10.pdf}
\caption{ $nvar \leq 10^2$. }
\label{fig:nvar_small}
\end{figure}
\begin{figure}[H]
\centering
\includegraphics[width=0.3\textwidth]{var1e2-1e4_con1e0-1e2_2014_02_10.pdf}
\includegraphics[width=0.3\textwidth]{var1e2-1e4_con1e2-1e4_2014_02_10.pdf}
\includegraphics[width=0.3\textwidth]{var1e2-1e4_con1e4-1e6_2014_02_10.pdf}
\caption{ $nvar \in (10^2,10^4]$. }
\label{fig:nvar_medium}
\end{figure}
\begin{figure}[H]
\centering
\includegraphics[width=0.3\textwidth]{var1e4-1e6_con1e0-1e2_2014_02_10.pdf}
\includegraphics[width=0.3\textwidth]{var1e4-1e6_con1e2-1e4_2014_02_10.pdf}
\includegraphics[width=0.3\textwidth]{var1e4-1e6_con1e4-1e6_2014_02_10.pdf}
\caption{ $nvar > 10^4$. }
\label{fig:nvar_big}
\end{figure}

Vemos que o \verb+dcicpp+ se destaca para problemas médios. Então vale a pena
investigar nessa vizinhança.

A Figura \ref{fig:both_medium_and_big} mostra o perfil de desempenho para os
problemas com número de variáveis e de restrições maiores que 100.
A Figura \ref{fig:nvar_medium_and_big} mostra o perfil de desempenho para os
problemas com número de variáveis maiores que 100, e qualquer quantidade de
restrições.
A Figura \ref{fig:nvar_medium_and_big_not_unc} mostra o perfil de desempenho
para os problemas com número de variáveis maiores que 100, e ao menos uma
restrição.

\begin{figure}[H]
\centering
\includegraphics[width=0.7\textwidth]{vargt1e2_congt1e2_2014_02_10.pdf}
\caption{ $nvar > 10^2$ e $ncon > 10^2$ }
\label{fig:both_medium_and_big}
\end{figure}

\begin{figure}[H]
\centering
\includegraphics[width=0.7\textwidth]{vargt1e2_conany_2014_02_10.pdf}
\caption{ $nvar > 10^2$ }
\label{fig:nvar_medium_and_big}
\end{figure}

\begin{figure}[H]
\centering
\includegraphics[width=0.7\textwidth]{vargt1e2_notunc_2014_02_10.pdf}
\caption{ $nvar > 10^2$ e apenas problemas com restrições }
\label{fig:nvar_medium_and_big_not_unc}
\end{figure}

\subsection {CUTEst}

Novos testes usando o CUTEst. Infelizmente o IPOPT não tem versão com CUTEst
ainda.

\begin{figure}[H]
  \centering
  \includegraphics[width=0.7\textwidth]{all_2014_02_12.pdf}
  \caption{ Todos small (767 problemas) }
\end{figure}

\begin{figure}[H]
\centering
\includegraphics[width=0.3\textwidth]{var1e0-1e2_con1e0-1e2_2014_02_12.pdf}
\includegraphics[width=0.3\textwidth]{var1e0-1e2_con1e2-1e4_2014_02_12.pdf}
\includegraphics[width=0.3\textwidth]{var1e0-1e2_con1e4-1e6_2014_02_12.pdf}
\caption{ $nvar \leq 10^2$. }
\end{figure}
\begin{figure}[H]
\centering
\includegraphics[width=0.3\textwidth]{var1e2-1e4_con1e0-1e2_2014_02_12.pdf}
\includegraphics[width=0.3\textwidth]{var1e2-1e4_con1e2-1e4_2014_02_12.pdf}
\includegraphics[width=0.3\textwidth]{var1e2-1e4_con1e4-1e6_2014_02_12.pdf}
\caption{ $nvar \in (10^2,10^4]$. }
\end{figure}
\begin{figure}[H]
\centering
\includegraphics[width=0.3\textwidth]{var1e4-1e6_con1e0-1e2_2014_02_12.pdf}
\includegraphics[width=0.3\textwidth]{var1e4-1e6_con1e2-1e4_2014_02_12.pdf}
\includegraphics[width=0.3\textwidth]{var1e4-1e6_con1e4-1e6_2014_02_12.pdf}
\caption{ $nvar > 10^4$. }
\end{figure}

\begin{figure}[H]
\centering
\includegraphics[width=0.7\textwidth]{vargt1e2_congt1e2_2014_02_12.pdf}
\caption{ $nvar > 10^2$ e $ncon > 10^2$ }
\end{figure}

\begin{figure}[!ht]
\centering
\includegraphics[width=0.7\textwidth]{vargt1e2_conany_2014_02_12.pdf}
\caption{ $nvar > 10^2$ }
\end{figure}

\begin{figure}[!ht]
\centering
\includegraphics[width=0.7\textwidth]{vargt1e2_notunc_2014_02_12.pdf}
\caption{ $nvar > 10^2$ e apenas problemas com restrições }
\end{figure}

\end{document}
